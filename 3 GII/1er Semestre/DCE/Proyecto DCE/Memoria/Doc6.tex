%Doc6
\chapter{Diseño}
\section{Estructura}
\subsection{Programa principal}
Para el desarrollo de los códigos, seguimos una estructura tal que, a partir del archivo \hyperlink{laberinto}{Robot.ino}, se acceda al resto de librerías que controlarán el robot.

En la librería \hyperlink{Robot}{Robot.hpp} se encuentra todo lo necesario para definir los pines y las funciones necesarias para el control de los sensores y motores.

\subsection{Celdas}
Para el mapeado de las celdas utilizamos la librería \hyperlink{Celda}{celda.hpp}.

\subsection{Sensores}
Para el control de los sensores, utilizamos una librería por cada tipo de sensor, de tal forma que tenemos tres librerías: \hyperlink{sUltrasonidos}{sUltrasonidos.hpp}, \hyperlink{sInfrarrojos}{sInfrarrojos.hpp} y \hyperlink{sCNY}{sCNY.hpp}.

\subsection{Actuadores}
Para el control de los motores hemos usado la librería \hyperlink{Motor}{motor.hpp}.

\section{Plan de pruebas}
%Pones el bicho y miras con el serial blueooth lo que va midiendo el nota.
\subsection{Primeras pruebas}
Al principio comenzamos con las \hyperlink{primerasPruebas}{pruebas individuales} de cada sensor, de los motores y de la conexión bluetooth usando los códigos de las prácticas de la asignatura para comprobar su correcto funcionamiento.

\subsection{Pruebas finales}
Una vez que los sensores estuvieron calibrados, pasamos a hacer \hyperlink{pruebasFinales}{pruebas en conjunto} con todos los sensores activos para ir corrigiendo los pequeños fallos que iban surgiendo.