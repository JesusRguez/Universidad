%Doc2
\chapter{Especificación de requisitos}
\section{Requisitos funcionales}
El robot es capaz de avanzar hacia delante, situarse en el centro de una casilla, realizar giros de $90º$ hacia izquierda y derecha.

También es capaz de realizar movimientos en función de la información recabada por los sensores en cada celda. Por lo tanto, está capacitado para salir del laberinto con la información proporcionada por sus propios sensores.

A la vez que va recorriendo el laberinto, el robot va enviando información al ordenador portátil sobre los obstáculos de cada celda, la trayectoria ejecutada y el ordenador portátil es capaz de mostrar una gráfica del laberinto.

La interfaz del ordenador es capaz de recopilar toda la información recabada por el robot durante su recorrido.

\section{Requisitos no funcionales}
%Incluir medidas del robot y algo sobre el consumo estimado del mismo.
%En el word de la memoria que  está en el campus, ver qué podemos completar de los errores como los que hay de ejemplo en la tabla de requisitos no funcionales.
El robot tiene las siguientes medidas: 9.5 x 9.5 x 10cm (largo, ancho, alto).

Cuando el robot se queda sin batería, observamos que se apaga directamente, por lo que es hora de cambiarle las pilas.