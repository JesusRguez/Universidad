%Doc7
\chapter{Implementación}
\section{Librerías}
\subsection{Propias}
Las \hypertarget{bibliotecas}{bibliotecas} usadas para dar soporte al código de Arduino son las siguientes:
\subsubsection{\texttt{Robot.hpp}}
\hypertarget{Robot}{}
\lstinputlisting[language=C]{Robot.hpp}
\subsubsection{\texttt{celda.hpp}}
\hypertarget{Celda}{}
\lstinputlisting[language=C]{celda.hpp}
\subsubsection{\texttt{sUltrasonidos.hpp}}
\hypertarget{sUltrasonidos}{}
\lstinputlisting[language=C]{sUltrasonidos.hpp}
\subsubsection{\texttt{sInfrarrojos.hpp}}
\hypertarget{sInfrarrojos}{}
\lstinputlisting[language=C]{sInfrarrojos.hpp}
\subsubsection{\texttt{sCNY.hpp}}
\hypertarget{sCNY}{}
\lstinputlisting[language=C]{sCNY.hpp}
\subsubsection{\texttt{motor.hpp}}
\hypertarget{Motor}{}
\lstinputlisting[language=C]{motor.hpp}

\subsection{Públicas}
%La que pilló Gabri por internet
\subsubsection{\texttt{StandardCplusplus}}
La librería \hypertarget{StandardCplusplus}{\texttt{StandardCplusplus.hpp}} la hemos descargado de Internet porque nos hacía falta para la implementación del TAD (tipo abstracto de dato) pila que nos ayudará en la vuelta a la casilla inicial.

\section{Apuntes sobre el código}
%Preguntar a Gabri
En la fase de salir del laberinto, entra en un bucle donde escanea la celda en a que se encuentra y luego va aplicando el algoritmo de la mano derecha hasta llegar a la casilla final.

Una vez que alcanza la meta deshace en orden inverso los movimientos que ha hecho, sin tener en cuenta los retrocesos, para volver a la casilla inicial por el camino más corto posible.