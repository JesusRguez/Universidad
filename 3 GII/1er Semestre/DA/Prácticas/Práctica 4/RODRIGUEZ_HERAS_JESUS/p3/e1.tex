%Escriba aquí su respuesta al ejercicio 1.
\begin{itemize}
	\item \textbf{Sin ordenación:} En el caso en el que no tenemos ordenación, he usado una matriz de tipo \texttt{float} del número de celdas de ancho (de anchura) por el número de celdas de alto (de altura).
	\item \textbf{Con ordenación:} En los casos en los que sí tenemos ordenación, he creado una clase llamada \texttt{posicionConValor}, la cual guarda tanto la posición (dos enteros) como el valor (\texttt{float}) de una celda.
	\lstinputlisting[language=C]{posicionConValor.cpp}
\end{itemize}