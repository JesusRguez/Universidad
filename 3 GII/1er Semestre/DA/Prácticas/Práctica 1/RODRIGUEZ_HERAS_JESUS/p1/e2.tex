%Escriba aquí su respuesta al ejercicio 2.
La función de factibilidad diseñada recibe los siguientes parámetros de entrada:
\begin{itemize}
	\item \texttt{float x}: Posición en el eje ``X'' donde se va a comprobar la factibilidad de colocar una defensa.
	\item \texttt{float y}: Posición en el eje ``Y'' donde se va a comprobar la factibilidad de colocar una defensa.
	\item \texttt{Defense* defensa}: Defensa que se va a comprobar.
	\item \texttt{std::list<Object*> obstaculos}: Lista de obstáculos del terreno de juego.
	\item \texttt{float mapWidth}: Tamaño de la anchura del mapa.
	\item \texttt{float mapHeight}: Tamaño del alto del mapa.
	\item \texttt{std::list<Defense*> defensas}: Lista de defensas del juego.
	\item \texttt{float cellWidth}: Ancho de las celdas.
	\item \texttt{float cellHeight}: Alto de las celdas.
\end{itemize}
Primeramente se declara una variable de tipo lógica llamada \texttt{entra} inicializada a \texttt{true}, que será lo que retornará esta función.

Luego se comprueban los siguientes factores:
\begin{enumerate}
	\item La defensa entra en el mapa. Es decir, no se sale por los bordes del mismo. Lo cual es calculado teniendo en cuenta el radio de la defensa y las dimensiones (ancho y alto) del mapa. En caso contrario, la variable \texttt{entra} tomaría el valor \texttt{false}.
	\item Colisión con obstáculo: Se comprueba que la defensa no colisiona con ninguno de los obstáculos del terreno de juego. En caso contrario, la variable \texttt{entra} tomaría el valor \texttt{false}.
	\item Colisión con otra defensa anteriormente colocada: Se recorre la lista de defensas comprobando que, de las ya colocadas, es decir, las $n-1$ defensas anteriores, no colisionen con la defensa que queremos colocar. En caso contrario, la variable \texttt{entra} tomaría el valor \texttt{false.}
\end{enumerate}
Finalmente se devuelve el valor de la variable \texttt{entra}.