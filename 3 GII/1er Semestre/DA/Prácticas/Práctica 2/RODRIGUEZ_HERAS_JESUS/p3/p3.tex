\documentclass[]{article}

\usepackage[left=2.00cm, right=2.00cm, top=2.00cm, bottom=2.00cm]{geometry}
\usepackage[spanish,es-noshorthands]{babel}
\usepackage[utf8]{inputenc} % para tildes y ñ
\usepackage{graphicx} % para las figuras
\usepackage{xcolor}
\usepackage{listings} % para el código fuente en c++

\lstdefinestyle{customc}{
  belowcaptionskip=1\baselineskip,
  breaklines=true,
  frame=single,
  xleftmargin=\parindent,
  language=C++,
  showstringspaces=false,
  basicstyle=\footnotesize\ttfamily,
  keywordstyle=\bfseries\color{green!40!black},
  commentstyle=\itshape\color{gray!40!gray},
  identifierstyle=\color{black},
  stringstyle=\color{orange},
}
\lstset{style=customc}


%opening
\title{Práctica 3. Divide y vencerás}
\author{Jesús Rodríguez Heras \\ % mantenga las dos barras al final de la línea y este comentario
jesus.roderiguezheras@alum.uca.es \\ % mantenga las dos barras al final de la línea y este comentario
Teléfono: 628576107 \\ % mantenga las dos barras al final de la linea y este comentario
NIF: 32088516C \\ % mantenga las dos barras al final de la línea y este comentario
}


\begin{document}

\maketitle

%\begin{abstract}
%\end{abstract}

% Ejemplo de ecuación a trozos
%
%$f(i,j)=\left\{ 
%  \begin{array}{lcr}
%      i + j & si & i < j \\ % caso 1
%      i + 7 & si & i = 1 \\ % caso 2
%      2 & si & i \geq j     % caso 3
%  \end{array}
%\right.$

\begin{enumerate}
\item Describa las estructuras de datos utilizados en cada caso para la representación del terreno de batalla. 

$$ f(radio,rango,salud,dano,coste,dispersion,ataquesPorSegundo)=\frac{rango*dano*ataquesPorSegundo*salud}{dispersion*coste} $$

La función que se encarga de asignar valores a las defensas la hemos planteado de forma simple tal que divide lo considerado ``bueno'' (rango, daño, ataques por segundo y salud) entre lo considerado ``malo'' (dispersión y coste).

\item Implemente su propia versión del algoritmo de ordenación por fusión. Muestre a continuación el código fuente relevante. 

%Escriba aquí su respuesta al ejercicio 2.
La estructura utilizada para representar la tabla de subproblemas resueltos es una matriz de tipo \texttt{float} de tantas filas como defensas hay y tantas columnas como ases tenemos a nuestra disposición.


\item Implemente su propia versión del algoritmo de ordenación rápida. Muestre a continuación el código fuente relevante. 

Escriba aquí su respuesta al ejercicio 3.

\item Realice pruebas de caja negra para asegurar el correcto funcionamiento de los algoritmos de ordenación implementados en los ejercicios anteriores. Detalle a continuación el código relevante.

Escriba aquí su respuesta al ejercicio 4.

\item Analice de forma teórica la complejidad de las diferentes versiones del algoritmo de colocación de defensas en función de la estructura de representación del terreno de batalla elegida. Comente a continuación los resultados. Suponga un terreno de batalla cuadrado en todos los casos. 

%Escriba aquí su respuesta al ejercicio 5.
\begin{itemize}
	\item \textbf{Sin ordenación:} Para este caso, el código para colocar las defensas es de orden $\theta(n^2)$.
	\item \textbf{Con ordenación:}
	\begin{itemize}
		\item \textbf{Montículo:} Para este caso, el algoritmo de colocación de defensas es de orden $\theta(n)$.
		\item \textbf{Fusión y ordenación rápida:} En estos casos tenemos tiempos en el orden de  $n\cdot log_{2}(n)$.
	\end{itemize}
\end{itemize}


\item Incluya a continuación una gráfica con los resultados obtenidos. Utilice un esquema indirecto de medida (considere un error absoluto de valor 0.01 y un error relativo de valor 0.001). Considere en su análisis los planetas con códigos 1500, 2500, 3500,..., 10500. Incluya en el análisis los planetas que considere oportunos para mostrar información relevante.

\begin{lstlisting}
void DEF_LIB_EXPORTED placeDefenses(bool** freeCells, int nCellsWidth,
                                    int nCellsHeight, float mapWidth,
                                    float mapHeight,
                                    std::list<Object*> obstacles,
                                    std::list<Defense*> defenses)
{
    //Continuacion directa del codigo del ejercicio 3.

    maxAttemps = 1000 * std::max(nCellsWidth,nCellsHeight);

    while(currentDefense != defenses.end() && maxAttemps > 0)
    {
        seleccion(cellValues, nCellsWidth, nCellsHeight, &fila, &columna);

        x = fila*cellWidth + cellWidth/2;
        y = columna*cellHeight + cellHeight/2;

        if(factible(x, y, (*currentDefense)->radio, mapWidth, mapHeight,
        cellWidth,cellHeight, obstacles, defenses))
        {
            (*currentDefense)->position.x = x;
            (*currentDefense)->position.y = y;
            (*currentDefense)->position.z = 0;
            currentDefense++;
        }
        maxAttemps--;
    }

  }
\end{lstlisting}


\end{enumerate}

Todo el material incluido en esta memoria y en los ficheros asociados es de mi autoría o ha sido facilitado por los profesores de la asignatura. Haciendo entrega de este documento confirmo que he leído la normativa de la asignatura, incluido el punto que respecta al uso de material no original.

\end{document}
