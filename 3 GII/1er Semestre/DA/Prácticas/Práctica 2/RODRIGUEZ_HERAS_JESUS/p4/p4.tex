\documentclass[]{article}

\usepackage[left=2.00cm, right=2.00cm, top=2.00cm, bottom=2.00cm]{geometry}
\usepackage[spanish,es-noshorthands]{babel}
\usepackage[utf8]{inputenc} % para tildes y ñ

%opening
\title{Práctica 4. Exploración de grafos}
\author{Jesús Rodríguez Heras \\ % mantenga las dos barras al final de la línea y este comentario
jesus.roderiguezheras@alum.uca.es \\ % mantenga las dos barras al final de la línea y este comentario
Teléfono: 628576107 \\ % mantenga las dos barras al final de la linea y este comentario
NIF: 32088516C \\ % mantenga las dos barras al final de la línea y este comentario
}


\begin{document}

\maketitle

%\begin{abstract}
%\end{abstract}

% Ejemplo de ecuación a trozos
%
%$f(i,j)=\left\{ 
%  \begin{array}{lcr}
%      i + j & si & i < j \\ % caso 1
%      i + 7 & si & i = 1 \\ % caso 2
%      2 & si & i \geq j     % caso 3
%  \end{array}
%\right.$

\begin{enumerate}
\item Comente el funcionamiento del algoritmo y describa las estructuras necesarias para llevar a cabo su implementación.

$$ f(radio,rango,salud,dano,coste,dispersion,ataquesPorSegundo)=\frac{rango*dano*ataquesPorSegundo*salud}{dispersion*coste} $$

La función que se encarga de asignar valores a las defensas la hemos planteado de forma simple tal que divide lo considerado ``bueno'' (rango, daño, ataques por segundo y salud) entre lo considerado ``malo'' (dispersión y coste).

\item Incluya a continuación el código fuente relevante del algoritmo.

%Escriba aquí su respuesta al ejercicio 2.
La estructura utilizada para representar la tabla de subproblemas resueltos es una matriz de tipo \texttt{float} de tantas filas como defensas hay y tantas columnas como ases tenemos a nuestra disposición.


\end{enumerate}

Todo el material incluido en esta memoria y en los ficheros asociados es de mi autoría o ha sido facilitado por los profesores de la asignatura. Haciendo entrega de esta práctica confirmo que he leído la normativa de la asignatura, incluido el punto que respecta al uso de material no original.

\end{document}
