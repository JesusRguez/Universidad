Continuación de la función \texttt{placeDefenses} del ejercicio 3:
%\lstinputlisting[language=C]{def6.cpp}
\begin{lstlisting}
void DEF_LIB_EXPORTED placeDefenses(bool** freeCells, int nCellsWidth, int nCellsHeight, float mapWidth, float mapHeight, std::list<Object*> obstacles, std::list<Defense*> defenses) {

    float cellWidth = mapWidth / nCellsWidth;
    float cellHeight = mapHeight / nCellsHeight;
    int maxAttemps = 1000;
    int fila = 0, columna = 0;
    float x = 0, y = 0;

    // ...
    // Ejercicio 3
    // ...
    // Continuacion del ejercicio 3

    while (currentDefense != defenses.end() && maxAttemps > 0) {
        seleccion(mapa, nCellsWidth, nCellsHeight, &fila, &columna);
        x = fila*cellWidth + cellWidth*0.5f;
        y = columna*cellHeight + cellHeight*0.5f;
        if(factibilidad(x, y, (*currentDefense), obstacles, mapWidth, mapHeight, defenses, cellWidth, cellHeight)){
            (*currentDefense)->position.x = x;
            (*currentDefense)->position.y = y;
            (*currentDefense)->position.z = 0;
            ++currentDefense;
        }
        --maxAttemps;
    }

    // Resto de codigo de PRINT_DEFENSE_STRATEGY que no hemos modificado

}

\end{lstlisting}
Función \texttt{factibilidad}:
%\lstinputlisting[language=C]{fac.cpp}
\begin{lstlisting}
bool factibilidad(float x, float y, Defense* defensa, std::list<Object*> obstaculos, float mapWidth, float mapHeight, std::list<Defense*> defensas, float cellWidth, float cellHeight){

    bool entra = true;

    if (x-defensa->radio < 0 || x+defensa->radio > mapWidth || y-defensa->radio < 0 || y+defensa->radio > mapHeight) {
        entra = false; //No cabe porque se sale de los limites del mapa
    }

    std::list<Object*>::const_iterator iterObst = obstaculos.begin();
    std::list<Defense*>::const_iterator iterDef = defensas.begin();
    Vector3 posicionDefensa(x, y, 0);
    while (iterObst!=obstaculos.end()) {
        if ((defensa->radio + (*iterObst)->radio) > (_distance(posicionDefensa, (*iterObst)->position))) {
            entra = false; //Se choca con un obsaculo
        }else{
            while ((*iterDef)!=defensa) {
                if ((defensa->radio + (*iterDef)->radio) > (_distance(posicionDefensa, (*iterDef)->position))) {
                    entra = false; //Se choca con una defensa
                }
                ++iterDef;
            }
        }
        ++iterObst;
    }

    return entra;
}

\end{lstlisting}
Función \texttt{selección}:
%\lstinputlisting[language=C]{sel.cpp}
\begin{lstlisting}
void seleccion(float** mapa, int nCellsWidth, int nCellsHeight, int* fila, int* columna){

    float maxi = 0;

    for (size_t i = 0; i < nCellsWidth; ++i) {
        for (size_t j = 0; j < nCellsHeight; ++j) {
            if (mapa[i][j] > maxi) {
                maxi = mapa[i][j];
                *fila = i;
                *columna = j;
            }
        }
    }
    mapa[*fila][*columna] = 0;
}

\end{lstlisting}