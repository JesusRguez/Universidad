\documentclass[]{article}

\usepackage[left=2.00cm, right=2.00cm, top=2.00cm, bottom=2.00cm]{geometry}
\usepackage[spanish,es-noshorthands]{babel}
\usepackage[utf8]{inputenc} % para tildes y ñ
\usepackage{graphicx} % para las figuras
\usepackage{xcolor}
\usepackage{listings} % para el código fuente en c++

\lstdefinestyle{customc}{
  belowcaptionskip=1\baselineskip,
  breaklines=true,
  frame=single,
  xleftmargin=\parindent,
  language=C++,
  showstringspaces=false,
  basicstyle=\footnotesize\ttfamily,
  keywordstyle=\bfseries\color{green!40!black},
  commentstyle=\itshape\color{gray!40!gray},
  identifierstyle=\color{black},
  stringstyle=\color{orange},
}
\lstset{style=customc}


%opening
\title{Práctica 2. Programación dinámica}
\author{Jesús Rodríguez Heras \\ % mantenga las dos barras al final de la línea y este comentario
jesus.roderiguezheras@alum.uca.es \\ % mantenga las dos barras al final de la línea y este comentario
Teléfono: 628576107 \\ % mantenga las dos barras al final de la linea y este comentario
NIF: 32088516C \\ % mantenga las dos barras al final de la línea y este comentario
}


\begin{document}

\maketitle

%\begin{abstract}
%\end{abstract}

% Ejemplo de ecuación a trozos
%
%$f(i,j)=\left\{ 
%  \begin{array}{lcr}
%      i + j & si & i < j \\ % caso 1
%      i + 7 & si & i = 1 \\ % caso 2
%      2 & si & i \geq j     % caso 3
%  \end{array}
%\right.$

\begin{enumerate}
\item Formalice a continuación y describa la función que asigna un determinado valor a cada uno de los tipos de defensas.

$$ f(radio,rango,salud,dano,coste,dispersion,ataquesPorSegundo)=\frac{rango*dano*ataquesPorSegundo*salud}{dispersion*coste} $$

La función que se encarga de asignar valores a las defensas la hemos planteado de forma simple tal que divide lo considerado ``bueno'' (rango, daño, ataques por segundo y salud) entre lo considerado ``malo'' (dispersión y coste).

\item Describa la estructura o estructuras necesarias para representar la tabla de subproblemas resueltos.

%Escriba aquí su respuesta al ejercicio 2.
La estructura utilizada para representar la tabla de subproblemas resueltos es una matriz de tipo \texttt{float} de tantas filas como defensas hay y tantas columnas como ases tenemos a nuestra disposición.

\item En base a los dos ejercicios anteriores, diseñe un algoritmo que determine el máximo beneficio posible a obtener dada una combinación de defensas y \emph{ases} disponibles. Muestre a continuación el código relevante.

Escriba aquí su respuesta al ejercicio 3.

\item Diseñe un algoritmo que recupere la combinación óptima de defensas a partir del contenido de la tabla de subproblemas resueltos. Muestre a continuación el código relevante.

Escriba aquí su respuesta al ejercicio 4.

\end{enumerate}

Todo el material incluido en esta memoria y en los ficheros asociados es de mi autoría o ha sido facilitado por los profesores de la asignatura. Haciendo entrega de este documento confirmo que he leído la normativa de la asignatura, incluido el punto que respecta al uso de material no original.

\end{document}
