%%\documentclass[a4paper,12pt,oneside]{llncs}
\documentclass[12pt,letterpaper]{article}
\usepackage[right=2cm,left=3cm,top=2cm,bottom=2cm,headsep=0cm]{geometry}

%%%%%%%%%%%%%%%%%%%%%%%%%%%%%%%%%%%%%%%%%%%%%%%%%%%%%%%%%%%
%% Juego de caracteres usado en el archivo fuente: UTF-8
\usepackage{ucs}
\usepackage[utf8x]{inputenc}

%%%%%%%%%%%%%%%%%%%%%%%%%%%%%%%%%%%%%%%%%%%%%%%%%%%%%%%%%%%
%% Juego de caracteres usado en la salida dvi
%% Otra posibilidad: \usepackage{t1enc}
\usepackage[T1]{fontenc}

%%%%%%%%%%%%%%%%%%%%%%%%%%%%%%%%%%%%%%%%%%%%%%%%%%%%%%%%%%%
%% Ajusta maergenes para a4
%\usepackage{a4wide}

%%%%%%%%%%%%%%%%%%%%%%%%%%%%%%%%%%%%%%%%%%%%%%%%%%%%%%%%%%%
%% Uso fuente postscript times, para que los ps y pdf queden y pequeños...
\usepackage{times}

%%%%%%%%%%%%%%%%%%%%%%%%%%%%%%%%%%%%%%%%%%%%%%%%%%%%%%%%%%%
%% Posibilidad de hipertexto (especialmente en pdf)
%\usepackage{hyperref}
\usepackage[bookmarks = true, colorlinks=true, linkcolor = black, citecolor = black, menucolor = black, urlcolor = black]{hyperref}

%%%%%%%%%%%%%%%%%%%%%%%%%%%%%%%%%%%%%%%%%%%%%%%%%%%%%%%%%%%
%% Graficos 
\usepackage{graphics,graphicx}

%%%%%%%%%%%%%%%%%%%%%%%%%%%%%%%%%%%%%%%%%%%%%%%%%%%%%%%%%%%
%% Ciertos caracteres "raros"...
\usepackage{latexsym}

%%%%%%%%%%%%%%%%%%%%%%%%%%%%%%%%%%%%%%%%%%%%%%%%%%%%%%%%%%%
%% Matematicas aun más fuertes (american math dociety)
\usepackage{amsmath}

%%%%%%%%%%%%%%%%%%%%%%%%%%%%%%%%%%%%%%%%%%%%%%%%%%%%%%%%%%%
\usepackage{multirow} % para las tablas
\usepackage[spanish,es-tabla]{babel}

%%%%%%%%%%%%%%%%%%%%%%%%%%%%%%%%%%%%%%%%%%%%%%%%%%%%%%%%%%%
%% Fuentes matematicas lo mas compatibles posibles con postscript (times)
%% (Esto no funciona para todos los simbolos pero reduce mucho el tamaño del
%% pdf si hay muchas matamaticas....
%\usepackage{mathptm}

%%% VARIOS:
\usepackage{slashbox}
\usepackage{verbatim}
\usepackage{array}
\usepackage{listings}
\usepackage{multirow}

%% MARCA DE AGUA
%% Este package de "draft copy" NO funciona con pdflatex
%%\usepackage{draftcopy}
%% Este package de "draft copy" SI funciona con pdflatex
%%%\usepackage{pdfdraftcopy}
%%%%%%%%%%%%%%%%%%%%%%%%%%%%%%%%%%%%%%%%%%%%%%%%%%%%%%%%%%%
%% Indenteacion en español...
\usepackage[spanish]{babel}
\usepackage[svgnames,x11names,table]{xcolor}
\usepackage{listings}
% Para escribir código en C
% \begin{lstlisting}[language=C]
% #include <stdio.h>
% int main(int argc, char* argv[]) {
% puts("Hola mundo!");
% }
% \end{lstlisting}


\title{Ettercap y SSL Strip}
\author{Jesús Rodríguez Heras\\Juan Pedro Rodríguez Gracia}

\begin{document}
	
	\maketitle
	\begin{abstract} %Poner esto en todas las prácticas de PCTR
		\begin{center}
			Definición de Etteercap, SSL Strip y demostración práctica de un ataque man-in-the-middle.
		\end{center}
	\end{abstract}
	\thispagestyle{empty}
	\newpage
	
	\tableofcontents
	\newpage
	
	%%\listoftables
	%%\newpage
	
	%%\listoffigures
	%%\newpage
	
	%%%% REAL WORK BEGINS HERE:
	
	%%Configuracion del paquete listings
	\lstset{language=bash, numbers=left, numberstyle=\tiny, numbersep=10pt, firstnumber=1, stepnumber=1, basicstyle=\small\ttfamily, tabsize=1, extendedchars=true, inputencoding=latin1}


\section{Ettercap}
\subsection{¿Qué es Ettercap?}
Es un sniffer para redes LAN que permite la inyección y modificación de datos en una conexión establecida gracias al ataque ``Man-in-the-middle''.

\subsubsection{Man-in-the-middle}
Es un ataque donde se puede leer, insertar y modificar los datos del usuario a voluntad del atacante.\\

El atacante se coloca entre el emisor original del mensaje (el host del usuario en cuestión) y el receptor original del mismo (AP o switch) sin que ninguno de dichos roles, sepa de la existencia del atacante. Debido a esto, ninguna de las partes sabe que el mensaje enviado/recibido ha sido violado por el atacante.\\

Es necesario destacar que, a diferencia de un RogueAP, no es necesario que estemos directamente conectados al usuario al que vamos a atacar y no importa que sea de forma inalámbrica o mediante Wi-Fi.

\section{SSL Strip}
\subsection{Definición}
Es un ataque que se centra en que, cuando la víctima está navegando por internet y entra a una página con protocolo HTTPS, su navegador la reciba en protocolo HTTP, por lo cual, a la hora de introducir unos credenciales de usuario, el atacante obtenga en texto plano dichas credenciales de usuario.


\section{Ejemplo de ataque}
A continuación realizaremos un ataque con Ettercap, ejecutándolo desde Kali Linux, para robar las credenciales de usuario a una persona que se conecta a una web con protocolo http (debido al protocolo https, no es posible realizar este ataque para todas las páginas web).

Una vez explicado todo esto, comencemos con el ataque:
\newpage
\begin{enumerate}
\begin{figure}
\item Accedemos a Ettercap ejecutándolo desde Kali Linux.
	\begin{center}
		\includegraphics[scale=0.3]{Captura1.png}
		\caption{Ettercap.}
		\label{fig: Ettercap}
	\end{center}
\end{figure}

\begin{figure}
	\item Vamos a ``Sniff'' y seleccionamos ``Unified sniffing...''. Ahora, seleccionamos la interfaz de red mediante la cual vamos a realizar el ataque.
	\begin{center}
		\includegraphics[scale=0.3]{Captura2.png}
		\caption{Selección de interfaz de red.}
		\label{fig: Selección de interfaz de red}
	\end{center}
\end{figure}

\begin{figure}
	\item Vamos a la pestaña ``Hosts'' y seleccionamos ``Scan for hosts''.
	\begin{center}
		\includegraphics[scale=0.3]{Captura3.png}
		\caption{Escáner de hosts.}
		\label{fig: Escáner de hosts}
	\end{center}
\end{figure}

\begin{figure}
	\item A continuación vamos a ``Hosts'' y seleccionamos ``Hosts list'' para ver los hosts disponibles en la red.
	\begin{center}
		\includegraphics[scale=0.3]{Captura4.png}
		\caption{Lista de hosts disponibles.}
		\label{fig: Lista de hosts disponibles}
	\end{center}
\end{figure}

\begin{figure}
	\item Ahora seleccionamos la víctima de la lista de hosts disponibles y hacemos click en ``Add to Target 1'' (podemos realizar el ataque hasta con dos hosts al mismo tiempo, pero ahora nos centraremos en un solo hosts).
	\begin{center}
		\includegraphics[scale=0.3]{Captura5.png}
		\caption{Selección de la víctima.}
		\label{fig: Selección de la víctima}
	\end{center}
\end{figure}

\begin{figure}
	\item Cuando hemos decidido nuestra víctima, vamos a ``Mitm'', seleccionamos ``Arp Poissoning'' y activamos la opción ``Sniff remote conections.''.
	\begin{center}
		\includegraphics[scale=0.3]{Captura6.png}
		\caption{Activación del ARP Poissoning.}
		\label{fig: Activación del ARP Poissoning}
	\end{center}
\end{figure}

\begin{figure}
	\item Una vez hechos los pasos anteriores estamos listos para hacer el ataque. Entonces, nos dirigimos a ``Start'' y seleccionamos ``Start sniffing''. Ahora solo queda que el usuario entre en alguna página con protocolo http y, en la consola de abajo, nos saldrán sus credenciales de usuario y la página web donde han sido empleados.
	\begin{center}
		\includegraphics[scale=0.3]{Captura7.png}
		\caption{Obtención de resultados.}
		\label{fig: Obtención de resultados}
	\end{center}
\end{figure}
\end{enumerate}
\end{document}