%%\documentclass[a4paper,12pt,oneside]{llncs}
\documentclass[12pt,letterpaper]{article}
\usepackage[right=2cm,left=3cm,top=2cm,bottom=2cm,headsep=0cm]{geometry}

%%%%%%%%%%%%%%%%%%%%%%%%%%%%%%%%%%%%%%%%%%%%%%%%%%%%%%%%%%%
%% Juego de caracteres usado en el archivo fuente: UTF-8
\usepackage{ucs}
\usepackage[utf8x]{inputenc}

%%%%%%%%%%%%%%%%%%%%%%%%%%%%%%%%%%%%%%%%%%%%%%%%%%%%%%%%%%%
%% Juego de caracteres usado en la salida dvi
%% Otra posibilidad: \usepackage{t1enc}
\usepackage[T1]{fontenc}

%%%%%%%%%%%%%%%%%%%%%%%%%%%%%%%%%%%%%%%%%%%%%%%%%%%%%%%%%%%
%% Ajusta maergenes para a4
%\usepackage{a4wide}

%%%%%%%%%%%%%%%%%%%%%%%%%%%%%%%%%%%%%%%%%%%%%%%%%%%%%%%%%%%
%% Uso fuente postscript times, para que los ps y pdf queden y pequeños...
\usepackage{times}

%%%%%%%%%%%%%%%%%%%%%%%%%%%%%%%%%%%%%%%%%%%%%%%%%%%%%%%%%%%
%% Posibilidad de hipertexto (especialmente en pdf)
%\usepackage{hyperref}
\usepackage[bookmarks = true, colorlinks=true, linkcolor = black, citecolor = black, menucolor = black, urlcolor = black]{hyperref}

%%%%%%%%%%%%%%%%%%%%%%%%%%%%%%%%%%%%%%%%%%%%%%%%%%%%%%%%%%%
%% Graficos 
\usepackage{graphics,graphicx}

%%%%%%%%%%%%%%%%%%%%%%%%%%%%%%%%%%%%%%%%%%%%%%%%%%%%%%%%%%%
%% Ciertos caracteres "raros"...
\usepackage{latexsym}

%%%%%%%%%%%%%%%%%%%%%%%%%%%%%%%%%%%%%%%%%%%%%%%%%%%%%%%%%%%
%% Matematicas aun más fuertes (american math dociety)
\usepackage{amsmath}

%%%%%%%%%%%%%%%%%%%%%%%%%%%%%%%%%%%%%%%%%%%%%%%%%%%%%%%%%%%
\usepackage{multirow} % para las tablas
%\usepackage[spanish,es-tabla]{babel}

%%%%%%%%%%%%%%%%%%%%%%%%%%%%%%%%%%%%%%%%%%%%%%%%%%%%%%%%%%%
%% Fuentes matematicas lo mas compatibles posibles con postscript (times)
%% (Esto no funciona para todos los simbolos pero reduce mucho el tamaño del
%% pdf si hay muchas matamaticas....
%\usepackage{mathptm}

%%% VARIOS:
\usepackage{slashbox}
\usepackage{verbatim}
\usepackage{array}
\usepackage{listings}
\usepackage{multirow}

%% MARCA DE AGUA
%% Este package de "draft copy" NO funciona con pdflatex
%%\usepackage{draftcopy}
%% Este package de "draft copy" SI funciona con pdflatex
%%%\usepackage{pdfdraftcopy}
%%%%%%%%%%%%%%%%%%%%%%%%%%%%%%%%%%%%%%%%%%%%%%%%%%%%%%%%%%%
%% Indenteacion en español...
%\usepackage[english]{babel}
\usepackage[svgnames,x11names,table]{xcolor}
\usepackage{listings}
% Para escribir código en C
% \begin{lstlisting}[language=C]
% #include <stdio.h>
% int main(int argc, char* argv[]) {
% puts("Hola mundo!");
% }
% \end{lstlisting}



\title{Inmobilized}
\author{Jesús Rodríguez Heras\\Isabel Pérez Fernández\\Juan Pedro Rodríguez Gracia\\Gabriel Fernando Sánchez Reina}
\date{May, 28th 2018}

\begin{document}
	
	\maketitle
%	\begin{abstract} %Poner esto en todas las prácticas de PCTR
%		\begin{center}
%			Definición de Etteercap, SSL Strip y demostración práctica de un ataque man-in-the-middle.
%		\end{center}
%	\end{abstract}
	\thispagestyle{empty}
	\newpage
	
%	\renewcommand{\tableofcontents}{\LARGE{\textbf{Index}}} 
	
	\tableofcontents
	\newpage
	
	%%\listoftables
	%%\newpage
	
	%%\listoffigures
	%%\newpage
	
	%%%% REAL WORK BEGINS HERE:
	
	%%Configuracion del paquete listings
	\lstset{language=bash, numbers=left, numberstyle=\tiny, numbersep=10pt, firstnumber=1, stepnumber=1, basicstyle=\small\ttfamily, tabsize=1, extendedchars=true, inputencoding=latin1}
	
	\section{Permissions per user}
	The users have the following permissions according to the type of user they belong to:
	\begin{itemize}
		\item Auditor: You will have permission to run the tests if they are in ``to do'' or ``rejected''. You can only see the test assigned to you. When the test is completed, the test status goes to ``to review''.
		\item Supervisor: You will have permissions to see the tests that are assigned to him and that are in ``to review''. If the test is well done, it will put the test status to ``approved'', if it does not approve it, it will go to ``rejected''.
		\item Manager: You will have permissions to see everything, but you will not be able to change the contents of the tables.
	\end{itemize}
		
	\section{Tests}
	The tests involved in the immobilized are the following:
	\begin{itemize}
		\item Validity: Fixed assets exist in the company.
		\item Court: The date of the expense must correspond to the year in which the exercise is.
		\item Ownership: All the included accounts of the fixed assets will be the property of the company.
		\item Accuracy: The amounts of each fixed asset transaction are correct.
		\item Valuation: The fixed assets annual accounts have to be correctly valued.
		\item Classification: The fixed assets are conveniently classified.
		\item Disclosure: The balance sheet of the company and the rest of the information must be properly presented.
	\end{itemize}
	
	\section{How to perform the tests?}
	\begin{itemize}
		\item Validity: The auditor must upload a file to the database to certify that it is valid.
		\item Court: The auditor must upload a document explaining why the test has said result.
		\item Ownership: The auditor must upload a file that shows the property of the asset.
		\item Accuracy: The auditor must upload a document that says whether the test is correct or not, because we do not have data with which to test them.
		\item Valuation: The auditor must upload a document stating whether the useful life corresponds to that of the type of the machine.
		\item Classification: The auditor must upload a document stating whether the classification of the assets is correct.
		\item Disclosure: The auditor must upload a document with the justification that the asset is properly presented.
	\end{itemize}

	\section{Incidents}
	If a test fails or returns an incident, the information of that incident will be entered in the database in the table of ``incidents'' that will be returned to the main program and will be what the administrator of the application will receive, who will be in charge of talking with the company's staff.
	
	\section{Conclusion \& future upgrades}
	When selecting the test, you can select the test of the table and not its identifier. \\
	
\noindent	
It would also be possible that, in the global table, the type of test, which machine is the test, etc. instead of depending on the name of the file. \\

\noindent
The table could be better constructed (functionally and aesthetically).


\end{document}