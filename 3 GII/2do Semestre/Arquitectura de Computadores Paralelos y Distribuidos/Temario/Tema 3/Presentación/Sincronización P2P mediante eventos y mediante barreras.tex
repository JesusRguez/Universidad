\documentclass{beamer}

%% Juego de caracteres usado en el archivo fuente: UTF-8
\usepackage{ucs}
\usepackage[utf8x]{inputenc}
\uselanguage{spanish}
%Para la identación del español
\usepackage[spanish]{babel}
\usepackage{animate}
\usepackage{listings}



% There are many different themes available for Beamer. A comprehensive
% list with examples is given here:
% http://deic.uab.es/~iblanes/beamer_gallery/index_by_theme.html
% You can uncomment the themes below if you would like to use a different
% one:
%\usetheme{AnnArbor}
%\usetheme{Antibes}
%\usetheme{Bergen}
%\usetheme{Berkeley}
%\usetheme{Berlin}
%\usetheme{Boadilla}
%\usetheme{boxes}
%\usetheme{CambridgeUS}
%\usetheme{Copenhagen}
%\usetheme{Darmstadt}
%\usetheme{default}
%\usetheme{Frankfurt}
%\usetheme{Goettingen}
%\usetheme{Hannover}
%\usetheme{Ilmenau}
%\usetheme{JuanLesPins}
%\usetheme{Luebeck}
\usetheme{Madrid}
%\usetheme{Malmoe}
%\usetheme{Marburg}
%\usetheme{Montpellier}
%\usetheme{PaloAlto}
%\usetheme{Pittsburgh}
%\usetheme{Rochester}
%\usetheme{Singapore}
%\usetheme{Szeged}
%\usetheme{Warsaw}

%Para la identación del español
\usepackage[spanish]{babel}

\title{Sincronización P2P mediante eventos y mediante barreras}

% A subtitle is optional and this may be deleted
%\subtitle{Optional Subtitle}

\author{Jesús Rodríguez Heras}
% - Give the names in the same order as the appear in the paper.
% - Use the \inst{?} command only if the authors have different
%   affiliation.

%\institute[Escuela Superior de Ingeniería] % (optional, but mostly needed)
%{
%  \inst{1}%
%  Department of Computer Science\\
%  University of Somewhere
%  \and
%  \inst{2}%
%  Department of Theoretical Philosophy\\
%  University of Elsewhere}
% - Use the \inst command only if there are several affiliations.
% - Keep it simple, no one is interested in your street address.

\date{1 de abril de 2019}
% - Either use conference name or its abbreviation.
% - Not really informative to the audience, more for people (including
%   yourself) who are reading the slides online

%\subject{Theoretical Computer Science}
% This is only inserted into the PDF information catalog. Can be left
% out. 

% If you have a file called "university-logo-filename.xxx", where xxx
% is a graphic format that can be processed by latex or pdflatex,
% resp., then you can add a logo as follows:

% pgfdeclareimage[height=0.5cm]{university-logo}{university-logo-filename}
% \logo{\pgfuseimage{university-logo}}

% Delete this, if you do not want the table of contents to pop up at
% the beginning of each subsection:
%\AtBeginSubsection[]
%{
%  \begin{frame}<beamer>{Índice}
%    \tableofcontents[currentsection,currentsubsection]
%  \end{frame}
%}

% Let's get started
\begin{document}

\begin{frame}
  \titlepage
\end{frame}

\begin{frame}{Índice}
  \tableofcontents
  % You might wish to add the option [pausesections]
\end{frame}

% Section and subsections will appear in the presentation overview
% and table of contents.

\section{Sincronización}
\begin{frame}{Sincronización}
	\begin{block}{Definición}
		Procedimiento por el que se ajusta el valor de un reloj a un tiempo físico de referencia con una precisión preestablecida.
	\end{block}
	\begin{alertblock}{Importante}
		Es importante asegurar una correcta sincronización entre los computadores de un sistema distribuido para que no se produzcan derivas.
	\end{alertblock}
	\begin{block}{Deriva}
		Representa el tiempo que un reloj se desvía respecto a la hora real con el paso del tiempo.
	\end{block}
\end{frame}

\section{P2P}
\begin{frame}{P2P}
	\begin{block}{Definición}
		Es un sistema auto-organizado de entidades iguales y autónomas cuyo objetivo es el uso compartido de recursos distribuidos en un ambiente de red evitando servicios centralizados.
	\end{block}
	\begin{exampleblock}{Ejemplo}
		BitTorrent, eMule, $\mu$Torrent, etc.
	\end{exampleblock}
\end{frame}

\section{Sincronización P2P mediante eventos}
%http://sedici.unlp.edu.ar/handle/10915/23619
%http://sedici.unlp.edu.ar/bitstream/handle/10915/23619/4787-paper_cacic_2012_a_presentar.pdf?sequence=1
\begin{frame}{Sincronización P2P mediante eventos}
	\begin{itemize}
		\item Para simplificarlo, supondremos que tenemos dos procesadores conectados mediante una red P2P (pont-to-point).
		
		\item Para sincronizar dichos procesadores tendremos una estructura de espera activa con una variable (o flag) común en ambos procesadores.
		
		\item Cuando uno de ellos realiza un cambio sobre dicha variable compartida (evento) el segundo se sincroniza con el primero.
	\end{itemize}
\end{frame}

\begin{frame}{Sincronización P2P mediante eventos}
Ejemplo de dos procesos sincronizados mediante un evento donde inicialmente la variable global \texttt{turno} vale 0 (o un valor distinto de 1).
	\begin{columns}
		\column{0.45\textwidth}
		\begin{block}{Procesador 1}
			\texttt{//Resto de código inicial\\
			while(turno != 1)\{\\
			\hspace{0.8cm}//Espera ocupada\\
			\}\\
			//Código sincronizado}
		\end{block}
		\column{0.45\textwidth}
		\begin{block}{Procesador 2}
			\texttt{//Resto de código inicial\\
			turno = 1;\\
			//Código sincronizado}
		\end{block}
	\end{columns}
\vspace{0.5cm}
En este ejemplo podemos ver cómo el procesador 1 se queda a la espera de que el procesador 2 le pase el turno (estableciendo a 1 la variable \texttt{turno}).

Con esto se sincronizan para realizar la sección de \texttt{Código sincronizado} o, simplemente, porque sea necesaria una sincronización en ese punto del programa.
\end{frame}

\section{Sincronización mediante barreras}
%http://webdiis.unizar.es/~ezpeleta/lib/exe/fetch.php?media=misdatos:pc:08.pdf
%http://ocw.uc3m.es/ingenieria-informatica/arquitectura-de-computadores/materiales/es-m5-03-sincr-ocw.pdf
\begin{frame}{Sincronización mediante barreras}
	\begin{block}{Barrera}
		Es un punto del código que ninguna entidad concurrente sobrepasa hasta que todas las entidades existentes han llegado a ella.
	\end{block}
\end{frame}

\begin{frame}{Sincronización mediante barreras}
Ejemplo de código Java para la gestión de una barrera y varios procesos:

\textcolor{white}{a\\}
	\texttt{//Código de Thread:\\
		public Hilo extends Thread\{\\
			\hspace{0.8cm}public Hilo (CyclicBarrier ba) \{...\}\\
			\hspace{0.8cm}public void run()\{\\
\hspace{0.8cm}\hspace{0.8cm}try\{\\
\hspace{0.8cm}\hspace{0.8cm}\hspace{0.8cm}int i = bar.await();\\
\hspace{0.8cm}\hspace{0.8cm}\}catch(Exception e)\{\}\\
\hspace{0.8cm}\hspace{0.8cm}//Código a ejecutar cuando se abre barrera...\\
\hspace{0.8cm}\}\\
		\}\\
	//Programa principal:\\
	int numHilos = n; //Número de hilos que abren barrera\\
	CyclicBarrier Barrera = new CyclicBarrier(numHilos);\\
	new Hilo(Barrera).start();}
\end{frame}

\begin{frame}{Sincronización mediante barreras}

\begin{columns}
	\column{0.45\textwidth}
	En el ejemplo anterior se ve cómo, hasta que no llegan todos los procesos a la barrera, el programa no seguirá ejecutándose. Y, cuando lo haga, lo harán todos los procesos a la vez.
	
	\vspace{0.5cm}
	A la derecha tenemos una figura donde se lanzan tres procesos a la vez y son sincronizados de por medio.
	\column{0.45\textwidth}
	\begin{figure}[h]
		\centering
		\includegraphics[scale=0.23]{Barreras.png}
	\end{figure}
\end{columns}
\end{frame}

\section{Bibliografía}

\begin{frame}{Bibliografía}

\begin{thebibliography}{10}
	
%	\beamertemplatebookbibitems
%	% Start with overview books.
%	
%	\bibitem{Author1990}
%	A.~Author.
%	\newblock {\em Handbook of Everything}.
%	\newblock Some Press, 1990.
	
	
	\beamertemplatearticlebibitems
	% Followed by interesting articles. Keep the list short. 
	
	%http://sedici.unlp.edu.ar/handle/10915/23619
	%http://sedici.unlp.edu.ar/bitstream/handle/10915/23619/4787-paper_cacic_2012_a_presentar.pdf?sequence=1
	
	%http://webdiis.unizar.es/~ezpeleta/lib/exe/fetch.php?media=misdatos:pc:08.pdf
	%http://ocw.uc3m.es/ingenieria-informatica/arquitectura-de-computadores/materiales/es-m5-03-sincr-ocw.pdf
	
	\bibitem{Eventos1}
	Sincronización de eventos discretos
	\newblock {\em \url{http://sedici.unlp.edu.ar/handle/10915/23619}}
	
	\bibitem{Eventos2}
	Sincronización de eventos discretos en sistemas distribuidos
	\newblock {\em \url{http://sedici.unlp.edu.ar/bitstream/handle/10915/23619/4787-paper_cacic_2012_a_presentar.pdf?sequence=1}}
	
	\bibitem{Barrera1}
	Sincronización por barrera
	\newblock {\em \url{http://webdiis.unizar.es/~ezpeleta/lib/exe/fetch.php?media=misdatos:pc:08.pdf}}
	
	\bibitem{Barrera2}
	Sincronización por barrera, Grupo ARCOS
	\newblock {\em \url{http://ocw.uc3m.es/ingenieria-informatica/arquitectura-de-computadores/materiales/es-m5-03-sincr-ocw.pdf}}
	
	
\end{thebibliography}
\end{frame}

\end{document}


