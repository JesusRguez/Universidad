\chapter{AWS}
\hypertarget{AWS}{}
\section{¿Qué es AWS?}
Amazon Web Services, en adelante AWS, es una colección de servicios de computación en la nube pública, también llamados servicios web\footnote{Es una tecnología que utiliza un conjunto de protocolos y estándares (como mensajería por XML) que sirven para intercambiar datos entre aplicaciones. No están vinculados a ningún sistema operativo, por lo que funcionan de forma independiente y simultánea. Los componentes de los servicios web son: SOAP, UDDI, WSDL, etc.}.

Es usado en aplicaciones como Dorpbox o Foursquare. Es una de las ofertas internacionales más importantes de la computación en la nube y compite directamente con \hyperlink{azure}{Microsoft Azure} aunque AWS es considerado como pionero en este campo.

\section{Historia}
AWS se lanza oficialmente en 2006 ofreciendo servicios en línea para otros sitios web o aplicaciones del lado del cliente.

La mayoría de estos servicios no están expuestos directamente a usuarios finales, sino que ofrecen una funcionalidad que otros desarrolladores puedan utilizar en sus aplicaciones.

El primer servicio de AWS lanzado para el uso público era Simple Queue Service (SQS). El cual es un servicio de colas de mensajes completamente administrado que permite desacoplar y ajustar la escala de microservicios, sistemas distribuidos y aplicaciones sin servidor. SQS elimina la complejidad y los gastos generales asociados con la gestión y el funcionamiento de middleware\footnote{Software que conecta componentes de software o aplicaciones para que puedan intercambiar datos entre ellas. Muy utilizado para soportar aplicaciones distribuidas.} orientado a mensajes, y permite a los desarrolladores centrarse en la diferenciación del trabajo.

Hoy en día, AWS proporciona una plataforma de infraestructura escalable, de confianza y de bajo costo en la nube que impula cientos de miles de negocios de 190 países de todo el mundo. Con centros de datos en Estados Unidos, Europa, Brasil, Singapur, Japón y Australia.

\section{Descripción}
Como ya se ha comentado anteriormente, AWS es una plataforma ideal para lanzar aplicaciones y proyectos de forma distribuida y escalable.

Entre sus beneficios principales, podemos destacar las siguientes:
\begin{itemize}
	\item \textbf{Bajo costo:} AWS ofrece precios bajos por uso, sin gastos anticipados ni compromisos a largo plazo. Solo pagas por lo que usas.
	\item \textbf{Agilidad y elasticidad instantánea:} AWS proporciona una infraestructura global y masiva en la nube que permite experimentar e iterar con rapidez. Puede implementar nuevas aplicaciones y aumentar su escala en cuanto crezca su carga de trabajo, o bien, reducirla en función de la demanda. Cuenta con redundancia y disponibilidad en todo momento.
	\item \textbf{Accesibilidad y flexibilidad:} AWS es una plataforma independiente del lenguaje y del sistema operativo.
	\item \textbf{Seguridad:} AWS es una plataforma tecnológica, segura y duradera que cuenta con certificaciones y auditorías reconocidas en el sector.
\end{itemize}