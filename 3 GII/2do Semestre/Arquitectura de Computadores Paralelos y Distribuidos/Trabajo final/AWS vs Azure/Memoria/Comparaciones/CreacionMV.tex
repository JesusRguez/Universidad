\chapter{Creación de máquinas virtuales en AWS y Azure}
En este capítulo se describe la creación de máquinas virtuales en los servicios de AWS y Azure para ver la escalabilidad que proporcionan dichos servicios.

Hemos seleccionado las siguientes máquinas virtuales en función del servicio:
\begin{itemize}
%	Cambiar un poco para que no sea plagio del todo
	\item \textbf{AWS:} Hemos seleccionado una máquina EC2, la cual proporciona capacidad de computación escalable en la nube de Amazon Web Services (AWS). El uso de Amazon EC2 elimina la necesidad de invertir inicialmente en hardware, de manera que puede desarrollar e implementar aplicaciones en menos tiempo. Puede usar Amazon EC2 para lanzar tantos servidores virtuales como necesite, configurar la seguridad y las redes y administrar el almacenamiento.
	\item \textbf{Azure:} Hemos seleccionado una máquina B1ls la cual es la opción ideal para servidores web pequeños, bases de datos pequeñas y entornos de desarrollo y pruebas. Ofrece una forma económica de implementar cargas de trabajo que no necesitan el uso pleno de la CPU de forma continuada e irrumpen en su rendimiento.
\end{itemize}
\section{Creación de una máquina virtual en AWS}


\section{Creación de una máquina virtual en Azure}
La máquina B1ls de Azure seleccionada cuenta con las siguientes especificaciones técnicas:
\begin{itemize}
	\item Un VCPU.
	\item 0.5 GB de RAM.
	\item 4 GB de almacenamiento (HDD o SSD).
	\item 200 MB de transferencia.
\end{itemize}

Para crear el servicio, debemos seguir los siguientes pasos:
\begin{enumerate}
	\item 
\end{enumerate}