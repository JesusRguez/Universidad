\documentclass{beamer}

%% Juego de caracteres usado en el archivo fuente: UTF-8
\usepackage{ucs}
\usepackage[utf8x]{inputenc}
\uselanguage{spanish}
%Para la identación del español
\usepackage[spanish]{babel}

% There are many different themes available for Beamer. A comprehensive
% list with examples is given here:
% http://deic.uab.es/~iblanes/beamer_gallery/index_by_theme.html
% You can uncomment the themes below if you would like to use a different
% one:
%\usetheme{AnnArbor}
%\usetheme{Antibes}
%\usetheme{Bergen}
%\usetheme{Berkeley}
%\usetheme{Berlin}
%\usetheme{Boadilla}
%\usetheme{boxes}
%\usetheme{CambridgeUS}
%\usetheme{Copenhagen}
%\usetheme{Darmstadt}
%\usetheme{default}
%\usetheme{Frankfurt}
%\usetheme{Goettingen}
%\usetheme{Hannover}
%\usetheme{Ilmenau}
%\usetheme{JuanLesPins}
%\usetheme{Luebeck}
\usetheme{Madrid}
%\usetheme{Malmoe}
%\usetheme{Marburg}
%\usetheme{Montpellier}
%\usetheme{PaloAlto}
%\usetheme{Pittsburgh}
%\usetheme{Rochester}
%\usetheme{Singapore}
%\usetheme{Szeged}
%\usetheme{Warsaw}

%Para la identación del español
\usepackage[spanish]{babel}

\title{SPI Plus}

% A subtitle is optional and this may be deleted
%\subtitle{Optional Subtitle}

\author{Jesús Rodríguez Heras \\ Roberto Muras González \\ Juan Pedro Rodríguez Gracia \\ Juan Antonio Muñoz Sánchez \\ Gabriel Fernando Sánchez Reina}
% - Give the names in the same order as the appear in the paper.
% - Use the \inst{?} command only if the authors have different
%   affiliation.

%\institute[Escuela Superior de Ingeniería] % (optional, but mostly needed)
%{
%  \inst{1}%
%  Department of Computer Science\\
%  University of Somewhere
%  \and
%  \inst{2}%
%  Department of Theoretical Philosophy\\
%  University of Elsewhere}
% - Use the \inst command only if there are several affiliations.
% - Keep it simple, no one is interested in your street address.

\date{31 de mayo de 2018}
% - Either use conference name or its abbreviation.
% - Not really informative to the audience, more for people (including
%   yourself) who are reading the slides online

%\subject{Theoretical Computer Science}
% This is only inserted into the PDF information catalog. Can be left
% out. 

% If you have a file called "university-logo-filename.xxx", where xxx
% is a graphic format that can be processed by latex or pdflatex,
% resp., then you can add a logo as follows:

% pgfdeclareimage[height=0.5cm]{university-logo}{university-logo-filename}
% \logo{\pgfuseimage{university-logo}}

% Delete this, if you do not want the table of contents to pop up at
% the beginning of each subsection:
%\AtBeginSubsection[]
%{
%  \begin{frame}<beamer>{Índice}
%    \tableofcontents[currentsection,currentsubsection]
%  \end{frame}
%}

% Let's get started
\begin{document}

\begin{frame}
  \titlepage
  
\end{frame}

\begin{frame}{Índice}
  \tableofcontents
  % You might wish to add the option [pausesections]
\end{frame}

% Section and subsections will appear in the presentation overview
% and table of contents.

\section{Características}
\begin{frame}{Características}
	Las características principales de nuestro bus SPI Plus son:
	\begin{itemize}
		\item Tiene en cuenta la disponibilidad del esclavo a la hora de recibir datos.
		\item Tiene en cuenta la disponibilidad del maestro a la hora de enviar datos.
		\item Cuenta con una transferencia serie de los datos entre maestro y esclavo.
		\item Cuenta con detección de errores.
		\item El tamaño de la memoria física es de 2 KiB.
		\item El tamaño de la tabla de páginas es de 32 entradas.
		\item el tamaño de la tabla de segmentos es de 4 entradas.
		\item El tamaño del buffer del bus es de 1 KiB.
	\end{itemize}
\end{frame}

\section{Implementación}
%diagrama de implementación (el debloques)
\begin{frame}{Implementación}
	\begin{center}
		\includegraphics[scale=0.6]{Diagrama_general.png}
	\end{center}
\end{frame}

\begin{frame}{Implementación}
\begin{center}
	\includegraphics[scale=0.6]{Diagrama_DMA.png}
\end{center}
\end{frame}

\begin{frame}{Implementación}
\begin{center}
	\includegraphics[scale=0.3]{Diagrama_BUS_SPIPlus.png}
\end{center}
\end{frame}

\begin{frame}{Implementación}
\begin{center}
	\includegraphics[scale=0.65]{Diagrama_maestro.png}
\end{center}
\end{frame}

\begin{frame}{Implementación}
\begin{center}
	\includegraphics[scale=0.65]{Diagrama_esclavo.png}
\end{center}
\end{frame}
\section{Funcionamiento}
%diagrama de funcionamiento (pseudocódigo raro)
\begin{frame}{Funcionamiento}
	\begin{center}
		\includegraphics[scale=0.38]{Funcionamiento_DMA.png}
	\end{center}
\end{frame}

\begin{frame}{Funcionamiento}
\begin{center}
	\includegraphics[scale=0.54]{Funcionamiento_SPIPlus.png}
\end{center}
\end{frame}

\end{document}


