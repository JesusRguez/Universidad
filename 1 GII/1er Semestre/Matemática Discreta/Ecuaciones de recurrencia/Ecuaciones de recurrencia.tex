%%\documentclass[a4paper,12pt,oneside]{llncs}
\documentclass[12pt,letterpaper]{article}
\usepackage[right=2cm,left=3cm,top=2cm,bottom=2cm,headsep=0cm]{geometry}

%%%%%%%%%%%%%%%%%%%%%%%%%%%%%%%%%%%%%%%%%%%%%%%%%%%%%%%%%%%
%% Juego de caracteres usado en el archivo fuente: UTF-8
\usepackage{ucs}
\usepackage[utf8x]{inputenc}

%%%%%%%%%%%%%%%%%%%%%%%%%%%%%%%%%%%%%%%%%%%%%%%%%%%%%%%%%%%
%% Juego de caracteres usado en la salida dvi
%% Otra posibilidad: \usepackage{t1enc}
\usepackage[T1]{fontenc}

%%%%%%%%%%%%%%%%%%%%%%%%%%%%%%%%%%%%%%%%%%%%%%%%%%%%%%%%%%%
%% Ajusta maergenes para a4
%\usepackage{a4wide}

%%%%%%%%%%%%%%%%%%%%%%%%%%%%%%%%%%%%%%%%%%%%%%%%%%%%%%%%%%%
%% Uso fuente postscript times, para que los ps y pdf queden y pequeños...
\usepackage{times}

%%%%%%%%%%%%%%%%%%%%%%%%%%%%%%%%%%%%%%%%%%%%%%%%%%%%%%%%%%%
%% Posibilidad de hipertexto (especialmente en pdf)
\usepackage{hyperref}

%%%%%%%%%%%%%%%%%%%%%%%%%%%%%%%%%%%%%%%%%%%%%%%%%%%%%%%%%%%
%% Graficos 
\usepackage{graphics,graphicx}

%%%%%%%%%%%%%%%%%%%%%%%%%%%%%%%%%%%%%%%%%%%%%%%%%%%%%%%%%%%
%% Ciertos caracteres "raros"...
\usepackage{latexsym}

%%%%%%%%%%%%%%%%%%%%%%%%%%%%%%%%%%%%%%%%%%%%%%%%%%%%%%%%%%%
%% Matematicas aun más fuertes (american math dociety)
\usepackage{amsmath}

%%%%%%%%%%%%%%%%%%%%%%%%%%%%%%%%%%%%%%%%%%%%%%%%%%%%%%%%%%%
\usepackage{multirow} % para las tablas
\usepackage[spanish,es-tabla]{babel}

%%%%%%%%%%%%%%%%%%%%%%%%%%%%%%%%%%%%%%%%%%%%%%%%%%%%%%%%%%%
%% Fuentes matematicas lo mas compatibles posibles con postscript (times)
%% (Esto no funciona para todos los simbolos pero reduce mucho el tamaño del
%% pdf si hay muchas matamaticas....
\usepackage{mathptm}

%%% VARIOS:
\usepackage{slashbox}
\usepackage{verbatim}
\usepackage{array}
\usepackage{listings}
\usepackage{multirow}
\usepackage{dsfont}

%% MARCA DE AGUA
%% Este package de "draft copy" NO funciona con pdflatex
%%\usepackage{draftcopy}
%% Este package de "draft copy" SI funciona con pdflatex
%%%\usepackage{pdfdraftcopy}
%%%%%%%%%%%%%%%%%%%%%%%%%%%%%%%%%%%%%%%%%%%%%%%%%%%%%%%%%%%
%% Indenteacion en español...
\usepackage[spanish]{babel}

\usepackage{listings}
% Para escribir código en C
% \begin{lstlisting}[language=C]
% #include <stdio.h>
% int main(int argc, char* argv[]) {
% puts("Hola mundo!");
% }
% \end{lstlisting}


\title{¿Cómo resolver ecuaciones de recurrencia homogéneas y no homogéneas paso a paso?}
\author{Jesús Rodríguez Heras}

\begin{document}
	\maketitle
	\thispagestyle{empty}
	\newpage
	
	\tableofcontents
	\newpage
	
	%%\listoftables
	%%\newpage
	
	%%\listoffigures
	%%\newpage
	
	%%%% REAL WORK BEGINS HERE:
	
	%%Configuracion del paquete listings
	\lstset{language=bash, numbers=left, numberstyle=\tiny, numbersep=10pt, firstnumber=1, stepnumber=1}

\part{Homogéneas}
\noindent
Utilizaremos el ejemplo siguiente (sacado de los apuntes) para ver los pasos de la resolución de una ecuación de recurrencia homogénea:\\\\
\textbf{Resolver la ecuación de recurrencia:
	\begin{center}
		$a_{n+2}+2a_{n+1}-3a_{n}=0, n\geq1$
	\end{center}
	con las condiciones iniciales, $a_{1}=0$ y $a_{2}=-12$.}
\section{Escribimos la ecuación dada en su forma general:}
\begin{center}
	$a_{1}=0$\\
	$a_{2}=-12$\\
	$a_{n+2}+2a_{n+1}-3a_{n}=0$
\end{center}

\section{Obtenemos la ecuación característica de la ecuación dada:}
\begin{center}
	$\lambda^{2}+2\lambda-3=0$
\end{center}
Resolviendo dicha ecuación característica obtenemos:
\begin{center}
	$\lambda=1$ o $\lambda=-3$
\end{center}

\section{La solución general de la ecuación característica es la sucesión \{$a_n$\}, tal que:}
\begin{center}
	$a_{n}=n^q\cdot\lambda^n$, $0\leq$$q$$\leq$$m-1$
\end{center}
Por lo tanto, siguiendo el ejemplo, nuestro $a_{n}$ será:
$a_{n}=\alpha_{1}+\alpha_{2}\cdot (-3)^n$, $n\leq 1$ y $\alpha_{1}, \alpha_{2} \in \mathds{R}$

\section{Obtenemos la solución única de la ecuación propuesta mediante las condiciones iniciales dadas:}
$a_1=0$ $\Rightarrow$ $\alpha_1-3\alpha_2=0$ $\Rightarrow$ $-\alpha_1+3\alpha_2=0$\\
$a_2=-12$ $\Rightarrow$ $\alpha_1+9\alpha_2=-12$ $\Rightarrow$ $\alpha_1+9\alpha_2=-12$\\
Despejando $\alpha_1$ tenemos: $\alpha_2=-1$.\\
Sustituyendo y despejando tenemos: $\alpha_1=-3$.

\section{Por lo tanto, la solución única a la ecuación propuesta es la sucesión \{$a_n$\}, tal que:}
\begin{center}
	$a_n=-3-1(-3)^n$, $n\geq1$
\end{center}

\newpage
\setcounter{section}{0}
\part{No homogéneas}
Utilizaremos el ejemplo siguiente (sacado de los apuntes) para ver los pasos de la resolución de una ecuación de recurrencia no homogénea:\\\\
\textbf{Resolver la ecuación de recurrencia:
	\begin{center}
		$a_{n+2}-2a_{n+1}+a_{n}=1, n\geq1$
	\end{center}
	con las condiciones iniciales, $a_{1}=1$ y $a_{2}=0$.}

\section{Escribimos la ecuación dada en su forma general:}
\begin{center}
	$a_{1}=1$\\
	$a_{2}=0$\\
	$a_{n+2}-2a_{n+1}+a_{n}=1$
\end{center}

\section{Obtenemos la ecuación homogénea asociada a la ecuación propuesta:}
\begin{center}
	$a_{n+2}-2a_{n+1}+a_{n}=0$
\end{center}

\section{Obtenemos la ecuación característica de la ecuación homogénea asociada:}
\begin{center}
	$\lambda^{2}-2\lambda+1=0$
\end{center}
\noindent
Resolviendo dicha ecuación característica obtenemos:
\begin{center}
	$\lambda=1$ o $\lambda=1$\\
\end{center}
Es decir, $\lambda=1$ com multiplicidad $m=2$.

\section{La solución general de la ecuación homogénea asociada será la sucesión \{$a_{n}^{(h)}$\}, tal que:}
\begin{center}
	$a_{n}^{(h)}=n^q\cdot\lambda^n$, $0\leq$$q$$\leq$$m-1$
\end{center}
Por lo tanto, siguiendo el ejemplo, nuestro $a_{n}^{(h)}$ será:
$a_{n}^{(h)}=\alpha_{1}+\alpha_{2}\cdot n$, $n\leq 1$ y $\alpha_{1}, \alpha_{2} \in \mathds{R}$

\section{Obtenemos la solución particular de la ecuación propuesta usando el método de los coeficientes indeterminados:}
\noindent
Usando el término $h(n)$ de la ecuación dada, tenemos:
\begin{center}
	$h(n)=r^n\cdot p_{0}$\\
	$h(n)=1\Rightarrow h(n)=1^n\cdot 1$
\end{center}
Luego, $r=1$.\\\\
Llegados a este punto se nos presentan dos casos:

\subsection{r no es raíz de la ecuación característica de la homogénea asociada:}
\noindent
Si se da este caso, ontinuaremos de la siguiente forma:\\
$a_n^{(p)}=r^n(A_0+A_1n+A_2n^2+A_3n^3+...+A_tn^t)$\\
Este caso no se da en este ejemplo pero se continuaría igual con el punto 6 en adelante.

\subsection{r sí es raíz de la ecuación característica de la homogénea asociada:}
\noindent
Si se da este caso, ontinuaremos de la siguiente forma:\\
$a_n^{(p)}=n^m\cdot r^n(A_0+A_1n+A_2n^2+A_3n^3+...+A_tn^t)$, siendo $m$ la multiplicidad.\\
Este es el caso que nos plantea este ejercicio ya que $r=\lambda$, por lo tanto:\\
\begin{center}
	$a_n^{(p)}=n^2\cdot 1^n(A_0)$ $\Rightarrow$ $a_n^{(p)}=n^2\cdot A_0$
\end{center}

\section{Sustituyendo en la ecuación dada obtendremos su solución particular:}
\noindent
\begin{center}
	$a_{n+2}^{(p)}-2a_{n+1}^{(p)}+a_n^{(p)}=1 \Rightarrow (n+2)^2A_0-2(n+1)^2A_0+n^2A_0=1$ $\Rightarrow$\\
	$\Rightarrow$ $(n^2+4n+4)A_0-2(n^2+2n+1)A_0+n^2A-0=1$ $\Rightarrow$\\
	$\Rightarrow$ $n^2A_0+4nA_0+4A_0-2n^2A_0-4nA_0-2A_0-n^2A_0=1$ $\Rightarrow$\\
	$\Rightarrow$ $2A_0=1$ $\Rightarrow$\\
	$\Rightarrow$ $A_0=\frac{1}{2}$
\end{center}

\section{La solución particular de la ecuación propuesta es la sucesión\{$a_n^{(p)}$\}, tal que:}
\begin{center}
	$a_n^{(p)}=\frac{1}{2}\cdot n^2$, $n\geq 1$
\end{center}

\section{La solución general de la ecuación propuesta es la sucesión \{$a_n$\}, tal que:}
\begin{center}
	$a_n=a_n^{(h)}+a_n^{(p)}$ $\Longrightarrow$ $a_n=\alpha_1+\alpha_2n+\frac{1}{2}n^2$, $n\geq 1$
\end{center}

\section{Obtendremos la solución única de la ecuación propuesta mediante las condiciones iniciales dadas:}
\noindent
$a_1=1$ $\Rightarrow$ $\alpha_1+\alpha_2+\frac{1}{2}=1$ $\Longrightarrow$ $\alpha_1+\alpha_2+\frac{1}{2}=1$\\
$a_2=0$ $\Rightarrow$ $\alpha_1+2\alpha_2+2^2\cdot \frac{1}{2}=0$ $\Rightarrow$ $-\alpha_1-2\alpha_2-2=0$\\
Despejando $\alpha_1$ tenemos: $\alpha_2=-\frac{5}{2}$.\\
Sustituyendo y despejando tenemos: $\alpha_1=3$.

\section{Por lo tanto, la solución única a la ecuación propuesta es la sucesión \{$a_n$\}, tal que:}
\begin{center}
	$a_n=3-\frac{5}{2}n+\frac{1}{2}n^2$, $n\geq1$
\end{center}

\end{document}
