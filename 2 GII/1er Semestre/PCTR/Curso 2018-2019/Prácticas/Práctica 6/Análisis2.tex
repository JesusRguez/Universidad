%%\documentclass[a4paper,12pt,oneside]{llncs}
\documentclass[12pt,letterpaper]{article}
\usepackage[right=2cm,left=3cm,top=2cm,bottom=2cm,headsep=0cm]{geometry}

%%%%%%%%%%%%%%%%%%%%%%%%%%%%%%%%%%%%%%%%%%%%%%%%%%%%%%%%%%%
%% Juego de caracteres usado en el archivo fuente: UTF-8
\usepackage{ucs}
\usepackage[utf8x]{inputenc}

%%%%%%%%%%%%%%%%%%%%%%%%%%%%%%%%%%%%%%%%%%%%%%%%%%%%%%%%%%%
%% Juego de caracteres usado en la salida dvi
%% Otra posibilidad: \usepackage{t1enc}
\usepackage[T1]{fontenc}

%%%%%%%%%%%%%%%%%%%%%%%%%%%%%%%%%%%%%%%%%%%%%%%%%%%%%%%%%%%
%% Ajusta maergenes para a4
%\usepackage{a4wide}

%%%%%%%%%%%%%%%%%%%%%%%%%%%%%%%%%%%%%%%%%%%%%%%%%%%%%%%%%%%
%% Uso fuente postscript times, para que los ps y pdf queden y pequeños...
\usepackage{times}

%%%%%%%%%%%%%%%%%%%%%%%%%%%%%%%%%%%%%%%%%%%%%%%%%%%%%%%%%%%
%% Posibilidad de hipertexto (especialmente en pdf)
%\usepackage{hyperref}
\usepackage[bookmarks = true, colorlinks=true, linkcolor = black, citecolor = black, menucolor = black, urlcolor = black]{hyperref}

%%%%%%%%%%%%%%%%%%%%%%%%%%%%%%%%%%%%%%%%%%%%%%%%%%%%%%%%%%%
%% Graficos 
\usepackage{graphics,graphicx}

%%%%%%%%%%%%%%%%%%%%%%%%%%%%%%%%%%%%%%%%%%%%%%%%%%%%%%%%%%%
%% Ciertos caracteres "raros"...
\usepackage{latexsym}

%%%%%%%%%%%%%%%%%%%%%%%%%%%%%%%%%%%%%%%%%%%%%%%%%%%%%%%%%%%
%% Matematicas aun más fuertes (american math dociety)
\usepackage{amsmath}

%%%%%%%%%%%%%%%%%%%%%%%%%%%%%%%%%%%%%%%%%%%%%%%%%%%%%%%%%%%
\usepackage{multirow} % para las tablas
\usepackage[spanish,es-tabla]{babel}

%%%%%%%%%%%%%%%%%%%%%%%%%%%%%%%%%%%%%%%%%%%%%%%%%%%%%%%%%%%
%% Fuentes matematicas lo mas compatibles posibles con postscript (times)
%% (Esto no funciona para todos los simbolos pero reduce mucho el tamaño del
%% pdf si hay muchas matamaticas....
\usepackage{mathptm}

%%% VARIOS:
%\usepackage{slashbox}
\usepackage{verbatim}
\usepackage{array}
\usepackage{listings}
\usepackage{multirow}

%% MARCA DE AGUA
%% Este package de "draft copy" NO funciona con pdflatex
%%\usepackage{draftcopy}
%% Este package de "draft copy" SI funciona con pdflatex
%%%\usepackage{pdfdraftcopy}
%%%%%%%%%%%%%%%%%%%%%%%%%%%%%%%%%%%%%%%%%%%%%%%%%%%%%%%%%%%
%% Indenteacion en español...
\usepackage[spanish]{babel}

\usepackage{listings}
% Para escribir código en C
% \begin{lstlisting}[language=C]
% #include <stdio.h>
% int main(int argc, char* argv[]) {
% puts("Hola mundo!");
% }
% \end{lstlisting}


\title{Análisis2}
\author{Jesús Rodríguez Heras}

\begin{document}
	
	\maketitle
	\begin{abstract} %Poner esto en todas las prácticas de PCTR
		\begin{center}
			Comparativa del SpeedUp del producto de una matriz por un vector.
		\end{center}
	\end{abstract}
	\thispagestyle{empty}
	\newpage
	
	%\tableofcontents
	\newpage
	
	%%\listoftables
	%%\newpage
	
	%%\listoffigures
	%%\newpage
	
	%%%% REAL WORK BEGINS HERE:
	
	%%Configuracion del paquete listings
	\lstset{language=bash, numbers=left, numberstyle=\tiny, numbersep=10pt, firstnumber=1, stepnumber=1, basicstyle=\small\ttfamily, tabsize=1, extendedchars=true, inputencoding=latin1}

\begin{center}
	\begin{table}[htbp]
		\begin{center}
			\begin{tabular}{|c|c|c|c|}
				\hline
				\textbf{Elementos} & \textbf{matVector} & \textbf{matVectorConcurrente} & \textbf{matVectorDenso}  \\
				\hline 
				$1000$ & 0.008 & 0.075 & 0.045 \\ \hline 
				$2000$ & 0.01 & 0.119 & 0.055 \\ \hline  
				$3000$ & 0.013 & 0.118 & 0.034 \\ \hline 
				$4000$ & 0.018 & 0.183 & 0.046 \\ \hline 
				$5000$ & 0.026 & 0.24 & 0.046 \\ \hline  
				$6000$ & 0.036 & 0.275 & 0.048 \\ \hline 
				$7000$ & 0.046 & 0.302 & 0.049 \\ \hline 
				$8000$ & 0.058 & 0.408 & 0.059 \\ \hline 
				$9000$ & 0.071 & 0.434 & 0.073 \\ \hline 
				$10000$ & 0.087 & 0.489 & 0.084 \\ \hline
			\end{tabular}
			\caption{Valores en segundos del tiempo usado por cada algoritmo.}
			\label{tabla:Valores en segundos del tiempo usado por cada algoritmo}
		\end{center}
	\end{table}
\end{center}

\begin{figure}[h]
	\begin{center}
				\includegraphics[scale=1]{SpeedUp1.png}
		\caption{Valores del SpeedUp.}
		\label{fig:Valores del SpeedUp}
	\end{center}
\end{figure}
Para la realización de la gráfica inicial, hemos tenido en cuenta dichos datos tomados en tiempo con los que calculamos el Speed-Up de los algoritmos de grano fino (\texttt{matVectorConcurrente.java}) y de grano grueso (\texttt{matVectorDenso.java}) respecto del algoritmo secuencial (\texttt{matVector.java}).

Como podemos apreciar en la imagen, el Speed-Up no empieza a ser notable hasta los 7000 elementos y, donde verdaderamente se le saca partido es a partir de los 10000 elementos donde alcanza un Speed-Up de $1.43$ y, como puede observarse, su tendencia nos indica que seguirá subiendo conforme aumentemos el número de elementos de las matrices. Esta prueba no la hemos podido realizar debido a que con más de 10000 elementos, se sobrepasa la memoria del heap de Java en el algoritmo de grano grueso \texttt{matVectorDenso.java}.

	

\end{document}