%%\documentclass[a4paper,12pt,oneside]{llncs}
\documentclass[12pt,letterpaper]{article}
\usepackage[right=2cm,left=3cm,top=2cm,bottom=2cm,headsep=0cm]{geometry}

%%%%%%%%%%%%%%%%%%%%%%%%%%%%%%%%%%%%%%%%%%%%%%%%%%%%%%%%%%%
%% Juego de caracteres usado en el archivo fuente: UTF-8
\usepackage{ucs}
\usepackage[utf8x]{inputenc}

%%%%%%%%%%%%%%%%%%%%%%%%%%%%%%%%%%%%%%%%%%%%%%%%%%%%%%%%%%%
%% Juego de caracteres usado en la salida dvi
%% Otra posibilidad: \usepackage{t1enc}
\usepackage[T1]{fontenc}

%%%%%%%%%%%%%%%%%%%%%%%%%%%%%%%%%%%%%%%%%%%%%%%%%%%%%%%%%%%
%% Ajusta maergenes para a4
%\usepackage{a4wide}

%%%%%%%%%%%%%%%%%%%%%%%%%%%%%%%%%%%%%%%%%%%%%%%%%%%%%%%%%%%
%% Uso fuente postscript times, para que los ps y pdf queden y pequeños...
\usepackage{times}

%%%%%%%%%%%%%%%%%%%%%%%%%%%%%%%%%%%%%%%%%%%%%%%%%%%%%%%%%%%
%% Posibilidad de hipertexto (especialmente en pdf)
%\usepackage{hyperref}
\usepackage[bookmarks = true, colorlinks=true, linkcolor = black, citecolor = black, menucolor = black, urlcolor = black]{hyperref}

%%%%%%%%%%%%%%%%%%%%%%%%%%%%%%%%%%%%%%%%%%%%%%%%%%%%%%%%%%%
%% Graficos 
\usepackage{graphics,graphicx}

%%%%%%%%%%%%%%%%%%%%%%%%%%%%%%%%%%%%%%%%%%%%%%%%%%%%%%%%%%%
%% Ciertos caracteres "raros"...
\usepackage{latexsym}

%%%%%%%%%%%%%%%%%%%%%%%%%%%%%%%%%%%%%%%%%%%%%%%%%%%%%%%%%%%
%% Matematicas aun más fuertes (american math dociety)
\usepackage{amsmath}

%%%%%%%%%%%%%%%%%%%%%%%%%%%%%%%%%%%%%%%%%%%%%%%%%%%%%%%%%%%
\usepackage{multirow} % para las tablas
\usepackage[spanish,es-tabla]{babel}

%%%%%%%%%%%%%%%%%%%%%%%%%%%%%%%%%%%%%%%%%%%%%%%%%%%%%%%%%%%
%% Fuentes matematicas lo mas compatibles posibles con postscript (times)
%% (Esto no funciona para todos los simbolos pero reduce mucho el tamaño del
%% pdf si hay muchas matamaticas....
\usepackage{mathptm}

%%% VARIOS:
%\usepackage{slashbox}
\usepackage{verbatim}
\usepackage{array}
\usepackage{listings}
\usepackage{multirow}

%% MARCA DE AGUA
%% Este package de "draft copy" NO funciona con pdflatex
%%\usepackage{draftcopy}
%% Este package de "draft copy" SI funciona con pdflatex
%%%\usepackage{pdfdraftcopy}
%%%%%%%%%%%%%%%%%%%%%%%%%%%%%%%%%%%%%%%%%%%%%%%%%%%%%%%%%%%
%% Indenteacion en español...
\usepackage[spanish]{babel}

\usepackage{listingsutf8}
% Para escribir código en C
% \begin{lstlisting}[language=C]
% #include <stdio.h>
% int main(int argc, char* argv[]) {
% puts("Hola mundo!");
% }
% \end{lstlisting}


\title{Memoria de prácticas}
\author{Jesús Rodríguez Heras}

\begin{document}
	
	\maketitle
%	\begin{abstract} %Poner esto en todas las prácticas de PCTR
%%		\begin{center}
%%			\noindent
%			
%%		\end{center}
%	\end{abstract}
	\thispagestyle{empty}
	\newpage
	
	\tableofcontents
	\newpage
	
	%%\listoftables
	%%\newpage
	
	%%\listoffigures
	%%\newpage
	
	%%%% REAL WORK BEGINS HERE:
	
	%%Configuracion del paquete listings
	\lstset{language=bash, numbers=left, numberstyle=\tiny, numbersep=10pt, firstnumber=1, stepnumber=1, basicstyle=\small\ttfamily, tabsize=1, extendedchars=true, inputencoding=utf8/latin1, breaklines=true}
	
\section{Datos generales}
\subsection{Datos de contacto de la empresa y tutor laboral}
\begin{itemize}
	\item \textbf{Empresa:} Sea Master Consulting \& Engineering.
	\begin{itemize}
		\item \textbf{Dirección:} C/ Manatial13, Edif. CEEI, Polígono Industrial Las Salinas, 11500,El Puerto de Santa María, Cádiz.
		\item \textbf{Teléfono:} 956 10 11 23.
	\end{itemize}
	\item \textbf{Tutor laboral:} Joaquín Galindo Mengibar (jgalindo@sea-master.eu).
\end{itemize}

\subsection{Datos de contacto del alumno}
\begin{itemize}
	\item \textbf{Nombre:} Jesús Rodríguez Heras
	\item \textbf{DNI:} 32088516C
	\item \textbf{Teléfono:} 628576107
	\item \textbf{Correo electrónico:} jesusrh1997@gmail.com
\end{itemize}

\subsection{Periodo de realización}
\textcolor{red}{¿Qué pongo aquí, solo tres meses?}

\subsection{Horarios (flexibilidad)}


\subsection{Breve descripción de la empresa}



\section{Líneas de trabajo}
\subsection{Actuaciones previstas inicialmente}

\subsection{Actividades que realmente has llevado a cabo en la empresa}

\subsection{Competencias y conocimientos adquiridos durante el transcurso de las prácticas}


\section{Valoración crítica personal}
\subsection{Aspectos más positivos y negativos de las prácticas}

\subsection{Te gustaría poder trabajar en un futuro en la empresa? ¿Recomendarías las prácticas a otros alumnos?}

\subsection{La duración de las prácticas ha sido adecuada? ¿El seguimiento por parte de la empresa ha sido correcto?}

\subsection{Otros aspectos que consideres de interés}


\end{document}