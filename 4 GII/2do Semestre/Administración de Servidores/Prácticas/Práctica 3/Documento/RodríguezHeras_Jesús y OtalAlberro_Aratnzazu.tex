%%\documentclass[a4paper,12pt,oneside]{llncs}
\documentclass[12pt,letterpaper]{article}
\usepackage[right=2cm,left=3cm,top=2cm,bottom=2cm,headsep=0cm]{geometry}

%%%%%%%%%%%%%%%%%%%%%%%%%%%%%%%%%%%%%%%%%%%%%%%%%%%%%%%%%%%
%% Juego de caracteres usado en el archivo fuente: UTF-8
\usepackage{ucs}
\usepackage[utf8x]{inputenc}

%%%%%%%%%%%%%%%%%%%%%%%%%%%%%%%%%%%%%%%%%%%%%%%%%%%%%%%%%%%
%% Juego de caracteres usado en la salida dvi
%% Otra posibilidad: \usepackage{t1enc}
\usepackage[T1]{fontenc}

%%%%%%%%%%%%%%%%%%%%%%%%%%%%%%%%%%%%%%%%%%%%%%%%%%%%%%%%%%%
%% Ajusta maergenes para a4
%\usepackage{a4wide}

%%%%%%%%%%%%%%%%%%%%%%%%%%%%%%%%%%%%%%%%%%%%%%%%%%%%%%%%%%%
%% Uso fuente postscript times, para que los ps y pdf queden y pequeños...
\usepackage{times}

%%%%%%%%%%%%%%%%%%%%%%%%%%%%%%%%%%%%%%%%%%%%%%%%%%%%%%%%%%%
%% Posibilidad de hipertexto (especialmente en pdf)
%\usepackage{hyperref}
\usepackage[bookmarks = true, colorlinks=true, linkcolor = black, citecolor = black, menucolor = black, urlcolor = black]{hyperref}

%%%%%%%%%%%%%%%%%%%%%%%%%%%%%%%%%%%%%%%%%%%%%%%%%%%%%%%%%%%
%% Graficos 
\usepackage{graphics,graphicx}

%%%%%%%%%%%%%%%%%%%%%%%%%%%%%%%%%%%%%%%%%%%%%%%%%%%%%%%%%%%
%% Ciertos caracteres "raros"...
\usepackage{latexsym}

%%%%%%%%%%%%%%%%%%%%%%%%%%%%%%%%%%%%%%%%%%%%%%%%%%%%%%%%%%%
%% Matematicas aun más fuertes (american math dociety)
\usepackage{amsmath}

%%%%%%%%%%%%%%%%%%%%%%%%%%%%%%%%%%%%%%%%%%%%%%%%%%%%%%%%%%%
\usepackage{multirow} % para las tablas
\usepackage[spanish,es-tabla]{babel}

%%%%%%%%%%%%%%%%%%%%%%%%%%%%%%%%%%%%%%%%%%%%%%%%%%%%%%%%%%%
%% Fuentes matematicas lo mas compatibles posibles con postscript (times)
%% (Esto no funciona para todos los simbolos pero reduce mucho el tamaño del
%% pdf si hay muchas matamaticas....
\usepackage{mathptm}

%%% VARIOS:
%\usepackage{slashbox}
\usepackage{verbatim}
\usepackage{array}
\usepackage{listings}
\usepackage{multirow}

%% MARCA DE AGUA
%% Este package de "draft copy" NO funciona con pdflatex
%%\usepackage{draftcopy}
%% Este package de "draft copy" SI funciona con pdflatex
%%%\usepackage{pdfdraftcopy}
%%%%%%%%%%%%%%%%%%%%%%%%%%%%%%%%%%%%%%%%%%%%%%%%%%%%%%%%%%%
%% Indenteacion en español...
\usepackage[spanish]{babel}

\usepackage{listings}
% Para escribir código en C
% \begin{lstlisting}[language=C]
% #include <stdio.h>
% int main(int argc, char* argv[]) {
% puts("Hola mundo!");
% }
% \end{lstlisting}


\title{Práctica 3}
\author{Jesús Rodríguez Heras\\
	Arantzazu Otal Alberro}

\begin{document}
	
	\maketitle
%	\begin{abstract} %Poner esto en todas las prácticas de PCTR
%%		\begin{center}
%%			\noindent
%			
%%		\end{center}
%	\end{abstract}
	\thispagestyle{empty}
	\newpage
	
%	\tableofcontents
%	\newpage
	
	%%\listoftables
	%%\newpage
	
	%%\listoffigures
	%%\newpage
	
	%%%% REAL WORK BEGINS HERE:
	
	%%Configuracion del paquete listings
	\lstset{language=bash, numbers=left, numberstyle=\tiny, numbersep=10pt, firstnumber=1, stepnumber=1, basicstyle=\small\ttfamily, tabsize=1, extendedchars=true, inputencoding=latin1}
	
\section{Instalación de máquinas virtuales mediante Vagrant}
En esta primera parte vamos a crear el entorno de trabajo, consiste en dos redes internas, conectadas al exterior mediante un router.
\begin{itemize}
	\item La primera red tendrá el rango de IPs 192.168.2.0.
	\item La segunda red tendrá el rango de IPs 192.168.3.0.
\end{itemize}

Cada red tendrá un par de máquinas virtuales (no hace falta conectarlas todas de forma simultánea). Además, las redes solo tendrán acceso al exterior a través de la máquina que actúa como router.

En este ejercicio se deberá:
\begin{itemize}
	\item \textbf{Crear el entorno de red mediante un único fichero Vagrant.}
	\lstinputlisting[language=Ruby]{../Vagrantfile1}
	
	\item \textbf{Configurar el cortafuegos para que de acceso al exterior.} \\
	Para configurar el cortafuegos, primero debemos deshabilitar la interfaz de red que nos permite salir a Internet desde cada una de las máquinas de las redes. Para ello, identificamos la interfaz que queremos deshabilitar y usamos el comando \texttt{sudo ifconfig eth0 down}.
	
	También debemos activar el ip forward en el router para tener conectividad entre las máquinas de las diferentes redes. Para ello usamos el comando\\\texttt{sudo echo 1 > /proc/sys/net/ipv4/ip\_forward}.
	
	A continuación, establecemos la puerta de enlace como la interfaz que tenemos en la máquina que hace de router. Para ello usamos el comando \texttt{sudo add default gw 192.168.x.1}, siendo \texttt{x} la red a la que pertenece cada máquina.
	
	Lo siguiente es habilitar el enrutamiento a partir de la máquina router al resto de máquinas con el comando\\\texttt{sudo iptables -t nat -A POSTROUTING -o eth0 -j MASQUERADE}.
	
	\item \textbf{Configurar manualmente los clientes de las redes para que se puedan conectar al servidor.} \\
	Solo con poner la puerta de enlace y el ip forwarding estaría hecho y ya se ha hecho en el apartado anterior.
\end{itemize}

\section{Servidor DHCP}
Instalar un servidor DHCP en el cortafuegos. Además, se deberá modificar el fichero Vagrant, para que en lugar de establecer una IP privada, la IP se asigne mediante DHCP.

%Cuando cambie el vagranfile poner:
%\lstinputlisting[language=Ruby]{Vagrantfile2}

También se puede probar dejando la IP privada y comprobando el funcionamiento del servidor DHCP mediante el cliente DHCP.

El servidor DHCP deberá asignar direcciones IP a cada una de las redes internas. Además, una máquina de la segunda red tendrá que tener una dirección fija.

Tras la configuración, mostrar el estado de los prestamos realizados por el servidor DHCP.

Para instalar el servidor DHCP introducimos el siguiente comando \texttt{sudo apt-get install isc-dhcp-server}.

\end{document}