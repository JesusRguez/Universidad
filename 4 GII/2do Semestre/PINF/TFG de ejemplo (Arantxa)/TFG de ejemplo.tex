\documentclass[12pt,spanish,intoc,bibtotoc,idxtotoc,BCOR0mm,tablecaptionabove]{scrbook}
\usepackage[utf8]{luainputenc}
\usepackage[a4paper]{geometry}
\geometry{verbose,lmargin=3.5cm,rmargin=2.5cm}
\usepackage{color}
\definecolor{gray}{rgb}{0.4,0.4,0.4}
\definecolor{darkblue}{rgb}{0.0,0.0,0.6}
\definecolor{cyan}{rgb}{0.0,0.6,0.6}
\usepackage[table,xcdraw]{xcolor}
\usepackage{babel}
\addto\shorthandsspanish{\spanishdeactivate{~<>}}

\usepackage{rotating}
\usepackage{longtable}
\usepackage{booktabs}
\usepackage{graphicx}
\usepackage[unicode=true,pdfusetitle,
 bookmarks=true,bookmarksnumbered=true,bookmarksopen=false,
 breaklinks=true,pdfborder={0 0 1},backref=false,colorlinks=true]
 {hyperref}
\hypersetup{
 linkcolor=black, citecolor=black, urlcolor=blue, filecolor=blue,pdfpagelayout=OneColumn, pdfnewwindow=true,pdfstartview=XYZ, plainpages=false, pdfpagelabels,pdftex}

\makeatletter

\setcounter{tocdepth}{4}
\setcounter{secnumdepth}{4}

%%%%%%%%%%%%%%%%%%%%%%%%%%%%%% LyX specific LaTeX commands.
\providecommand{\LyX}{\texorpdfstring%
  {L\kern-.1667em\lower.25em\hbox{Y}\kern-.125emX\@}
  {LyX}}
%% Because html converters don't know tabularnewline
\providecommand{\tabularnewline}{\\}

%%%%%%%%%%%%%%%%%%%%%%%%%%%%%% User specified LaTeX commands.
% DO NOT ALTER THIS PREAMBLE!!!
%
% This preamble is designed to ensure that the User's Guide prints
% out as advertised. If you mess with this preamble,
% parts of the User's Guide may not print out as expected.  If you
% have problems LaTeXing this file, please contact 
% the documentation team
% email: lyx-docs@lists.lyx.org

%\usepackage[utf8]{inputenc}

\usepackage{ifpdf} % part of the hyperref bundle
\ifpdf % if pdflatex is used

 % set fonts for nicer pdf view
 \IfFileExists{lmodern.sty}{\usepackage{lmodern}}{}

\fi % end if pdflatex is used

% for correct jump positions whe clicking on a link to a float
\usepackage{float}
\usepackage[figure]{hypcap}
%\usepackage[printonlyused]{acronym-custom}
\usepackage{acronym-custom}

% the pages of the TOC is numbered roman
% and a pdf-bookmark for the TOC is added
\let\myTOC\tableofcontents
\renewcommand\tableofcontents{%
  \frontmatter
  \pdfbookmark[1]{\contentsname}{}
  \myTOC
  \mainmatter }

% redefine the \LyX macro for PDF bookmarks
\def\LyX{\texorpdfstring{%
  L\kern-.1667em\lower.25em\hbox{Y}\kern-.125emX\@}
  {LyX}}

% define a short command for \textvisiblespace
\newcommand{\spce}{\textvisiblespace}

% macro for italic page numbers in the index
\newcommand{\IndexDef}[1]{\textit{#1}}

% redefine the greyed out note
%\renewenvironment{lyxgreyedout}
 %{\textcolor{blue}\bgroup}{\egroup}

% workaround for a makeindex bug, 
% see sec. "Index Entry Order"
% only uncomment this when you are using makindex
%\let\OrgIndex\index 
%\renewcommand*{\index}[1]{\OrgIndex{#1}}

\usepackage{graphicx}

\usepackage{listings}
\renewcommand{\lstlistlistingname}{Listados}
\renewcommand{\lstlistingname}{Listado}

\usepackage{hyphenat}
\usepackage{verbatim}

\AtBeginDocument{
   \renewcommand{\tablename}{Tabla}
   \renewcommand{\listtablename}{�ndice de tablas}
}

\newenvironment{changemargin}[2]{%
  \begin{list}{}{%
    \setlength{\topsep}{0pt}%
    \setlength{\leftmargin}{#1}%
    \setlength{\rightmargin}{#2}%
    \setlength{\listparindent}{\parindent}%
    \setlength{\itemindent}{\parindent}%
    \setlength{\parsep}{\parskip}%
  }%
  \item[]}{\end{list}}

\newenvironment{nota}{
  \begin{changemargin}{2em}{2em}
    \textbf{\textsc{Nota: }}
}{
  \end{changemargin}
}

\lstset{
  basicstyle=\ttfamily,
  columns=fullflexible,
  showstringspaces=false,
  commentstyle=\color{gray}\upshape
}

\lstdefinelanguage{XML}
{
  morestring=[b]",
  morestring=[s]{>}{<},
  morecomment=[s]{<?}{?>},
  stringstyle=\color{black},
  identifierstyle=\color{darkblue},
  keywordstyle=\color{cyan},
  morekeywords={xmlns,version,type,name,bpel,partnerLink,portType,operation,variable,createInstance, inputVariable, outputVariable, validate, part}% list your attributes here
}

\makeatother

\begin{document}
\begin{titlepage}
  \centering
  \includegraphics[width=.25\textwidth]{logo_uca}

  \bigskip
  \bigskip
  \bigskip
  
  \begin{changemargin}{3em}{3em}
    \centering

    {\LARGE \textsc{\nohyphens{Escuela Superior de Ingenier�a}}}
    
    \bigskip
    \bigskip
    \bigskip

    {\LARGE \nohyphens{Grado en Ingenier�a Inform�tica}}

    \bigskip
    \bigskip
    \bigskip
    \bigskip
    \bigskip
    \bigskip

    {\LARGE \nohyphens{Reingenier�a de Software para An�lisis Arm�nico de las Mareas}}

    \bigskip
    \bigskip
    \bigskip
    \bigskip
    \bigskip

    {\large Curso 2018-2019}

    \bigskip
    \bigskip
    \bigskip
    \bigskip
    \bigskip
    \bigskip
      
    \end{changemargin}

    {\Large Arantzazu Otal Alberro \\}
    \bigskip
    \bigskip 
    \bigskip 
    {\large Puerto Real, \today}

\end{titlepage}
\newpage{\pagestyle{empty}\cleardoublepage}  
{
  \thispagestyle{empty} 
  \centering
  \includegraphics[width=.2\textwidth]{logo_uca}

  \bigskip
  \bigskip
  \bigskip
  
  \begin{changemargin}{3em}{3em}

    \begin{center}
      {\LARGE \textsc{\nohyphens{Escuela Superior de Ingenier�a}}}
      
      \bigskip
      \bigskip
      
      {\LARGE \nohyphens{Grado en Ingenier�a Inform�tica}}
      
      \bigskip
      \bigskip
      \bigskip
      \bigskip
      
      {\LARGE \nohyphens{Reingenier�a de Software para An�lisis Arm�nico de Mareas}}
      
      \bigskip
      \bigskip
      \bigskip
      \bigskip
      
    \end{center}
  \end{changemargin}

  \begin{flushleft}
    \Large

    \textsc{Departamento}: \nohyphens{Ingenier�a Inform�tica.} \\
    \textsc{Directora del proyecto}: \nohyphens{M� del Carmen de Castro Cabrera} \\
    \textsc{Autor del Proyecto}: \nohyphens{Arantzazu Otal Alberro}. \\
  \end{flushleft}
  
  \bigskip
  \bigskip
  \bigskip
  
  \begin{flushright}
    \large
    Puerto Real, \today

    \bigskip    
    \bigskip
    \bigskip
    \bigskip
    \bigskip
    \bigskip
    \bigskip
    \bigskip
    Fdo.: Arantzazu Otal Alberro
    
  \end{flushright}

}


\input{Conclusiones/Agradecimientos.tex}


\clearpage
\newpage{\pagestyle{empty}\cleardoublepage}  



\tableofcontents{}

\listoffigures


\listoftables


%----------------------------


\chapter{Introducci�n\label{cap:introducci=0000F3n}}
\chapter{Introducción}
\section{Motivación}
Es un trabajo de la asignatura ``Proyectos Informáticos'' que, a nivel profesional, nos sirve para ganar experiencia laboral y enfrentarnos a situaciones reales de cara a una clientela exigente.

\section{Descripción del sistema actual}
Inicialmente, nuestra clienta contaba con una aplicación que mostraba en una página la información a cerca de un tema y los alumnos no se centraban en aprender, sino que iban directamente a hacer el cuestionario final con el objetivo de terminar antes. Esto hace que los alumnos no aprendan como es debido ni fomenten su imaginación ni su creatividad.

\section{Objetivos y alcance del proyecto}
\subsection{Objetivos}
Motivación de la creatividad y fomento de la imaginación en niños.

Para cumplir con el objetivo general, tendremos que cubrir los siguientes puntos:
\begin{itemize}
	\item Recursos de aprendizaje interactivos.
	\item Es evaluable por un profesor.
	\item Se pueden compartir historias entre usuarios.
	\item Es simple y manejable por alumnos de primaria.
	\item Fomenta las habilidades y enseñanzas STEM (science, technology, engeneering and maths).
\end{itemize}

\subsection{Alcance}
Los alumnos podrán crear fantasías, compartirlas y podrán ser evaluadas por los profesores, que podrán mandar como tarea el hacer fantasías.

\section{Organización del documento}
Nada por ahora.


\section{Motivaci�n\label{sec:Motivacion}}
\input{Introduccion/Motivacion.tex}

\section{Objetivos\label{sec:Objetivos}}
\input{Introduccion/Objetivos.tex}

\section{Alcance\label{sec:Alcance}}
\input{Introduccion/Alcance.tex}

\section{Lenguaje WS-BPEL\label{sec:WSBPEL}}
\input{Introduccion/ws-bpel.tex}

\subsection{Actividades del lenguaje WS-BPEL\label{sec:Actividades}}
\input{Introduccion/bpelActivities.tex}

\subsection{Desaf�os en la prueba de composiciones BPEL\label{sec:DesafiosBPEL}}
\input{Introduccion/PruebaBPEL.tex}

\subsection{Casos de prueba en composiciones WS-BPEL\label{sec:CasosPruebaBPEL}}
\input{Introduccion/CasosPrueba.tex}

\section{Composici�n del tri�ngulo\label{sec:ComposicionTriangulo}}
\input{Introduccion/Triangulo.tex}

\section{JSON, JavaScript Object Notation\label{sec:JSON}}
\input{Introduccion/FormatoJSON.tex}



\section{Glosario\label{sec:Glosario}}
\input{Introduccion/Glosario.tex}

\subsection{Acr�nimos\label{sec:Acronimos}}
\input{Introduccion/Acronimos.tex}

\subsection{Definiciones\label{sec:Definiciones}}
\input{Introduccion/Definiciones.tex}

%----------------------------
\chapter{T�cnica de prueba metam�rfica\label{sec:TecnicaMetamorfica}}
\input{Metamorfica/Inicio.tex}

\section{Definici�n\label{sec:DefinicionMetamorfica}}
\input{Metamorfica/Definicion.tex}

\section{Aplicaci�n a una composici�n\label{sec:AplicacionComposicion}}
\input{Metamorfica/AplicacionComposicion.tex}

%----------------------------

%\chapter{Estado del arte\label{sec:EstadoDelArte}}
%\input{EstadoDelArte/EstadoDelArte.tex}

%----------------------------

\chapter{Desarrollo del calendario\label{cap:Calendario}}
\input{Calendario/DesarrolloCalendario.tex}

%\section{Art�culo JCIS 2016\label{sec:JCIS16}}
%\input{Calendario/JCIS16.tex}

%\section{Presentaci�n SGSOACS 2016\label{sec:SGSOACS16}}
%\input{Calendario/SGSOACS16.tex}

\section{Fases\label{sec:Fases}}
\input{Calendario/Fases.tex}

\section{Diagrama de Gantt\label{sec:Gantt}}
\input{Calendario/Gantt.tex}


%----------------------------

\chapter{Descripci�n general del proyecto\label{cap:DescripcionGeneral}}
%\input{Descripcion/DescripcionGeneral.tex}

\section{Perspectiva del producto\label{sec:Perspectiva}}
\input{Descripcion/Perspectiva.tex}

\subsection{Entorno del producto\label{sec:Entorno}}
\input{Descripcion/Entorno.tex}

\subsection{Interfaz de usuario\label{sec:Interfaz}}
\input{Descripcion/Interfaz.tex}

\section{Funciones\label{sec:Funciones}}
\input{Descripcion/Funciones.tex}

\section{Caracter�sticas del usuario\label{sec:CaracteristicasUsuario}}
\input{Descripcion/CaracteristicasUsuario.tex}

\section{Restricciones generales\label{sec:Restricciones}}
\input{Descripcion/Restricciones.tex}

\subsection{Control de versiones\label{sec:ControlVersiones}}
\input{Descripcion/ControlVersiones.tex}

\subsection{Lenguajes de programaci�n y tecnolog�as\label{sec:Tecnologias}}
\input{Descripcion/Tecnologias.tex}

\subsection{Herramientas\label{sec:Herramientas}}
\input{Descripcion/Herramientas.tex}

\subsection{Sistemas operativos y hardware\label{sec:SistemasOperativos}}
\input{Descripcion/SistemasOperativos.tex}



%----------------------------

\chapter{Desarrollo del proyecto\label{cap:DesarrolloProyecto}}
\input{Desarrollo/DesarrolloProyecto.tex}

\section{Modelo de ciclo de vida\label{sec:ModeloCiclo}}
\input{Desarrollo/ModeloCiclo.tex}

\section{Requisitos\label{sec:Requisitos}}
\input{Desarrollo/Requisitos.tex}

\subsection{Funcionales\label{sec:Funcionales}}
\input{Desarrollo/Funcionales.tex}

\subsection{De informaci�n\label{sec:DeInformacion}}
\input{Desarrollo/DeInformacion.tex}

\subsection{Reglas de negocio\label{sec:ReglasNegocio}}
\input{Desarrollo/ReglasNegocio.tex}

\subsection{Interfaz\label{sec:Interfaz}}
\input{Desarrollo/Interfaz.tex}

\subsection{No funcionales\label{sec:NoFuncionales}}
\input{Desarrollo/NoFuncionales.tex}

\section{An�lisis del sistema\label{sec:Analisis}}
\input{Desarrollo/Analisis.tex}

\subsection{Casos de uso\label{sec:CasoUso}}
\input{Desarrollo/CasosDeUso.tex}

%\subsection{Modelo conceptual de datos del dominio\label{sec:ModeloConceptual}}
%\input{Desarrollo/ModeloConceptual.tex}

\subsection{Diagramas de secuencia\label{sec:DiagramaSecuencia}}
\input{Desarrollo/DiagramasDeSecuencia.tex}

\section{Dise�o del sistema\label{sec:DisenoSistema}}
\input{Desarrollo/DisenoSistema.tex}

\section{Implementaci�n\label{sec:Implementacion}}
\chapter{Implementación del sistema}
\section{Entorno tecnológico}
El entorno tecnológico de programación usado en el proyecto Fantasy ha sido el framework \href{https://laravel.com/}{Laravel}, el cual tiene integración con PHP y MySQL.

Gracias a PHP hemos conseguido que la página web del proyecto Fantasy sea lo más parecida a la de STIMEY, para su posterior integración en la plataforma.

Y, con MySQL, gestionamos la base de datos de las fantasías que realizan los alumnos y profesores, teniendo un control de la propiedad y acceso a las fantasías.

\section{Código fuente}
\textcolor{red}{SEGURO QUE QUEREMOS PONER ALGO AQUI?}

\section{Calidad de código}
\textcolor{red}{SEGURO QUE QUEREMOS PONER ALGO AQUI?}

\section{Pruebas y validaci�n\label{sec:PruebasValidacion}}
\input{Desarrollo/PruebasValidacion.tex}

%----------------------------

\chapter{An�lisis y Obtenci�n de Informaci�n\label{sec:AnalisisTodo}}
\chapter{Introducción}
\section{Motivación}
Es un trabajo de la asignatura ``Proyectos Informáticos'' que, a nivel profesional, nos sirve para ganar experiencia laboral y enfrentarnos a situaciones reales de cara a una clientela exigente.

\section{Descripción del sistema actual}
Inicialmente, nuestra clienta contaba con una aplicación que mostraba en una página la información a cerca de un tema y los alumnos no se centraban en aprender, sino que iban directamente a hacer el cuestionario final con el objetivo de terminar antes. Esto hace que los alumnos no aprendan como es debido ni fomenten su imaginación ni su creatividad.

\section{Objetivos y alcance del proyecto}
\subsection{Objetivos}
Motivación de la creatividad y fomento de la imaginación en niños.

Para cumplir con el objetivo general, tendremos que cubrir los siguientes puntos:
\begin{itemize}
	\item Recursos de aprendizaje interactivos.
	\item Es evaluable por un profesor.
	\item Se pueden compartir historias entre usuarios.
	\item Es simple y manejable por alumnos de primaria.
	\item Fomenta las habilidades y enseñanzas STEM (science, technology, engeneering and maths).
\end{itemize}

\subsection{Alcance}
Los alumnos podrán crear fantasías, compartirlas y podrán ser evaluadas por los profesores, que podrán mandar como tarea el hacer fantasías.

\section{Organización del documento}
Nada por ahora.


\section{Proceso de An�lisis\label{sec:ProcesoAnalisis}}
\input{Resultados/ProcesoAnalisis.tex}

\section{Resultado del An�lisis\label{sec:ResultadoAnalisis}}
\input{Resultados/ResultadoAnalisis.tex}

\subsection{Ficheros JSON\label{sec:FicherosJSON}}
\input{Resultados/JSON.tex}

\subsection{Lenguaje Natural\label{sec:LenguajeNatural}}
\input{Resultados/LenguajeNatural.tex}

%----------------------------


%\chapter{Resumen\label{cap:Resumen}}
%\newpage{\pagestyle{empty}\cleardoublepage} 
\vspace*{\fill}
\begin{center}
	\textbf{Resumen}
\end{center}
Aplicación web para fomentar el aprendizaje mediante la imaginación y creatividad de niños entre 10 y 13 años en temas científicos-tecnológicos en colaboración con el proyecto europeo STIMEY.

A modo de juego, los niños podrán crear historias interactivas y los profesores podrán evaluarlos.\\

\textbf{Palabras clave:}
Fantasía, aprendizaje, desarrollo, ilusiona, entretenimiento, creatividad, cuestionario, evalua-
ción, enseñanza, ciencia, unión europea.
\vspace*{\fill}

\newpage


%----------------------------

\chapter{Conclusiones y Trabajo Futuro\label{cap:ConclusionesFuturo}}
\input{Conclusiones/ConclusionesFuturo.tex}

\section{Valoraci�n\label{sec:Valoracion}}
\input{Conclusiones/Valoracion.tex}

\section{Objetivos cumplidos\label{sec:ObjetivosCumplidos}}
\input{Conclusiones/ObjetivosCumplidos.tex}

\section{Lecciones aprendidas\label{sec:LeccionesAprendidas}}
\input{Conclusiones/LeccionesAprendidas.tex}

\section{Trabajo futuro\label{sec:Futuro}}
\input{Conclusiones/Futuro.tex}

%----------------------------


\appendix

\chapter{Composiciones Estudiadas\label{cap:ComposicionesEstudiadas}}
\input{Anexos/ComposicionesUtilizadas.tex}

\chapter{Manual de instalaci�n\label{cap:ManualInstalacion}}
\input{Anexos/ManualInstalacion.tex}

\chapter{Manual de usuario\label{cap:ManualUsuario}}
\chapter{Manual de usuario}
\section{Introducción}

Aquí se narrará el manual que han de seguir los usuarios del sistema para usar la aplicación\footnote{Para mayor detalle sobre el manual de usuario, consultar el apéndice.}.

Teniendo en cuenta que será una aplicación web, el primer paso será dirigirse a la dirección web donde esté ubicado el servicio\footnote{Si ha sido integrado en la plataforma de STIMEY, estará en la zona de laboratorios.}.


\section{Características}

La aplicación Fantasy permite al usuario crear fantasías para que los estudiantes aprendan de una forma más creativa y divertida viendo los puntos activos y desarrollando los \textit{quizzes} asociados a cada punto activo, y, finalmente, el \textit{quiz} final asociado a cada fantasía.


Esta puntuación será enviada a la plataforma de STIMEY para ser almacenada en el perfil de cada alumno.


Los alumnos también podrán realizar fantasías que sean ordenadas por sus profesores como tarea. Esta tarea, podrá ser en pareja o individual, y será posteriormente evaluada por el profesor que puso la tarea.


\section{Requisitos previos}

Los requisitos previos a la hora de usar la aplicación del proyecto Fantasy es entrar en la dirección web donde se encuentre soportado el servicio y registrarse con la cuenta del usuario que vaya a usar la aplicación.


Por lo tanto, no es necesario tener nada instalado en el ordenador del usuario debido a que la aplicación se encuentra alojada en un servidor.


\section{Utilización}
A la hora de utilizar la aplicación y, habiendo accedido con nuestro usuario y contraseña a la plataforma, tendremos una pantalla principal donde podremos ver nuestras fantasías (las que hayamos creado nosotros) y las fantasías que hemos marcado como favoritas para volver a jugarlas.

\subsection{Crear fantasía}
Si queremos crear una fantasía, hacemos click sobre el icono de crear una nueva fantasía.

A continuación, rellenamos los campos necesarios y ya estaría creada la fantasía.

Una vez hayamos terminado con la creación de la fantasía, podremos volver atrás y verla en la parte reservada a nuestras fantasías.

\subsection{Modificar fantasía}
Si queremos modificar una fantasía, deberemos hacer click izquierdo sobre el icono de editar fantasía.

A continuación, modificaremos los campos que deseemos cambiar.

\subsection{Eliminar fantasía}
Si queremos eliminar una fantasía, haremos click izquierdo sobre el icono de borrar fantasía.

\subsection{Duplicar fantasía}

%Se supone que esto debe estar hecho

\subsection{Crear punto activo}
A la hora de añadir los puntos activos a la fantasía. Para ello hacemos click izquierdo en el icono de añadir un nuevo punto activo.

A continuación, rellenamos los campos necesarios y ya tendríamos creado el primer punto activo.

Podremos crear hasta un máximo de diez puntos activos siguiendo los pasos anteriormente mencionados para cada punto activo.


\subsection{Modificar punto activo}
En los puntos activos tendremos la posibilidad de mover y redimensionar el punto activo a nuestro gusto de forma que se quede donde y como nosotros deseemos.

Además, haciendo doble click izquierdo sobre ellos, tendremos la posibilidad de cambiar el contenido de sus campos.

\subsection{Eliminar punto activo}
%se supone que esto debe estar hecho

\subsection{Jugar fantasía}
Cuando queramos jugar una fantasía (bien sea creada por nosotros o por otro usuario), tendremos que hacer click izquierdo en el icono de jugar la fantasía y se podrá jugar\footnote{Tenemos que tener en cuenta que para los alumnos, solo guardaremos la primera nota que saquen, aunque puedan repetir al fantasía tantas veces como quieran.}.


\chapter{Manual del desarrollador\label{cap:ManualDesarrollador}}
\input{Anexos/ManualDesarrollador.tex}

\chapter{GNU Free Documentation License\label{cap:LicenciaGNU}}
\input{Anexos/GNULicense.tex}
%----------------------------

\bibliographystyle{splncs03}
\bibliography{referencias}

\end{document}
