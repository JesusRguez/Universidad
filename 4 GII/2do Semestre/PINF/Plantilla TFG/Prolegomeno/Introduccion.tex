\chapter{Introducción}
\section{Motivación}
Es un trabajo de la asignatura ``Proyectos Informáticos'' que, a nivel profesional, nos sirve para ganar experiencia laboral y enfrentarnos a situaciones reales de cara a una clientela exigente.

\section{Descripción del sistema actual}
Inicialmente, nuestra clienta contaba con una aplicación que mostraba en una página la información a cerca de un tema y los alumnos no se centraban en aprender, sino que iban directamente a hacer el cuestionario final con el objetivo de terminar antes. Esto hace que los alumnos no aprendan como es debido ni fomenten su imaginación ni su creatividad.

\section{Objetivos y alcance del proyecto}
\subsection{Objetivos}
Motivación de la creatividad y fomento de la imaginación en niños.

Para cumplir con el objetivo general, tendremos que cubrir los siguientes puntos:
\begin{itemize}
	\item Recursos de aprendizaje interactivos.
	\item Es evaluable por un profesor.
	\item Se pueden compartir historias entre usuarios.
	\item Es simple y manejable por alumnos de primaria.
	\item Fomenta las habilidades y enseñanzas STEM (science, technology, engeneering and maths).
\end{itemize}

\subsection{Alcance}
Los alumnos podrán crear fantasías, compartirlas y podrán ser evaluadas por los profesores, que podrán mandar como tarea el hacer fantasías.

\section{Organización del documento}
Nada por ahora.
