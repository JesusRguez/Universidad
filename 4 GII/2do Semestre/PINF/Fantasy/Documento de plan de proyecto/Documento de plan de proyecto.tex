%%\documentclass[a4paper,12pt,oneside]{llncs}
\documentclass[12pt,letterpaper]{article}
\usepackage[right=2cm,left=3cm,top=2cm,bottom=2cm,headsep=0cm]{geometry}

%%%%%%%%%%%%%%%%%%%%%%%%%%%%%%%%%%%%%%%%%%%%%%%%%%%%%%%%%%%
%% Juego de caracteres usado en el archivo fuente: UTF-8
\usepackage{ucs}
\usepackage[utf8x]{inputenc}

%%%%%%%%%%%%%%%%%%%%%%%%%%%%%%%%%%%%%%%%%%%%%%%%%%%%%%%%%%%
%% Juego de caracteres usado en la salida dvi
%% Otra posibilidad: \usepackage{t1enc}
\usepackage[T1]{fontenc}

%%%%%%%%%%%%%%%%%%%%%%%%%%%%%%%%%%%%%%%%%%%%%%%%%%%%%%%%%%%
%% Ajusta maergenes para a4
%\usepackage{a4wide}

%%%%%%%%%%%%%%%%%%%%%%%%%%%%%%%%%%%%%%%%%%%%%%%%%%%%%%%%%%%
%% Uso fuente postscript times, para que los ps y pdf queden y pequeños...
\usepackage{times}

%%%%%%%%%%%%%%%%%%%%%%%%%%%%%%%%%%%%%%%%%%%%%%%%%%%%%%%%%%%
%% Posibilidad de hipertexto (especialmente en pdf)
%\usepackage{hyperref}
\usepackage[bookmarks = true, colorlinks=true, linkcolor = black, citecolor = black, menucolor = black, urlcolor = black]{hyperref}

%%%%%%%%%%%%%%%%%%%%%%%%%%%%%%%%%%%%%%%%%%%%%%%%%%%%%%%%%%%
%% Graficos 
\usepackage{graphics,graphicx}

%%%%%%%%%%%%%%%%%%%%%%%%%%%%%%%%%%%%%%%%%%%%%%%%%%%%%%%%%%%
%% Ciertos caracteres "raros"...
\usepackage{latexsym}

%%%%%%%%%%%%%%%%%%%%%%%%%%%%%%%%%%%%%%%%%%%%%%%%%%%%%%%%%%%
%% Matematicas aun más fuertes (american math dociety)
\usepackage{amsmath}

%%%%%%%%%%%%%%%%%%%%%%%%%%%%%%%%%%%%%%%%%%%%%%%%%%%%%%%%%%%
\usepackage{multirow} % para las tablas
\usepackage[spanish,es-tabla]{babel}

%%%%%%%%%%%%%%%%%%%%%%%%%%%%%%%%%%%%%%%%%%%%%%%%%%%%%%%%%%%
%% Fuentes matematicas lo mas compatibles posibles con postscript (times)
%% (Esto no funciona para todos los simbolos pero reduce mucho el tamaño del
%% pdf si hay muchas matamaticas....
\usepackage{mathptm}

%%% VARIOS:
%\usepackage{slashbox}
\usepackage{verbatim}
\usepackage{array}
\usepackage{listings}
\usepackage{multirow}

%% MARCA DE AGUA
%% Este package de "draft copy" NO funciona con pdflatex
%%\usepackage{draftcopy}
%% Este package de "draft copy" SI funciona con pdflatex
%%%\usepackage{pdfdraftcopy}
%%%%%%%%%%%%%%%%%%%%%%%%%%%%%%%%%%%%%%%%%%%%%%%%%%%%%%%%%%%
%% Indenteacion en español...
\usepackage[spanish]{babel}

\usepackage{listings}
% Para escribir código en C
% \begin{lstlisting}[language=C]
% #include <stdio.h>
% int main(int argc, char* argv[]) {
% puts("Hola mundo!");
% }
% \end{lstlisting}


\title{Documento de plan de proyecto}
\author{Luis Gutiérrez Flores\\
	Nicolás Ruiz Requejo\\
	Jesús Rodríguez Heras\\
	Arantzazu Otal Alberro\\
	Alejandro Segovia Gallardo\\
	Alejandro José Caraballo García\\
	Gabriel Fernando Sánchez Reina}

\begin{document}
	
	\maketitle
%	\begin{abstract} %Poner esto en todas las prácticas de PCTR
%%		\begin{center}
%%			\noindent
%%		\end{center}
%	\end{abstract}

	\thispagestyle{empty}
	\newpage
	
	\tableofcontents
	\newpage
	
	%%\listoftables
	%%\newpage
	
	%%\listoffigures
	%%\newpage
	
	%%%% REAL WORK BEGINS HERE:
	
	%%Configuracion del paquete listings
	\lstset{language=bash, numbers=left, numberstyle=\tiny, numbersep=10pt, firstnumber=1, stepnumber=1, basicstyle=\small\ttfamily, tabsize=1, extendedchars=true, inputencoding=latin1}

\section{Visión general}
Este documento recoge la planificación, el planteamiento y el principio de un proyecto al que hemos denominado ``\textbf{Fantasy}'', un portal web en el que los profesores pueden realizar una serie de tareas (fantasías) con el objetivo de que los alumnos puedan jugarlas y así aprendan de forma creativa.

Los alumnos, también tendrán la posibilidad de crear las fantasías que el profesor les mande como trabajo y luego serán evaluadas por dicho profesor.

\section{Roles}
\begin{enumerate}
	\item \textbf{Scrum master:} Luis Gutiérrez Flores.
	\item \textbf{Administrador de sistemas:} Alejandro José Caraballo García.
	\item \textbf{Producto owner:} Jesús Rodríguez Heras.
	\item \textbf{Analista:} Nicolás Ruiz Requejo.
	\item \textbf{Arquitecto Software:} Arantzazu Otal Alberro.
	\item \textbf{Desarrollador:} Todos.
	\item \textbf{Diseñador de interfaz de usuario:} Alejandro Segovia Gallardo.
	\item \textbf{Tester:} Gabriel Fernando Sánchez Reina.
\end{enumerate}

\section{Política de equipo}
\subsection{Política de reuniones}
El equipo ha decidido hacer reuniones con una periodicidad de una semana con el cliente, a lo largo de la semana, los integrantes del equipo intentarán establecer reuniones entre ellos con la duración necesaria para continuar avanzando en el proyecto (tiempo estimado: dos horas).

\subsection{Concepto de tarea realizada}
Cuando está implementada una funcionalidad y testeada tanto por el grupo de programadores como por el tester. Se da el visto bueno y se le encarga al propietario (producto owner) que marque como realizada dicha tarea.

\section{Hitos} %Seguir poniendo los demás sprints
\subsection{Sprint 0 (27 de febrero al 6 de marzo)}
\begin{enumerate}
	\item Creación de plataformas de trabajo y control de versiones (GitHub).
	\item Creación del boceto de requisitos.
	\item Creación de casos de uso y sus descripciones.
\end{enumerate}
\subsection{Sprint 1 (6 de marzo al 27 de marzo)}
\begin{enumerate}
	\item Creación de mockups de la plataforma.
	\item Implementación de la base de datos.
\end{enumerate}

\subsection{Sprint 2 (27 de marzo al 3 de abril)}
\begin{enumerate}
	\item Implementación de front-end.
	\item Migraciones de la base de datos.
	\item Adaptación del proyecto al framework Laravel.
	\item Creación de fantasías.
\end{enumerate}

\subsection{Sprint 3 (3 de abril al 10 de abril)}
\begin{enumerate}
	\item Finalización de front-end.
	\item Migraciones finales de la base de datos.
	\item Finalización de la creación de fantasías.
\end{enumerate}

\subsection{Sprint 4 (10 de abril al 24 de abril)}
\begin{enumerate}
	\item Finalización de las migraciones de la base de datos.
	\item Creación de puntos activos con sus características básicas.
\end{enumerate}

\subsection{Sprint 5 (24 de abril al 1 de mayo)}
\begin{enumerate}
	\item Creación de puntos activos con todas sus caracterísitcas.
\end{enumerate}


\section{Reuniones}
\subsection{Reunión 0 (27 de febrero)}
\begin{enumerate}
	\item Análisis de requisitos del sistema.
	\item Distribución de las tareas del sprint 0.
\end{enumerate}

\subsection{Reunión 1 (4 de marzo)}
\begin{enumerate}
	\item Corrección de casos de uso y creación de las descripciones de todos los casos de uso.
	\item Planteamiento de la base de datos del sistema.
\end{enumerate}

\subsection{Reunión 2 (6 de marzo)}
\begin{enumerate}
	\item Finalización del sprint 0.
	\item Distribución de la información a buscar.
	\item Comienzo del sprint 1.
\end{enumerate}

\subsection{Reunión 3 (27 de marzo)}
\begin{enumerate}
	\item Finalización del sprint 1.
	\item Comienzo del sprint 2.
\end{enumerate}

\subsection{Reunión 4 (3 de abril)}
\begin{enumerate}
	\item Finalización del sprint 2.
	\item Comienzo del sprint 3.
\end{enumerate}

\subsection{Reunión 5 (10 de abril)}
\begin{enumerate}
	\item Finalización del sprint 3.
	\item Comienzo del sprint 4.
\end{enumerate}

\subsection{Reunión 6 (24 de abril)}
\begin{enumerate}
	\item Finalización del sprint 4.
	\item Comienzo del sprint 5.
\end{enumerate}

\end{document}