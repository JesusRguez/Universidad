\chapter{Use cases}
\noindent
All the use cases described below have the following implicit precondition to be able to use said use cases in the final application:
\begin{itemize}
	\item The user (teachers/students) must have an account on the Stimey platform and have logged in with that account.
\end{itemize}

\section{CRUD fantasy}
\hypertarget{crearfantasia}{}
\subsection{Create fantasy}
\begin{itemize}
	\item \textbf{Description:} Create a new fantasy.
	\item \textbf{Actors:} Creator-editor (user).
	\item \textbf{Preconditions:} The user must have permission to create a new fantasy.
	\item \textbf{Postconditions:} Fantasy is stored in the system.
	\item \textbf{Main stage:}
	\begin{enumerate}
		\item The user selects the option ``Create new fantasy''.
		\item The system requests the name of the fantasy.
		\item The user enters the name of the fantasy.
		\item The system requests the fantasy code.
		\item The user enters the fantasy code.
		\item The system gives to choose if the fantasy will be public (by default), shared or private.
		\item The user selects `` Public ''.
		\item The user creates the fantasy.
		\item The fantasy is stored in the system.
	\end{enumerate}
	\item \textbf{Extensions:} \\7. a) The user selects that the fantasy will be shared.
	\begin{enumerate}
		\item The system allows to insert in a list the identifiers of other users with which the fantasy will be shared.
		\item The user enters the identifiers of the users who will share the fantasy.
		\item Step 8.
	\end{enumerate}
	7. b) The user selects that the fantasy will be private.
	\begin{enumerate}
		\item The system marks the fantasy as private for that user without giving the possibility of sharing.
		\item Step 8.
	\end{enumerate}
	*a) At any time, the user can go back to the main menu.
	\item \textbf{Variations:} None.
	\item \textbf{Not-functional:} None.
	\item \textbf{Issues:} None.
\end{itemize}

\subsection{Visualize fantasy}
\begin{itemize}
	\item \textbf{Description:} Read an existing fantasy.
	\item \textbf{Actors:} Creator-editor (user).
	\item \textbf{Preconditions:} The fantasy must exist in the system and the user must have modification permissions.
	\item \textbf{Postconditions:} There are no changes in the fantasy.
	\item \textbf{Main stage:}
	\begin{enumerate}
		\item The user selects the option `` My fantasies ''
		\item The system displays a list of fantasies accessible by the user.
		\item The user selects the fantasy that he wants to visualize.
		\item The system displays a pop-up window with the fantasy information and its options.
		\item The user selects the option `` Visualize fantasy ''.
		\item The system shows the fantasy.
		\item The user reads the fantasy without making any changes and, when it is over, closes the fantasy.
		\item The fantasy remains unchanged.
	\end{enumerate}
	\item \textbf{Extensions:} \\ *a) At any time, the user can go back to the main menu.
	\item \textbf{Variations:} None.
	\item \textbf{Not-functional:} None.
	\item \textbf{Issues:} None.
\end{itemize}

\subsection{Update fantasy}
\begin{itemize}
	\item \textbf{Description:} Modify an existing fantasy.
	\item \textbf{Actors:} Creator-editor (user).
	\item \textbf{Preconditions:} The fantasy must exist in the system and the user must have modification permissions.
	\item \textbf{Postconditions:} The fantasy is modified.
	\item \textbf{Main stage:}
	\begin{enumerate}
		\item The user selects the option ``My fantasies''.
		\item The system displays a list of fantasies accessible by the user.
		\item The user selects the fantasy that he wants to modify.
		\item The system displays a pop-up window with the fantasy information and its options.
		\item The user selects the option ``Modify fantasy''.
		\item The system displays the fantasy creation screen for modification.
	\end{enumerate}
	\item \textbf{Extensions:} \\ *a) At any time, the user can go back to the main menu.
	\item \textbf{Variations:} None.
	\item \textbf{Not-functional:} None.
	\item \textbf{Issues:} None.
\end{itemize}

\subsection{Delete fantasía}
\begin{itemize}
	\item \textbf{Description:} Erase an existing fantasy.
	\item \textbf{Actors:} Creator-editor (user).
	\item \textbf{Preconditions:} The fantasy must exist in the system and the user must have removal permissions.
	\item \textbf{Postconditions:} Fantasy is eliminated from the system.
	\item \textbf{Main stage:}
	\begin{enumerate}
		\item The user selects the option ``My fantasies''.
		\item The system displays a list of fantasies accessible by the user.
		\item The user selects the fantasy that he wants to modify.
		\item The system displays a pop-up window with the fantasy information and its options.
		\item The user selects the option ``Clear fantasy''.
		\item The system displays a confirmation message.
		\item The user selects ``Accept''.
		\item The system erases the fantasy.
	\end{enumerate}
	\item \textbf{Extensions:}  \\7. a) The user selects ``Cancel''.
	\begin{enumerate}
		\item The system closes the pop-up window.
		\item Step 1.
	\end{enumerate}
	*a) At any time, the user can go back to the main menu.
	\item \textbf{Variations:} None.
	\item \textbf{Not-functional:} None.
	\item \textbf{Issues:} None.
\end{itemize}


\section{Choose language}
\begin{itemize}
	\item \textbf{Description:} Change the language of the application.
	\item \textbf{Actors:} Teacher or student (user).
	\item \textbf{Preconditions:} None. %el usuario es capaz de encontrar el menú de idioma con la mirada
	\item \textbf{Postconditions:} The application changes to the language selected by the user.
	\item \textbf{Main stage:}
	\begin{enumerate}
		\item The user presses the language change button.
		\item The system displays a list of available languages.
		\item The user selects a language from those that are available in the system.
		\item The application changes the language.
	\end{enumerate}
	\item \textbf{Extensions:} None.
	\item \textbf{Variations:} None.
	\item \textbf{Not-functional:} None.
	\item \textbf{Issues:} None.
\end{itemize}

\section{Copy fantasy}
\begin{itemize}
	\item \textbf{Description:} Clone a fantasy.
	\item \textbf{Actores:} Creator-editor (user).
	\item \textbf{Precondiciones:} The fantasy must exist in the system and the user must have modification permissions.
	\item \textbf{Postcondiciones:} Create a copy of the selected fantasy.
	\item \textbf{Main stage:}
	\begin{enumerate}
		\item The user selects the option ``My fantasies''.
		\item The system displays a list of fantasies accessible by the user.
		\item The user selects the fantasy that he wants to copy.
		\item The system displays a pop-up window with the fantasy information and its options.
		\item The user selects the option ``Copy fantasy''.
		\item The system creates a copy of the selected fantasy.
	\end{enumerate}
	\item \textbf{Extensions:} \\ *a) At any time, the user can go back to the main menu.
	\item \textbf{Variations:} None.
	\item \textbf{Not-functional:} None.
	\item \textbf{Issues:} None.
\end{itemize}

\section{CRUD background}
\begin{itemize}
	\item \textbf{Description:} Allows selecting, modifying and deleting the background.
	\item \textbf{Actors:} Creator-editor (user).
	\item \textbf{Preconditions:} The fantasy must exist in the system and the user must have modification permissions.
	\item \textbf{Postconditions:} The fund that the user has chosen is established.
	\item \textbf{Main stage:}
	\begin{enumerate}
		\item The user selects the ``Background'' option.
		\item The system displays a window to add a background to the workspace.
		\item The user selects an image.
		\item The system sets the background selected by the user.
	\end{enumerate}
	\item \textbf{Extensions:} \\ *a) At any time, the user can go back to the main menu. 
	\item \textbf{Variations:} None.
	\item \textbf{Not-functional:} None.
	\item \textbf{Issues:} None.
\end{itemize}


\section{CRUD active point}
\subsection{Create active point}
\begin{itemize}
	\item \textbf{Description:} Create a new active point.
	\item \textbf{Actors:} Creator-editor (user).
	\item \textbf{Preconditions:} The fantasy must exist in the system and the user must have modification permissions.
	\item \textbf{Postconditions:} An empty active point is created in the workspace.
	\item \textbf{Main stage:}
	\begin{enumerate}
		\item The user selects the option ``New active point''.
		\item The system creates a new active point in the workspace.
		\item The user can move the active point to the area of the workspace that he wants.
		\item The system will save the active point in the fantasy.
	\end{enumerate}
	\item \textbf{Extensiones:} \\ *a) At any time, the user can go back.
	\item \textbf{Variations:} None.
	\item \textbf{Not-functional:} None.
	\item \textbf{Issues:} None.
\end{itemize}

\subsection{Visualize active point}
\begin{itemize}
	\item \textbf{Description:} Shows an existing active point for reading.
	\item \textbf{Actors:} Creator-editor (user).
	\item \textbf{Preconditions:} Fantasy must exist in the system and the active point must exist in fantasy. In addition, the user must have modification permissions.
	\item \textbf{Postconditions:} The active point is shown for reading.
	\item \textbf{Main stage:}
	\begin{enumerate}
		\item The user selects the active point that he wants to view.
		\item The system displays a window with the information of the active point and its options.
		\item The user selects the option ``Visualize''.
		\item The system displays a window with the summary of that active point.
	\end{enumerate}
	\item \textbf{Extensions:} \\ *a) At any time, the user can go back.
	\item \textbf{Variations:} None.
	\item \textbf{Not-functional:} None.
	\item \textbf{Issues:} None.
\end{itemize}

\subsection{Update active point}
\begin{itemize}
	\item \textbf{Description:} Modifies an existing active point.
	\item \textbf{Actors:} Creator-editor (user).
	\item \textbf{Preconditions:} Fantasy must exist in the system and the active point must exist in fantasy. In addition, the user must have modification permissions.
	\item \textbf{Postconditions:} Modifies the selected active point.
	\item \textbf{Main stage:}
	\begin{enumerate}
		\item The user selects the active point that he wants to modify.
		\item The system displays a window with the information of the active point and its options.
		\item The user selects the option ``Modify''.
		\item The system displays the creation window of the active point.
	\end{enumerate}
	\item \textbf{Extensions:} \\ *a) At any time, the user can go back.
	\item \textbf{Variations:} None.
	\item \textbf{Not-functional:} None.
	\item \textbf{Issues:} None.
\end{itemize}

\subsection{Delete active point}
\begin{itemize}
	\item \textbf{Description:} Deletes an existing active point.
	\item \textbf{Actors:} Creator-editor (user).
	\item \textbf{Preconditions:} Fantasy must exist in the system and the active point must exist in fantasy. In addition, the user must have modification permissions.
	\item \textbf{Postconditions:} Deletes the selected active point.
	\item \textbf{Main stage:}
	\begin{enumerate}
		\item The user selects the active point that he wants to delete.
		\item The system displays a window with the information of the active point and its options.
		\item The user selects the option ``Delete''.
		\item The system displays a confirmation message.
		\item The user selects ``Accept''.
		\item The system deletes the active point.
	\end{enumerate}
	\item \textbf{Extensions:} \\ 5. a) The user selects ``Cancel''.
	\begin{enumerate}
		\item The system closes the pop-up window.
		\item Step 1.
	\end{enumerate}
	*a) At any time, the user can go back.
	\item \textbf{Variations:} None.
	\item \textbf{Not-functional:} None.
	\item \textbf{Issues:} None.
\end{itemize}

\section{CRUD image}
\hypertarget{crearimagen}{}
\subsection{Create image}
\begin{itemize}
	\item \textbf{Description:} Inserta una imagen en un punto activo.
	\item \textbf{Actors:} Creator-editor (user).
	\item \textbf{Preconditions:} The corresponding active point must exist and the fantasy must be being edited.
	\item \textbf{Postconditions:} Insert an image in the selected active point.
	\item \textbf{Main stage:}
	\begin{enumerate}
		\item The user selects the corresponding active point within the fantasy.
		\item The system displays a pop-up window with the information of the active point.
		\item The user selects the option ``Insert image''.
		\item The system shows a window in which the user chooses where to choose the image (Internet, local, image already used in fantasy).
		\item The user chooses the option ``Internet'' to include an image of the Internet.
		\item The system asks the user for the url of the image.
		\item The user inserts the correct url of the image.
		\item The active point takes the shape of the image.
	\end{enumerate}
	\item \textbf{Extensions:} \\5. a) The user chooses the ``Local'' option to include an image from his computer.
	\begin{enumerate}
		\item The system opens a file browser window.
		\item The user selects the desired image and press ``Accept''.
		\item The system closes the file browser window.
		\item Step 8.
	\end{enumerate}
	5. b) The user chooses the option ``Image previously used'' to include an image already used.
	\begin{enumerate}
		\item The system opens a window with the images previously used.
		\item The user selects the desired image and press ``Accept''.
		\item The system closes the pop-up window.
		\item Step 8.
	\end{enumerate}
	7. a) The url is not correct.
	\begin{enumerate}
		\item The system displays an error message.
		\item Step 6.
	\end{enumerate}
	*a) At any time, the user can go back.
	\item \textbf{Variations:} None.
	\item \textbf{Not-functional:} None.
	\item \textbf{Issues:} None. %Ya dijimos que si podía, con el pixlr
\end{itemize}

\subsection{Update image}
\begin{itemize}
	\item \textbf{Description:} Update an image.
	\item \textbf{Actors:} Creator-editor (user).
	\item \textbf{Preconditions:} There must be the corresponding active point, you must be editing the fantasy and there must be an image.
	\item \textbf{Postconditions:} The image is modified.
	\item \textbf{Main stage:}
	\begin{enumerate}
		\item The user selects the corresponding active point within the fantasy.
		\item The system opens a popup window with the information of the active point.
		\item Step 4 of \hyperlink{crearimagen}{Create image}.
	\end{enumerate}
	\item \textbf{Extensiones:} \\ *a) At any time, the user can go back.
	\item \textbf{Variations:} None.
	\item \textbf{Not-functional:} None.
	\item \textbf{Issues:} None.
\end{itemize}

\subsection{Delete image}
\begin{itemize}
	\item \textbf{Description:} Deletes an image of an active point.
	\item \textbf{Actors:} Creator-editor (user).
	\item \textbf{Preconditions:} There must be the corresponding active point, you must be editing the fantasy and there must be an image.
	\item \textbf{Postconditions:} Delete the image and leave the active point in its default state.
	\item \textbf{Main stage:}
	\begin{enumerate}
		\item The user selects the corresponding active point within the fantasy.
		\item The system opens a popup window with the information of the active point.
		\item The user selects the image and presses the ``Delete'' button.
		\item The system displays a confirmation message.
		\item The user selects ``Accept''.
		\item The system deletes the image of the active point.
	\end{enumerate}
	\item \textbf{Extensions:} \\ 5. a) The user selects ``Cancel''.
	\begin{enumerate}
		\item The system closes the pop-up window.
		\item Step 1.
	\end{enumerate}
	*a) At any time, the user can go back.
	\item \textbf{Variations:} None.
	\item \textbf{Not-functional:} None.
	\item \textbf{Issues:} None.
\end{itemize}

\section{CRUD video}
\hypertarget{crearvideo}{}
\subsection{Create video}
\begin{itemize}
	\item \textbf{Description:} Insert a video within an active point.
	\item \textbf{Actors:} Creator-editor (user).
	\item \textbf{Preconditions:} The corresponding active point must exist and the fantasy must be being edited.
	\item \textbf{Postconditions:} Insert a video in the selected active point.
	\item \textbf{Main stage:}
	\begin{enumerate}
		\item The user selects the corresponding active point within the fantasy.
		\item The system displays a pop-up window with the information of the active point.
		\item The user selects the option ``Insert video''.
		\item The system shows a window in which the user chooses where to choose the image (Internet, local, video already used in fantasy).
		\item The user chooses the option ``Internet'' to include an Internet video.
		\item The system asks the user for the url of the video.
		\item The user enters the correct url of the video.
		\item The system saves the video in the active point.
	\end{enumerate}
	\item \textbf{Extensions:} \\ 5. a) The user chooses the ``Local'' option to include a video from his computer.
	\begin{enumerate}
		\item The system shows a file browser sale.
		\item The user selects the desired image and press ``Accept''.
		\item The system closes the file browser window.
		\item Step 8.
	\end{enumerate}
	5. b) The user chooses the option ``Video previously used'' to include a video already used.
	\begin{enumerate}
		\item The system opens a window with the videos previously used.
		\item The user selects the desired video and press ``Accept''.
		\item The system closes the pop-up window.
		\item Step 4.
	\end{enumerate}
	7. a) The url is not correct.
	\begin{enumerate}
		\item The system displays an error message.
		\item Step 6.
	\end{enumerate}
	*a) At any time, the user can go back.
	\item \textbf{Variations:} None.
	\item \textbf{Not-functional:} None.
	\item \textbf{Issues:} None.
\end{itemize}

\subsection{Update video}
\begin{itemize}
	\item \textbf{Description:} Update a video.
	\item \textbf{Actors:} Creator-editor (user).
	\item \textbf{Preconditions:} There must be the corresponding active point, you must be editing the fantasy and there must be a video.
	\item \textbf{Postconditions:} The video is modified.
	\item \textbf{Main stage:}
	\begin{enumerate}
		\item The user selects the corresponding active point within the fantasy.
		\item The system opens a popup window with the information of the active point.
		\item Step 4 of \hyperlink{crearvideo}{Create video}.
	\end{enumerate}
	\item \textbf{Extensions:} \\ *a) At any time, the user can go back.
	\item \textbf{Variations:} None.
	\item \textbf{Not-functional:} None.
	\item \textbf{Issues:} None.
\end{itemize}

\subsection{Delete video}
\begin{itemize}
	\item \textbf{Description:} Delete a video of an active point.
	\item \textbf{Actors:} Creator-editor (user).
	\item \textbf{Preconditions:} There must be the corresponding active point, you must be editing the fantasy and there must be a video.
	\item \textbf{Postconditions:} Delete the video of the active point.
	\item \textbf{Main stage:}
	\begin{enumerate}
		\item The user selects the corresponding active point within the fantasy.
		\item The system opens a popup window with the information of the active point.
		\item The user selects the video and presses the ``Delete'' button.
		\item The system displays a confirmation message.
		\item The user selects ``Accept''.
		\item The system deletes the video from the active point.
	\end{enumerate}
	\item \textbf{Extensions:} \\ 5. a) The user selects ``Cancel''.
	\begin{enumerate}
		\item The system closes the pop-up window.
		\item Step 1.
	\end{enumerate}
	*a) At any time, the user can go back.
	\item \textbf{Variations:} None.
	\item \textbf{Not-functional:} None.
	\item \textbf{Issues:} None.
\end{itemize}

\section{CRUD text}
\begin{itemize}
	\item \textbf{Description:} Insert a text in an active point.
	\item \textbf{Actors:} Creator-editor (user).
	\item \textbf{Preconditions:} The corresponding active point must exist and the fantasy must be being edited.
	\item \textbf{Postconditions:} Insert a text in the selected active point.
	\item \textbf{Main stage:}
	\begin{enumerate}
		\item The user selects the active point to which he wants to add-edit the text.
		\item The system displays a pop-up window with the information of the active point.
		\item The user selects enter the desired text in the ``Text'' field with the formatting options that you want.
		\item The user clicks on the ``Accept'' button.
		\item The system saves the text in the corresponding active point.
	\end{enumerate}
	\item \textbf{Extensions:} \\ *a) At any time, the user can go back.
	\item \textbf{Variations:} None.
	\item \textbf{Not-functional:} None.
	\item \textbf{Issues:} None.
\end{itemize}

\section{CRUD \textit{quiz}}
\hypertarget{crearquiz}{}
\subsection{Create \textit{quiz}}
\begin{itemize}
	\item \textbf{Description:} Create a small questionnaire about the subject that the active point deals with.
	\item \textbf{Actors:} Creator-editor (user).
	\item \textbf{Preconditions:} The corresponding active point must exist and the fantasy must be being edited.
	\item \textbf{Postconditions:} Create a small questionnaire in relation to the corresponding active point.
	\item \textbf{Main stage:}
	\begin{enumerate}
		\item The user selects the corresponding active point.
		\item The system displays a pop-up window with the information of the active point.
		\item The user selects the option ``Create \textit{quiz}''.
		\item The system shows the possible options.
		\item The user selects ``Simple answer''.
		\item The system displays a pop-up window to create the question with its possible answers.
		\item The user populates the pop-up window with the question and the appropriate answers and press ``Accept'' when it finishes.
		\item The system closes the pop-up window.
		\item The questionnaire is registered in the selected active point.
	\end{enumerate}
	\item \textbf{Extensions:} \\3. a) The user chooses the option ``Word''.
	\begin{enumerate}
		\item The system opens a pop-up window to create the question and its answer.
		\item The user populates the pop-up window with the question and the appropriate answer and press ``Accept'' when it finishes.
		\item Step 8.
	\end{enumerate}
	3. b) The user chooses the option ``Quiz with images''.
	\begin{enumerate}
		\item The system opens a pop-up window to create the question with the image and its response.
		\item The user fills in the pop-up window with the question, the image and the appropriate answer, and press ``Accept'' when it finishes.
		\item Step 8.
	\end{enumerate}
	3. c) The user chooses the ``Join'' option.
	\begin{enumerate}
		\item The system opens a pop-up window to create the join quiz.
		\item The user populates the pop-up window with the possible answers and their correct answer and press ``Accept'' when it finishes.
		\item Step 8.
	\end{enumerate}
	*a) At any time, the user can go back.
	\item \textbf{Variations:} None.
	\item \textbf{Not-functional:} None.
	\item \textbf{Issues:} None.
\end{itemize}

\subsection{Visualizar \textit{quiz}}
\begin{itemize}
	\item \textbf{Descripción:} Muestra el estado del \textit{quiz}.
	\item \textbf{Actores:} Creador-editor (usuario).
	\item \textbf{Precondiciones:} Debe existir el punto activo correspondiente, se debe estar editando la fantasía y debe existir un \textit{quiz}.
	\item \textbf{Postcondiciones:} Muestra el estado del \textit{quiz} en el punto activo correspondiente.
	\item \textbf{Escenario principal:}
	\begin{enumerate}
		\item El usuario selecciona el punto activo correspondiente.
		\item El sistema muestra una ventana emergente con la información del punto activo.
		\item El usuario selecciona la opción de ``Leer \textit{quiz}''.
		\item El sistema muestra una ventana emergente con la visión final del \textit{quiz}.
	\end{enumerate}
	\item \textbf{Extensiones:} \\ *a) En cualquier momento, el usuario puede volver atrás.
	\item \textbf{Variaciones:} Ninguna.
	\item \textbf{No-funcional:} Ninguna.
	\item \textbf{Cuestiones:} Ninguna.
\end{itemize}

\subsection{Modificar \textit{quiz}}
\begin{itemize}
	\item \textbf{Descripción:} Permite modificar el \textit{quiz}.
	\item \textbf{Actores:} Creador-editor (usuario).
	\item \textbf{Precondiciones:} Debe existir el punto activo correspondiente, se debe estar editando la fantasía y debe existir un \textit{quiz}.
	\item \textbf{Postcondiciones:} Modifica el \textit{quiz} de un punto activo.
	\item \textbf{Escenario principal:}
	\begin{enumerate}
		\item El usuario selecciona el punto activo correspondiente.
		\item El sistema muestra una ventana emergente con la información del punto activo.
		\item El usuario selecciona la opción ``Modificar \textit{quiz}''.
		\item Paso 4 de \hyperlink{crearquiz}{\textcolor{blue}{Crear \textit{Quiz}}}
	\end{enumerate}
	\item \textbf{Extensiones:} \\ *a) En cualquier momento, el usuario puede volver atrás.
	\item \textbf{Variaciones:} Ninguna.
	\item \textbf{No-funcional:} Ninguna.
	\item \textbf{Cuestiones:} Ninguna.
\end{itemize}

\subsection{Borrar \textit{quiz}}
\begin{itemize}
	\item \textbf{Descripción:} Borra el \textit{quiz} del punto activo seleccionado.
	\item \textbf{Actores:} Creador-editor (usuario).
	\item \textbf{Precondiciones:} Debe existir el punto activo correspondiente, se debe estar editando la fantasía y debe existir un \textit{quiz}.
	\item \textbf{Postcondiciones:} Borra el \textit{quiz} del punto activo seleccionado.
	\item \textbf{Escenario principal:}
	\begin{enumerate}
		\item El usuario selecciona el punto activo correspondiente.
		\item El sistema muestra una ventana emergente con la información del punto activo.
		\item El usuario selecciona la opción de ``Borrar \textit{quiz}''.
		\item El sistema muestra un mensaje de confirmación.
		\item El usuario selecciona ``Aceptar''.
		\item El sistema borra el \textit{quiz} del punto activo.
	\end{enumerate}
	\item \textbf{Extensiones:} \\ 5. a) El usuario selecciona ``Cancelar''.
	\begin{enumerate}
		\item El sistema cierra la ventana emergente.
		\item Paso 1.
	\end{enumerate}
	*a) En cualquier momento, el usuario puede volver atrás.
	\item \textbf{Variaciones:} Ninguna.
	\item \textbf{No-funcional:} Ninguna.
	\item \textbf{Cuestiones:} Ninguna.
\end{itemize}

\section{CRUD efecto de audio}
\hypertarget{crearaudio}{}
\subsection{Crear efecto de audio}
\begin{itemize}
	\item \textbf{Descripción:} Establece un efecto de audio de fondo en el punto activo.
	\item \textbf{Actores:} Creador-editor (usuario).
	\item \textbf{Precondiciones:} Debe existir el punto activo correspondiente y se debe estar editando la fantasía.
	\item \textbf{Postcondiciones:} Establece el efecto de audio de fondo.
	\item \textbf{Escenario principal:}
	\begin{enumerate}
		\item El usuario selecciona el punto activo correspondiente.
		\item El sistema muestra una ventana emergente con la información del punto activo.
		\item El usuario selecciona la opción de ``Añadir efecto de audio''.
		\item El sistema muestra una ventana emergente en la que da a elegir al usuario de donde quiere seleccionar el audio (Internet, local, audio ya usado en la fantasía).
		\item EL usuario elige la opción ``Internet'' para incluir un audio de Internet.
		\item El sistema le pide al usuario la url del audio.
		\item El usuario inserta la url del audio.
		\item El sistema guarda el audio en el punto activo.
	\end{enumerate}
	\item \textbf{Extensiones:} \\ 5. a) El usuario elige la opción ``Local'' para incluir un audio desde su ordenador.
	\begin{enumerate}
		\item El sistema abre una ventana del explorador de archivos.
		\item El usuario selecciona el audio deseado y pulsa ``Aceptar''.
		\item El sistema cierra la ventana del explorador de archivos.
		\item Paso 8.
	\end{enumerate}
	5. b) El usuario elige la opción ``Audio usado anteriormente'' para incluir un audio ya usado.
	\begin{enumerate}
		\item El sistema abre una ventana con los audios usados anteriormente.
		\item El usuario selecciona el audio deseado y pulsa aceptar.
		\item El sistema cierra la ventana emergente.
		\item Paso 8.
	\end{enumerate}
	7. a) La url no es correcta.
	\begin{enumerate}
		\item El sistema muestra un mensaje de error.
		\item Paso 6.
	\end{enumerate}
	*a) En cualquier momento, el usuario puede volver atrás.
	\item \textbf{Variaciones:} Ninguna.
	\item \textbf{No-funcional:} Ninguna.
	\item \textbf{Cuestiones:} Ninguna.
\end{itemize}

\subsection{Modificar efecto de audio}
\begin{itemize}
	\item \textbf{Descripción:} Modificar efecto de audio.
	\item \textbf{Actores:} Creador-editor (usuario).
	\item \textbf{Precondiciones:} Debe existir el punto activo correspondiente, se debe estar editando la fantasía y debe existir un audio.
	\item \textbf{Postcondiciones:} Modifica el audio.
	\item \textbf{Escenario principal:}
	\begin{enumerate}
		\item El usuario selecciona el punto activo.
		\item El sistema abre una ventana emergente con la información del punto activo.
		\item Paso 4 de \hyperlink{crearaudio}{\textcolor{blue}{Crear audio}}
	\end{enumerate}
	\item \textbf{Extensiones:} \\ *a) En cualquier momento, el usuario puede volver atrás.
	\item \textbf{Variaciones:} Ninguna.
	\item \textbf{No-funcional:} Ninguna.
	\item \textbf{Cuestiones:} Ninguna.
\end{itemize}

\subsection{Borrar efecto de audio}
\begin{itemize}
	\item \textbf{Descripción:} Borra un efecto de audio de un punto activo.
	\item \textbf{Actores:} Creador-editor (usuario).
	\item \textbf{Precondiciones:} Debe existir el punto activo correspondiente, se debe estar editando la fantasía y debe existir un audio.
	\item \textbf{Postcondiciones:} Borra un efecto de audio de un punto activo.
	\item \textbf{Escenario principal:}
	\begin{enumerate}
		\item El usuario selecciona el punto activo.
		\item El sistema abre una ventana emergente con la información del punto activo.
		\item El usuario selecciona el audio y pulsa el botón ``Suprimir''.
		\item El sistema muestra un mensaje de confirmación.
		\item El usuario selecciona ``Aceptar''.
		\item El sistema borra el audio del punto activo.
	\end{enumerate}
	\item \textbf{Extensiones:} \\ 5. a) El usuario selecciona ``Cancelar''.
	\begin{enumerate}
		\item El sistema cierra la ventana emergente.
		\item Paso 1.
	\end{enumerate}
	*a) En cualquier momento, el usuario puede volver atrás.
	\item \textbf{Variaciones:} Ninguna.
	\item \textbf{No-funcional:} Ninguna.
	\item \textbf{Cuestiones:} Ninguna.
\end{itemize}


\section{CRUD información adicional}
\begin{itemize}
	\item \textbf{Descripción:} Inserta un texto como información adicional de la fantasía.
	\item \textbf{Actores:} Creador-editor (usuario).
	\item \textbf{Precondiciones:} Se debe estar editando la fantasía correspondiente.
	\item \textbf{Postcondiciones:} Inserta un texto como información adicional de la fantasía.
	\item \textbf{Escenario principal:}
	\begin{enumerate}
		\item El usuario selecciona la opción ``Información adicional''.
		\item El sistema muestra una ventana emergente con un cuadro de texto.
		\item El usuario introduce el texto deseado en el cuadro de texto con las opciones de formato que desee. 
		\item El usuario pulsa en el botón ``Aceptar''.
		\item El sistema guarda la información adicional en la fantasía correspondiente.
	\end{enumerate}
	\item \textbf{Extensiones:} \\ *a) En cualquier momento, el usuario puede volver atrás.
	\item \textbf{Variaciones:} Ninguna.
	\item \textbf{No-funcional:} Ninguna.
	\item \textbf{Cuestiones:} Ninguna.
\end{itemize}

\section{Organizar puntos activos}
\begin{itemize}
	\item \textbf{Descripción:} Organiza la aparición de los puntos activos.
	\item \textbf{Actores:} Creador-editor (usuario).
	\item \textbf{Precondiciones:} La fantasía debe estar creada.
	\item \textbf{Postcondiciones:} Establece el orden de aparición de los puntos activos de la fantasía.
	\item \textbf{Escenario principal:}
	\begin{enumerate}
		\item El usuario selecciona la fantasía correspondiente.
		\item El sistema muestra una ventana con la información de la fantasía y las opciones disponibles.
		\item El usuario selecciona la opción ``Organizar puntos activos''.
		\item El sistema muestra una ventana emergente con el nombre de los puntos activos existentes en la fantasía y un recuadro para establecer el orden de aparición.
		\item El usuario establece el orden de aparición en los recuadros junto al nombre de los puntos activos de la fantasía.
		\item El usuario pulsa ``Aceptar'' para guardar los cambios realizados.
		\item El sistema guarda el orden de aparición de los puntos activos.
	\end{enumerate}
	\item \textbf{Extensiones:} \\ *a) En cualquier momento, el usuario puede volver atrás.
	\item \textbf{Variaciones:} Ninguna.
	\item \textbf{No-funcional:} Ninguna.
	\item \textbf{Cuestiones:} Ninguna.
\end{itemize}

\section{CRUD \textit{quiz} final}
\hypertarget{crearquizfinal}{}
\subsection{Crear \textit{quiz} final}
\begin{itemize}
	\item \textbf{Descripción:} Crea un cuestionario sobre el tema del que trata la fantasía.
	\item \textbf{Actores:} Creador-editor (usuario).
	\item \textbf{Precondiciones:} Debe existir y se debe estar editando la fantasía.
	\item \textbf{Postcondiciones:} Crea un cuestionario en relación al tema de la fantasía..
	\item \textbf{Escenario principal:}
	\begin{enumerate}
		\item El usuario pulsa el botón de ``Cuestionario final''.
		\item El sistema muestra las posibles opciones.
		\item El usuario selecciona ``Crear nuevo quiz''.
		\item El sistema muestras las posibles opciones de creación.
		\item El usuario selecciona ``Respuesta simple''.
		\item El sistema muestra una ventana emergente para crear la pregunta con sus posibles respuestas.
		\item El usuario rellena la ventana emergente con la pregunta y las respuestas convenientes y pulsa ``Aceptar'' cuando termina.
		\item El sistema cierra la ventana emergente.
		\item El cuestionario queda registrado en la fantasía.
	\end{enumerate}
	\item \textbf{Extensiones:} \\5. a) El usuario elige la opción ``Palabra''.
	\begin{enumerate}
		\item El sistema abre una ventana emergente para crear la pregunta y su respuesta.
		\item El usuario rellena la ventana emergente con la pregunta y la respuesta conveniente y pulsa ``Aceptar'' cuando termina.
		\item Paso 8.
	\end{enumerate}
	5. b) El usuario elige la opción ``Quiz con imágenes''.
	\begin{enumerate}
		\item El sistema abre una ventana emergente para crear la pregunta con la imagen y su respuesta.
		\item El usuario rellena la ventana emergente con la pregunta, la imagen y la respuesta conveniente, y pulsa ``Aceptar'' cuando termina.
		\item Paso 8.
	\end{enumerate}
	5. c) El usuario elige la opción ``Unir''.
	\begin{enumerate}
		\item El sistema abre una ventana emergente para crear el quiz de unión.
		\item El usuario rellena la ventana emergente con las posibles respuestas y su respuesta correcta y pulsa ``Aceptar'' cuando termina.
		\item Paso 8.
	\end{enumerate}
	*a) En cualquier momento, el usuario puede volver atrás.
	\item \textbf{Variaciones:} Ninguna.
	\item \textbf{No-funcional:} Ninguna.
	\item \textbf{Cuestiones:} Ninguna.
\end{itemize}

\subsection{Visualizar \textit{quiz} final}
\begin{itemize}
	\item \textbf{Descripción:} Muestra el estado del \textit{quiz}.
	\item \textbf{Actores:} Creador-editor (usuario).
	\item \textbf{Precondiciones:} Debe existir y se debe estar editando la fantasía, y debe existir el \textit{quiz} final.
	\item \textbf{Postcondiciones:} Muestra el estado del \textit{quiz} final.
	\item \textbf{Escenario principal:}
	\begin{enumerate}
		\item El usuario selecciona el botón de ``Cuestionario final''.
		\item El sistema muestra las posibles opciones.
		\item El usuario selecciona la opción de ``Leer \textit{quiz} final''.
		\item El sistema muestra una ventana emergente con la visión final del \textit{quiz}.
	\end{enumerate}
	\item \textbf{Extensiones:} \\ *a) En cualquier momento, el usuario puede volver atrás.
	\item \textbf{Variaciones:} Ninguna.
	\item \textbf{No-funcional:} Ninguna.
	\item \textbf{Cuestiones:} Ninguna.
\end{itemize}

\subsection{Modificar \textit{quiz} final}
\begin{itemize}
	\item \textbf{Descripción:} Permite modificar el \textit{quiz} final.
	\item \textbf{Actores:} Creador-editor (usuario).
	\item \textbf{Precondiciones:} Debe existir y se debe estar editando la fantasía, y debe existir el \textit{quiz} final.
	\item \textbf{Postcondiciones:} Modifica el \textit{quiz} final de la fantasía.
	\item \textbf{Escenario principal:}
	\begin{enumerate}
		\item El usuario pulsa el botón de "Cuestionario final".
		\item El sistema muestra las posibles opciones.
		\item El usuario selecciona la opción ``Modificar \textit{quiz} final''.
		\item Caso de uso \hyperlink{crearquizfinal}{\textcolor{blue}{Crear \textit{Quiz} final}}
	\end{enumerate}
	\item \textbf{Extensiones:} \\ *a) En cualquier momento, el usuario puede volver atrás.
	\item \textbf{Variaciones:} Ninguna.
	\item \textbf{No-funcional:} Ninguna.
	\item \textbf{Cuestiones:} Ninguna.
\end{itemize}

\subsection{Borrar \textit{quiz} final}
\begin{itemize}
	\item \textbf{Descripción:} Borra el \textit{quiz} final de la fantasía.
	\item \textbf{Actores:} Creador-editor (usuario).
	\item \textbf{Precondiciones:} Debe existir y se debe estar editando la fantasía, y debe existir el \textit{quiz} final.
	\item \textbf{Postcondiciones:} Borra el \textit{quiz} final de la fantasía.
	\item \textbf{Escenario principal:}
	\begin{enumerate}
		\item El usuario pulsa el botón de ``Cuestionario final''.
		\item El sistema muestra las posibles opciones.
		\item El usuario selecciona la opción de ``Borrar \textit{quiz}'' final.
		\item El sistema muestra un mensaje de confirmación.
		\item El usuario selecciona ``Aceptar''.
		\item El sistema borra el \textit{quiz} final de la fantasía.
	\end{enumerate}
	\item \textbf{Extensiones:} \\ 5. a) El usuario selecciona ``Cancelar''.
	\begin{enumerate}
		\item El sistema cierra la ventana emergente.
		\item Paso 1.
	\end{enumerate}
	*a) En cualquier momento, el usuario puede volver atrás.
	\item \textbf{Variaciones:} Ninguna.
	\item \textbf{No-funcional:} Ninguna.
	\item \textbf{Cuestiones:} Ninguna.
\end{itemize}


\section{Gestionar porcentaje de un punto activo}
\begin{itemize}
	\item \textbf{Descripción:}
	\item \textbf{Actores:}
	\item \textbf{Precondiciones:}
	\item \textbf{Postcondiciones:}
	\item \textbf{Escenario principal:}
	\item \textbf{Extensiones:}
	\item \textbf{Variaciones:}
	\item \textbf{No-funcional:}
	\item \textbf{Cuestiones:}
\end{itemize}

\section{Gestionar ficha alumno}
\begin{itemize}
	\item \textbf{Descripción:}
	\item \textbf{Actores:}
	\item \textbf{Precondiciones:}
	\item \textbf{Postcondiciones:}
	\item \textbf{Escenario principal:}
	\item \textbf{Extensiones:}
	\item \textbf{Variaciones:}
	\item \textbf{No-funcional:}
	\item \textbf{Cuestiones:}
\end{itemize}

\section{Asignar nota final}
\begin{itemize}
	\item \textbf{Descripción:}
	\item \textbf{Actores:}
	\item \textbf{Precondiciones:}
	\item \textbf{Postcondiciones:}
	\item \textbf{Escenario principal:}
	\item \textbf{Extensiones:}
	\item \textbf{Variaciones:}
	\item \textbf{No-funcional:}
	\item \textbf{Cuestiones:}
\end{itemize}

\section{Asignar fantasía}
\begin{itemize}
	\item \textbf{Descripción:}
	\item \textbf{Actores:}
	\item \textbf{Precondiciones:}
	\item \textbf{Postcondiciones:}
	\item \textbf{Escenario principal:}
	\item \textbf{Extensiones:}
	\item \textbf{Variaciones:}
	\item \textbf{No-funcional:}
	\item \textbf{Cuestiones:}
\end{itemize}