\chapter{Analisys of requirements}
\section{Workspace}
The design used should be the same as that of the \href{https://stimey.eu/home}{Stimey} page in all the icons used.

%El diseño gráfico usado deberá ser el mismo que el de la página de \href{https://stimey.eu/home}{Stimey} en todos los iconos usados.

We can highlight two user roles: teachers and students.

%Podremos destacar dos roles de usuario: profesores y usuarios.

The teacher will have more permissions and privileges than the student, so that he can create fantasies to evaluate his students and they will have to complete them or create the ones that the teacher puts them as work.

%El profesor tendrá más permisos y privilegios que el alumno, de forma que pueda crear fantasías para evaluar a sus alumnos y ellos tendrán que completarlas o crear las que el profesor les ponga como trabajo.

In the workspace we will have a series of options that will be available for both the teacher and student roles depending on the permissions of each role:

%En el workspace tendremos una serie de opciones que estarán disponibles tanto para el rol de profesor como de alumno en función de los permisos de cada rol:
\begin{itemize}
	\item \textbf{Background:} A window opens where you can select a previously used image, google or computer. This image will cover all the workspace. It will also be possible to enter text by adding it manually or through a link.
	\item \textbf{Active point:} Set an active point in the workspace (drag and drop).
	\begin{itemize}
		\item It will be possible to move it and modify the contents of it.
		\item If an image is added to the active puto, that point is adapted to the shape of the image.
		\item Once the active point originates, a pop-up with a text opens.
		\item You can also assign a video, which will open a window to play it, or an audio. In case there is no audio or video, the respective button will not be displayed.
		\item Active points may have background music that will be muted if the audio or video playback assigned to that point by the faculty begins. The music will be restored when the corresponding audio or video ends.
		\item The active points can be reorganized by the teachers so that they emerge in the order they want.
		\item A student can not continue with the next active point without finishing the current one.
		\item The active points quiz should be fun.
		\item The questions raised in the quiz of the active points should be 2 and not too difficult (multiple answer, write a word, quiz with images and questions about it, join items, etc).
		\item The quiz will appear on screen when the current active point is closed.
		\item Once the quiz is finished, the next active point appears in the order established by the faculty in the workspace.
		\item Each active point will have a score to add up to a maximum of 100 points.
		\item When assigning a score to an active point, it will be subtracted from the total we carry (maximum 100 points). If an active point is eliminated, the general counter recovers the score assigned to that active point.
		\item The student does not know the total number of active points in total.
		\item When the student obtains a score when completing an active point, this amount is added to the global counter.
		\item Finally, we can have a statistical summary with the right/failed questions of each active point.
		\item Only the score obtained will be saved the first time a quiz is done, then it can be done more times, but the note will not be recorded.
		\item The student will have the option to save their progress with a save button manually or through the auto-save option.
	\end{itemize}
\end{itemize}

\section{Characteristics}
\begin{itemize}
	\item At the end of all active points there will be a button at the bottom right of ``\textbf{more information}'' and in the center a new quiz that will be the final exam. This exam will have an independent score to all active points and will not have the statistical summary. If this quiz is repeated, the note would be updated with a percentage of the new grade, plus the previous note with the objective that a student who repeats a quiz can not get the best grade per repetition of it.
	\item Teachers can send students to make fantasies to learn as homework. These tasks can be done in groups of students based on two ideas:
	\begin{enumerate}
		\item \textbf{Obligatory:} One student makes the fantasy and the rest looks for additional information.
		\item \textbf{Optional:} Concurrent edition of the fantasy among all the members of the group.
	\end{enumerate}
	\item Each fantasy will have a code to be shared.
	\item We will have two types of permissions in the fantasies: ``\textbf{see}'' and ``\textbf{see and edit}''.
	\item The platform will notify the faculty when the students have finished their respective jobs.
	\item Fantasies may be private or public. By default, they will always be public and can be accessed by everyone who uses the platform.
	\item Private fantasies can be accessed by other people with a password.
	\item Fantasies can be cloned.
\end{itemize}
