\chapter{Introduction}
\section{Motivation}
It is a work of the subject ``Proyectos Informaticos'' which, at a professional level, helps us to gain work experience and face real situations facing a demanding clientele.

\section{Description of the current system}
Initially, our client had an application that showed information about a topic on a page and the students did not focus on learning, but went directly to the final questionnaire in order to finish earlier. This means that students did not learn properly or encourage their imagination or creativity.

\section{Objectives and scope of the project}
\subsection{Objetives}
Motivation of creativity and promotion of imagination in children.

To fulfill the general objective, we will have to cover the following points:
\begin{itemize}
	\item Interactive learning resources.
	\item Can be evaluated by a teacher.
	\item You can share fantasies between users.
	\item It is simple and manageable by primary school students.
	\item Encourage STEM teaching skills (science, technology, engeneering and maths).
\end{itemize}

\subsection{Scope}
The students will be able to create fantasies, share them and they will be able to be evaluated by the professors, who will be able to send as a task the making of fantasies.

\section{Organization of the document}
This document is organized according to the specifications set out for the presentation of an end-of-degree project following the following sections:
\begin{enumerate}
	\item Introduction.
	\item Project plan.
	\item Analysis of requirements.
	\item System design.
	\item Implementation of the system.
	\item System tests.
	\item User manual.
	\item Installation manual.
	\item Conclusions.
\end{enumerate}

In addition to this document, we also have an appendix where the user manual is narrated, step by step.
