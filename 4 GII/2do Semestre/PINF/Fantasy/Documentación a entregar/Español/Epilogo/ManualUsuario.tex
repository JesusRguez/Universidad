\chapter{Manual de usuario}
\section{Introducción}

Aquí se narrará el manual que han de seguir los usuarios del sistema para usar la aplicación\footnote{Para mayor detalle sobre el manual de usuario, consultar el apéndice.}.

Teniendo en cuenta que será una aplicación web, el primer paso será dirigirse a la dirección web donde esté ubicado el servicio\footnote{Si ha sido integrado en la plataforma de STIMEY, estará en la zona de laboratorios.}.


\section{Características}

La aplicación Fantasy permite al usuario crear fantasías para que los estudiantes aprendan de una forma más creativa y divertida viendo los puntos activos y desarrollando los \textit{quizzes} asociados a cada punto activo, y, finalmente, el \textit{quiz} final asociado a cada fantasía.


Esta puntuación será enviada a la plataforma de STIMEY para ser almacenada en el perfil de cada alumno.


Los alumnos también podrán realizar fantasías que sean ordenadas por sus profesores como tarea. Esta tarea, podrá ser en pareja o individual, y será posteriormente evaluada por el profesor que puso la tarea.


\section{Requisitos previos}

Los requisitos previos a la hora de usar la aplicación del proyecto Fantasy es entrar en la dirección web donde se encuentre soportado el servicio y registrarse con la cuenta del usuario que vaya a usar la aplicación.


Por lo tanto, no es necesario tener nada instalado en el ordenador del usuario debido a que la aplicación se encuentra alojada en un servidor.


\section{Utilización}
A la hora de utilizar la aplicación y, habiendo accedido con nuestro usuario y contraseña a la plataforma, tendremos una pantalla principal donde podremos ver nuestras fantasías (las que hayamos creado nosotros) y las fantasías que hemos marcado como favoritas para volver a jugarlas.

\subsection{Crear fantasía}
Si queremos crear una fantasía, hacemos click sobre el icono de crear una nueva fantasía.

A continuación, rellenamos los campos necesarios y ya estaría creada la fantasía.

Una vez hayamos terminado con la creación de la fantasía, podremos volver atrás y verla en la parte reservada a nuestras fantasías.

\subsection{Modificar fantasía}
Si queremos modificar una fantasía, deberemos hacer click izquierdo sobre el icono de editar fantasía.

A continuación, modificaremos los campos que deseemos cambiar.

\subsection{Eliminar fantasía}
Si queremos eliminar una fantasía, haremos click izquierdo sobre el icono de borrar fantasía.

\subsection{Duplicar fantasía}

%Se supone que esto debe estar hecho

\subsection{Crear punto activo}
A la hora de añadir los puntos activos a la fantasía. Para ello hacemos click izquierdo en el icono de añadir un nuevo punto activo.

A continuación, rellenamos los campos necesarios y ya tendríamos creado el primer punto activo.

Podremos crear hasta un máximo de diez puntos activos siguiendo los pasos anteriormente mencionados para cada punto activo.


\subsection{Modificar punto activo}
En los puntos activos tendremos la posibilidad de mover y redimensionar el punto activo a nuestro gusto de forma que se quede donde y como nosotros deseemos.

Además, haciendo doble click izquierdo sobre ellos, tendremos la posibilidad de cambiar el contenido de sus campos.

\subsection{Eliminar punto activo}
%se supone que esto debe estar hecho

\subsection{Jugar fantasía}
Cuando queramos jugar una fantasía (bien sea creada por nosotros o por otro usuario), tendremos que hacer click izquierdo en el icono de jugar la fantasía y se podrá jugar\footnote{Tenemos que tener en cuenta que para los alumnos, solo guardaremos la primera nota que saquen, aunque puedan repetir al fantasía tantas veces como quieran.}.
