\chapter{Casos de uso}
\noindent
Todos los casos de uso descritos a continuación, tienen la siguiente precondición implícita para poder usar dichos casos de uso en la aplicación final:
\begin{itemize}
	\item El usuario (profesorado/alumnado) debe tener una cuenta en la plataforma de Stimey y haber iniciado sesión con dicha cuenta.
\end{itemize}

\section{CRUD fantasía}
\hypertarget{crearfantasia}{}
\subsection{Crear fantasía}
\begin{itemize}
	\item \textbf{Descripción:} Crea una nueva fantasía.
	\item \textbf{Actores:} Creador-editor (usuario).
	\item \textbf{Precondiciones:} El usuario debe tener permisos para crear una nueva fantasía.
	\item \textbf{Postcondiciones:} La fantasía queda almacenada en el sistema.
	\item \textbf{Escenario principal:}
	\begin{enumerate}
		\item El usuario selecciona la opción ``Crear nueva fantasía''.
		\item El sistema solicita el nombre de la fantasía.
		\item El usuario introduce el nombre de la fantasía.
		\item El sistema solicita el código de la fantasía.
		\item El usuario introduce el código de la fantasía.
		\item El sistema da a elegir si la fantasía será pública (por defecto), compartida o privada.
		\item El usuario selecciona ``Pública''.
		\item El usuario crea la fantasía.
		\item La fantasía queda almacenada en el sistema.
	\end{enumerate}
	\item \textbf{Extensiones:} \\7. a) El usuario selecciona que la fantasía será compartida.
	\begin{enumerate}
		\item El sistema permite insertar en una lista los identificadores de otros usuarios con los que quedará compartida la fantasía.
		\item El usuario introduce los identificadores de los usuarios que compartirán la fantasía.
		\item Paso 8.
	\end{enumerate}
	7. b) El usuario selecciona que la fantasía será privada.
	\begin{enumerate}
		\item El sistema marca la fantasía como privada para dicho usuario sin dar la posibilidad de compartir.
		\item Paso 8.
	\end{enumerate}
	*a) En cualquier momento, el usuario puede volver atrás al menú principal.
	\item \textbf{Variaciones:} Ninguna.
	\item \textbf{No-funcional:} Ninguna.
	\item \textbf{Cuestiones:} Ninguna.
\end{itemize}

\subsection{Visualizar fantasía}
\begin{itemize}
	\item \textbf{Descripción:} Lee una fantasía ya existente.
	\item \textbf{Actores:} Creador-editor (usuario).
	\item \textbf{Precondiciones:} La fantasía debe existir en el sistema y el usuario debe tener permisos de modificación.
	\item \textbf{Postcondiciones:} No se producen cambios en la fantasía.
	\item \textbf{Escenario principal:}
	\begin{enumerate}
		\item El usuario selecciona la opción ``Mis fantasías''
		\item El sistema muestra una lista de las fantasías accesibles por el usuario.
		\item El usuario selecciona la fantasía que desea visualizar.
		\item El sistema muestra una ventana emergente con la información de la fantasía y sus opciones.
		\item El usuario selecciona la opción ``Visualizar fantasía''.
		\item El sistema muestra la fantasía.
		\item El usuario lee la fantasía sin hacer ningún cambio y, cuando acaba, cierra la fantasía.
		\item La fantasía queda sin modificar.
	\end{enumerate}
	\item \textbf{Extensiones:} \\ *a) En cualquier momento, el usuario puede volver atrás al menú principal.
	\item \textbf{Variaciones:} Ninguna.
	\item \textbf{No-funcional:} Ninguna.
	\item \textbf{Cuestiones:} Ninguna.
\end{itemize}

\subsection{Modificar fantasía}
\begin{itemize}
	\item \textbf{Descripción:} Modifica una fantasía ya existente.
	\item \textbf{Actores:} Creador-editor (usuario).
	\item \textbf{Precondiciones:} La fantasía debe existir en el sistema y el usuario debe tener permisos de modificación.
	\item \textbf{Postcondiciones:} La fantasía queda modificada.
	\item \textbf{Escenario principal:}
	\begin{enumerate}
		\item El usuario selecciona la opción ``Mis fantasías''.
		\item El sistema muestra una lista de las fantasías accesibles por el usuario.
		\item El usuario selecciona la fantasía que desea modificar.
		\item El sistema muestra una ventana emergente con la información de la fantasía y sus opciones.
		\item El usuario selecciona la opción ``Modificar fantasía''.
		\item El sistema muestra la pantalla de creación de la fantasía para su modificación.
	\end{enumerate}
	\item \textbf{Extensiones:} \\ *a) En cualquier momento, el usuario puede volver atrás al menú principal.
	\item \textbf{Variaciones:} Ninguna.
	\item \textbf{No-funcional:} Ninguna.
	\item \textbf{Cuestiones:} Ninguna.
\end{itemize}

\subsection{Borrar fantasía}
\begin{itemize}
	\item \textbf{Descripción:} Borra una fantasía ya existente.
	\item \textbf{Actores:} Creador-editor (usuario).
	\item \textbf{Precondiciones:} La fantasía debe existir en el sistema y el usuario debe tener permisos de eliminación.
	\item \textbf{Postcondiciones:} La fantasía es eliminada del sistema.
	\item \textbf{Escenario principal:}
	\begin{enumerate}
		\item El usuario selecciona la opción ``Mis fantasías''.
		\item El sistema muestra una lista de las fantasías accesibles por el usuario.
		\item El usuario selecciona la fantasía que desea modificar.
		\item El sistema muestra una ventana emergente con la información de la fantasía y sus opciones.
		\item El usuario selecciona la opción ``Borrar fantasía''.
		\item El sistema muestra un mensaje de confirmación.
		\item El usuario selecciona ``Aceptar''.
		\item El sistema borra la fantasía.
	\end{enumerate}
	\item \textbf{Extensiones:}  \\7. a) El usuario selecciona ``Cancelar''.
	\begin{enumerate}
		\item El sistema cierra la ventana emergente.
		\item Paso 1.
	\end{enumerate}
	*a) En cualquier momento, el usuario puede volver atrás al menú principal.
	\item \textbf{Variaciones:} Ninguna.
	\item \textbf{No-funcional:} Ninguna.
	\item \textbf{Cuestiones:} Ninguna.
\end{itemize}


\section{Elegir idioma}
\begin{itemize}
	\item \textbf{Descripción:} Cambia el idioma de la aplicación.
	\item \textbf{Actores:} Profesor o alumno (usuario).
	\item \textbf{Precondiciones:} Ninguna. %el usuario es capaz de encontrar el menú de idioma con la mirada
	\item \textbf{Postcondiciones:} La aplicación cambia al idioma seleccionado por el usuario.
	\item \textbf{Escenario principal:}
	\begin{enumerate}
		\item El usuario pulsa el botón de cambio de idioma.
		\item El sistema despliega una lista de los idiomas disponibles.
		\item El usuario  selecciona un idioma de los que están disponibles en el sistema.
		\item La aplicación cambia el idioma.
	\end{enumerate}
	\item \textbf{Extensiones:} Ninguna.
	\item \textbf{Variaciones:} Ninguna.
	\item \textbf{No-funcional:} Ninguna.
	\item \textbf{Cuestiones:} Ninguna.
\end{itemize}

\section{Copiar fantasía}
\begin{itemize}
	\item \textbf{Descripción:} Clona una fantasía.
	\item \textbf{Actores:} Creador-editor (usuario).
	\item \textbf{Precondiciones:} La fantasía debe existir en el sistema y el usuario debe tener permisos de modificación.
	\item \textbf{Postcondiciones:} Crea una copia de la fantasía seleccionada.
	\item \textbf{Escenario principal:}
	\begin{enumerate}
		\item El usuario selecciona la opción ``Mis fantasías''.
		\item El sistema muestra una lista de las fantasías accesibles por el usuario.
		\item El usuario selecciona la fantasía que desea copiar.
		\item El sistema muestra una ventana emergente con la información de la fantasía y sus opciones.
		\item El usuario selecciona la opción ``Copiar fantasía''.
		\item El sistema crea una copia de la fantasía seleccionada.
	\end{enumerate}
	\item \textbf{Extensiones:} \\ *a) En cualquier momento, el usuario puede volver atrás al menú principal.
	\item \textbf{Variaciones:} Ninguna.
	\item \textbf{No-funcional:} Ninguna.
	\item \textbf{Cuestiones:} Ninguna.
\end{itemize}

\section{CRUD background}
\begin{itemize}
	\item \textbf{Descripción:} Permite seleccionar, modificar y borrar el background.
	\item \textbf{Actores:} Creador-editor (usuario).
	\item \textbf{Precondiciones:} La fantasía debe existir en el sistema y el usuario debe tener permisos de modificación.
	\item \textbf{Postcondiciones:} Se establece el fondo que el usuario haya elegido.
	\item \textbf{Escenario principal:}
	\begin{enumerate}
		\item El usuario selecciona la opción ``Background``.
		\item El sistema muestra una ventana para añadir un background al workspace.
		\item El usuario selecciona una imagen. %puede ser un color
		\item El sistema establece el background seleccionado por el usuario.
	\end{enumerate}
	\item \textbf{Extensiones:} \\ *a) En cualquier momento, el usuario puede volver atrás al menú principal. 
	\item \textbf{Variaciones:} Ninguna.
	\item \textbf{No-funcional:} Ninguna.
	\item \textbf{Cuestiones:} Ninguna.
\end{itemize}


\section{CRUD punto activo}
\subsection{Crear punto activo}
\begin{itemize}
	\item \textbf{Descripción:} Crea un punto activo nuevo.
	\item \textbf{Actores:} Creador-editor (usuario).
	\item \textbf{Precondiciones:} La fantasía debe existir en el sistema y el usuario debe tener permisos de modificación.
	\item \textbf{Postcondiciones:} Se crea un punto activo vacío en el workspace.
	\item \textbf{Escenario principal:}
	\begin{enumerate}
		\item El usuario selecciona la opción ``Nuevo punto activo''.
		\item El sistema crea un nuevo punto activo en el workspace.
		\item El usuario puede mover el punto activo a la zona del workspace que desee.
		\item El sistema guardará el punto activo en la fantasía.
	\end{enumerate}
	\item \textbf{Extensiones:} \\ *a) En cualquier momento, el usuario puede volver atrás.
	\item \textbf{Variaciones:} Ninguna.
	\item \textbf{No-funcional:} Ninguna.
	\item \textbf{Cuestiones:} Ninguna.
\end{itemize}

\subsection{Visualizar punto activo}
\begin{itemize}
	\item \textbf{Descripción:} Muestra un punto activo existente para su lectura.
	\item \textbf{Actores:} Creador-editor (usuario).
	\item \textbf{Precondiciones:} La fantasía debe existir en el sistema y el punto activo debe existir en la fantasía. Además, el usuario debe tener permisos de modificación.
	\item \textbf{Postcondiciones:} Se muestra el punto activo para su lectura.
	\item \textbf{Escenario principal:}
	\begin{enumerate}
		\item El usuario selecciona el punto activo que desea visualizar.
		\item El sistema muestra una ventana con la información del punto activo y sus opciones.
		\item El usuario selecciona la opción ``Visualizar''.
		\item El sistema muestra una ventana con el resumen de dicho punto activo.
	\end{enumerate}
	\item \textbf{Extensiones:} \\ *a) En cualquier momento, el usuario puede volver atrás.
	\item \textbf{Variaciones:} Ninguna.
	\item \textbf{No-funcional:} Ninguna.
	\item \textbf{Cuestiones:} Ninguna.
\end{itemize}

\subsection{Modificar punto activo}
\begin{itemize}
	\item \textbf{Descripción:} Modifica un punto activo existente.
	\item \textbf{Actores:} Creador-editor (usuario).
	\item \textbf{Precondiciones:} La fantasía debe existir en el sistema y el punto activo debe existir en la fantasía. Además, el usuario debe tener permisos de modificación.
	\item \textbf{Postcondiciones:} Modifica el punto activo seleccionado.
	\item \textbf{Escenario principal:}
	\begin{enumerate}
		\item El usuario selecciona el punto activo que desea modificar.
		\item El sistema muestra una ventana con la información del punto activo y sus opciones.
		\item El usuario selecciona la opción ``Modificar''.
		\item El sistema muestra la ventana de creación del punto activo.
	\end{enumerate}
	\item \textbf{Extensiones:} \\ *a) En cualquier momento, el usuario puede volver atrás.
	\item \textbf{Variaciones:} Ninguna.
	\item \textbf{No-funcional:} Ninguna.
	\item \textbf{Cuestiones:} Ninguna.
\end{itemize}

\subsection{Borrar punto activo}
\begin{itemize}
	\item \textbf{Descripción:} Borra un punto activo existente.
	\item \textbf{Actores:} Creador-editor (usuario).
	\item \textbf{Precondiciones:} La fantasía debe existir en el sistema y el punto activo debe existir en la fantasía. Además, el usuario debe tener permisos de modificación.
	\item \textbf{Postcondiciones:} Borra el punto activo seleccionado.
	\item \textbf{Escenario principal:}
	\begin{enumerate}
		\item El usuario selecciona el punto activo que desea eliminar.
		\item El sistema muestra una ventana con la información del punto activo y sus opciones.
		\item El usuario selecciona la opción ``Borrar''.
		\item El sistema muestra un mensaje de confirmación.
		\item El usuario selecciona ``Aceptar''.
		\item El sistema borra el punto activo.
	\end{enumerate}
	\item \textbf{Extensiones:} \\ 5. a) El usuario selecciona ``Cancelar''.
	\begin{enumerate}
		\item El sistema cierra la ventana emergente.
		\item Paso 1.
	\end{enumerate}
	*a) En cualquier momento, el usuario puede volver atrás.
	\item \textbf{Variaciones:} Ninguna.
	\item \textbf{No-funcional:} Ninguna.
	\item \textbf{Cuestiones:} Ninguna.
\end{itemize}

\section{CRUD imagen} %suponemos que es para un punto activo
\hypertarget{crearimagen}{}
\subsection{Crear imagen}
\begin{itemize}
	\item \textbf{Descripción:} Inserta una imagen en un punto activo.
	\item \textbf{Actores:} Creador-editor (usuario).
	\item \textbf{Precondiciones:} Debe existir el punto activo correspondiente y se debe estar editando la fantasía.
	\item \textbf{Postcondiciones:} Inserta una imagen en el punto activo seleccionado.
	\item \textbf{Escenario principal:}
	\begin{enumerate}
		\item El usuario selecciona el punto activo correspondiente dentro de la fantasía.
		\item El sistema muestra una ventana emergente con la información del punto activo.
		\item El usuario selecciona la opción de ``Insertar imagen''.
		\item El sistema muestra una ventana en la que da a elegir al usuario de dónde quiere seleccionar la imagen (Internet, local, imagen ya usada en la fantasía).
		\item El usuario elige la opción ``Internet'' para incluir una imagen de Internet.
		\item El sistema le pide al usuario la url de la imagen.
		\item El usuario inserta la url correcta de la imagen.
		\item El punto activo toma la forma de la imagen.
	\end{enumerate}
	\item \textbf{Extensiones:} \\5. a) El usuario elige la opción ``Local'' para incluir una imagen desde su ordenador.
	\begin{enumerate}
		\item El sistema abre una ventana del explorador de archivos.
		\item El usuario selecciona la imagen deseada y pulsa ``Aceptar''.
		\item El sistema cierra la ventana del explorador de archivos.
		\item Paso 8.
	\end{enumerate}
	5. b) El usuario elige la opción ``Imagen usada anteriormente'' para incluir una imagen ya usada.
	\begin{enumerate}
		\item El sistema abre una ventana con las imágenes usadas anteriormente.
		\item El usuario selecciona la imagen deseada y pulsa ``Aceptar''.
		\item El sistema cierra la ventana emergente.
		\item Paso 8.
	\end{enumerate}
	7. a) La url no es correcta.
	\begin{enumerate}
		\item El sistema muestra un mensaje de error.
		\item Paso 6.
	\end{enumerate}
	*a) En cualquier momento, el usuario puede volver atrás.
	\item \textbf{Variaciones:} Ninguna.
	\item \textbf{No-funcional:} Ninguna.
	\item \textbf{Cuestiones:} ¿Podrá modificar el tamaño original de la imagen o hacer recortes?
\end{itemize}

\subsection{Modificar imagen}
\begin{itemize}
	\item \textbf{Descripción:} Modifica una imagen.
	\item \textbf{Actores:} Creador-editor (usuario).
	\item \textbf{Precondiciones:} Debe existir el punto activo correspondiente, se debe estar editando la fantasía y debe existir una imagen.
	\item \textbf{Postcondiciones:} La imagen queda modificada.
	\item \textbf{Escenario principal:}
	\begin{enumerate}
		\item El usuario selecciona el punto activo correspondiente dentro de la fantasía.
		\item El sistema abre una ventana emergente con la información del punto activo.
		\item Paso 4 de \hyperlink{crearimagen}{\textcolor{blue}{Crear imagen}}.
	\end{enumerate}
	\item \textbf{Extensiones:} \\ *a) En cualquier momento, el usuario puede volver atrás.
	\item \textbf{Variaciones:} Ninguna.
	\item \textbf{No-funcional:} Ninguna.
	\item \textbf{Cuestiones:} Ninguna.
\end{itemize}

\subsection{Borrar imagen}
\begin{itemize}
	\item \textbf{Descripción:} Borra una imagen de un punto activo.
	\item \textbf{Actores:} Creador-editor (usuario).
	\item \textbf{Precondiciones:} Debe existir el punto activo correspondiente, se debe estar editando la fantasía y debe existir una imagen.
	\item \textbf{Postcondiciones:} Borra la imagen y deja el punto activo en su estado por defecto.
	\item \textbf{Escenario principal:}
	\begin{enumerate}
		\item El usuario selecciona el punto activo correspondiente dentro de la fantasía.
		\item El sistema abre una ventana emergente con la información del punto activo.
		\item El usuario selecciona la imagen y pulsa el botón ``Suprimir''.
		\item El sistema muestra un mensaje de confirmación.
		\item El usuario selecciona ``Aceptar''.
		\item El sistema borra la imagen del punto activo.
	\end{enumerate}
	\item \textbf{Extensiones:} \\ 5. a) El usuario selecciona ``Cancelar''.
	\begin{enumerate}
		\item El sistema cierra la ventana emergente.
		\item Paso 1.
	\end{enumerate}
	*a) En cualquier momento, el usuario puede volver atrás.
	\item \textbf{Variaciones:} Ninguna.
	\item \textbf{No-funcional:} Ninguna.
	\item \textbf{Cuestiones:} Ninguna.
\end{itemize}

\section{CRUD vídeo}
\hypertarget{crearvideo}{}
\subsection{Crear vídeo}
\begin{itemize}
	\item \textbf{Descripción:} Inserta un vídeo dentro de un punto activo.
	\item \textbf{Actores:} Creador-editor (usuario).
	\item \textbf{Precondiciones:} Debe existir el punto activo correspondiente y se debe estar editando la fantasía.
	\item \textbf{Postcondiciones:} Inserta un vídeo en el punto activo seleccionado.
	\item \textbf{Escenario principal:}
	\begin{enumerate}
		\item El usuario selecciona el punto activo correspondiente dentro de la fantasía.
		\item El sistema muestra una ventana emergente con la información del punto activo.
		\item El usuario selecciona la opción de ``Insertar vídeo''.
		\item El sistema muestra una ventana en la que da a elegir al usuario de dónde quiere seleccionar la imagen (Internet, local, vídeo ya usado en la fantasía).
		\item El usuario elige la opción ``Internet'' para incluir un vídeo de Internet.
		\item El sistema le pide al usuario la url del vídeo.
		\item El usuario introduce la url correcta del vídeo.
		\item El sistema guarda el vídeo en el punto activo.
	\end{enumerate}
	\item \textbf{Extensiones:} \\ 5. a) El usuario elige la opción ``Local'' para incluir un vídeo desde su ordenador.
	\begin{enumerate}
		\item El sistema muestra una venta del explorador de archivos.
		\item El usuario selecciona la imagen deseada y pulsa ``Aceptar''.
		\item El sistema cierra la ventana del explorador de archivos.
		\item Paso 8.
	\end{enumerate}
	5. b) El usuario elige la opción ``Vídeo usado anteriormente'' para incluir un vídeo ya usado.
	\begin{enumerate}
		\item El sistema abre una ventana con los vídeos usados anteriormente.
		\item El usuario selecciona el vídeo deseado y pulsa ``Aceptar''.
		\item El sistema cierra la ventana emergente.
		\item Paso 4.
	\end{enumerate}
	7. a) La url no es correcta.
	\begin{enumerate}
		\item El sistema muestra un mensaje de error.
		\item Paso 6.
	\end{enumerate}
	*a) En cualquier momento, el usuario puede volver atrás.
	\item \textbf{Variaciones:} Ninguna.
	\item \textbf{No-funcional:} Ninguna.
	\item \textbf{Cuestiones:} Ninguna.
\end{itemize}

\subsection{Modificar vídeo}
\begin{itemize}
	\item \textbf{Descripción:} Modifica un vídeo.
	\item \textbf{Actores:} Creador-editor (usuario).
	\item \textbf{Precondiciones:} Debe existir el punto activo correspondiente, se debe estar editando la fantasía y debe existir un vídeo.
	\item \textbf{Postcondiciones:} El vídeo queda modificado.
	\item \textbf{Escenario principal:}
	\begin{enumerate}
		\item El usuario selecciona el punto activo correspondiente dentro de la fantasía.
		\item El sistema abre una ventana emergente con la información del punto activo.
		\item Paso 4 de \hyperlink{crearvideo}{\textcolor{blue}{Crear vídeo}}.
	\end{enumerate}
	\item \textbf{Extensiones:} \\ *a) En cualquier momento, el usuario puede volver atrás.
	\item \textbf{Variaciones:} Ninguna.
	\item \textbf{No-funcional:} Ninguna.
	\item \textbf{Cuestiones:} Ninguna.
\end{itemize}

\subsection{Borrar vídeo}
\begin{itemize}
	\item \textbf{Descripción:} Borra un vídeo de un punto activo.
	\item \textbf{Actores:} Creador-editor (usuario).
	\item \textbf{Precondiciones:} Debe existir el punto activo correspondiente, se debe estar editando la fantasía y debe existir un vídeo.
	\item \textbf{Postcondiciones:} Borra el vídeo del punto activo.
	\item \textbf{Escenario principal:}
	\begin{enumerate}
		\item El usuario selecciona el punto activo correspondiente dentro de la fantasía.
		\item El sistema abre una ventana emergente con la información del punto activo.
		\item El usuario selecciona el vídeo y pulsa el botón ``Suprimir''.
		\item El sistema muestra un mensaje de confirmación.
		\item El usuario selecciona ``Aceptar''.
		\item El sistema borra el vídeo del punto activo.
	\end{enumerate}
	\item \textbf{Extensiones:} \\ 5. a) El usuario selecciona ``Cancelar''.
	\begin{enumerate}
		\item El sistema cierra la ventana emergente.
		\item Paso 1.
	\end{enumerate}
	*a) En cualquier momento, el usuario puede volver atrás.
	\item \textbf{Variaciones:} Ninguna.
	\item \textbf{No-funcional:} Ninguna.
	\item \textbf{Cuestiones:} Ninguna.
\end{itemize}

\section{CRUD texto}
\begin{itemize}
	\item \textbf{Descripción:} Inserta un texto en un punto activo.
	\item \textbf{Actores:} Creador-editor (usuario).
	\item \textbf{Precondiciones:} Debe existir el punto activo correspondiente y se debe estar editando la fantasía.
	\item \textbf{Postcondiciones:} Inserta un texto en el punto activo seleccionado.
	\item \textbf{Escenario principal:}
	\begin{enumerate}
		\item El usuario selecciona el punto activo al que le quiere añadir-editar el texto.
		\item El sistema muestra una ventana emergente con la información del punto activo.
		\item El usuario selecciona introduce el texto deseado en el campo ``Texto'' con las opciones de formato que desee. 
		\item El usuario pulsa en el botón ``Aceptar''.
		\item El sistema guarda el texto en el punto activo correspondiente.
	\end{enumerate}
	\item \textbf{Extensiones:} \\ *a) En cualquier momento, el usuario puede volver atrás.
	\item \textbf{Variaciones:} Ninguna.
	\item \textbf{No-funcional:} Ninguna.
	\item \textbf{Cuestiones:} Ninguna.
\end{itemize}

\section{CRUD \textit{quiz}}
\hypertarget{crearquiz}{}
\subsection{Crear \textit{quiz}}
\begin{itemize}
	\item \textbf{Descripción:} Crea un pequeño cuestionario sobre el tema del que trata el punto activo.
	\item \textbf{Actores:} Creador-editor (usuario).
	\item \textbf{Precondiciones:} Debe existir el punto activo correspondiente y se debe estar editando la fantasía.
	\item \textbf{Postcondiciones:} Crea un pequeño cuestionario en relación al punto activo correspondiente.
	\item \textbf{Escenario principal:}
	\begin{enumerate}
		\item El usuario selecciona el punto activo correspondiente.
		\item El sistema muestra una ventana emergente con la información del punto activo.
		\item El usuario selecciona la opción de ``Crear \textit{quiz}''.
		\item El sistema muestra las posibles opciones.
		\item El usuario selecciona ``Respuesta simple''.
		\item El sistema muestra una ventana emergente para crear la pregunta con sus posibles respuestas.
		\item El usuario rellena la ventana emergente con la pregunta y las respuestas convenientes y pulsa ``Aceptar'' cuando termina.
		\item El sistema cierra la ventana emergente.
		\item El cuestionario queda registrado en el punto activo seleccionado.
	\end{enumerate}
	\item \textbf{Extensiones:} \\3. a) El usuario elige la opción ``Palabra''.
	\begin{enumerate}
		\item El sistema abre una ventana emergente para crear la pregunta y su respuesta.
		\item El usuario rellena la ventana emergente con la pregunta y la respuesta conveniente y pulsa ``Aceptar'' cuando termina.
		\item Paso 8.
	\end{enumerate}
	3. b) El usuario elige la opción ``Quiz con imágenes''.
	\begin{enumerate}
		\item El sistema abre una ventana emergente para crear la pregunta con la imagen y su respuesta.
		\item El usuario rellena la ventana emergente con la pregunta, la imagen y la respuesta conveniente, y pulsa ``Aceptar'' cuando termina.
		\item Paso 8.
	\end{enumerate}
	3. c) El usuario elige la opción ``Unir''.
	\begin{enumerate}
		\item El sistema abre una ventana emergente para crear el quiz de unión.
		\item El usuario rellena la ventana emergente con las posibles respuestas y su respuesta correcta y pulsa ``Aceptar'' cuando termina.
		\item Paso 8.
	\end{enumerate}
	*a) En cualquier momento, el usuario puede volver atrás.
	\item \textbf{Variaciones:} Ninguna.
	\item \textbf{No-funcional:} Ninguna.
	\item \textbf{Cuestiones:} Ninguna.
\end{itemize}

\subsection{Visualizar \textit{quiz}}
\begin{itemize}
	\item \textbf{Descripción:} Muestra el estado del \textit{quiz}.
	\item \textbf{Actores:} Creador-editor (usuario).
	\item \textbf{Precondiciones:} Debe existir el punto activo correspondiente, se debe estar editando la fantasía y debe existir un \textit{quiz}.
	\item \textbf{Postcondiciones:} Muestra el estado del \textit{quiz} en el punto activo correspondiente.
	\item \textbf{Escenario principal:}
	\begin{enumerate}
		\item El usuario selecciona el punto activo correspondiente.
		\item El sistema muestra una ventana emergente con la información del punto activo.
		\item El usuario selecciona la opción de ``Leer \textit{quiz}''.
		\item El sistema muestra una ventana emergente con la visión final del \textit{quiz}.
	\end{enumerate}
	\item \textbf{Extensiones:} \\ *a) En cualquier momento, el usuario puede volver atrás.
	\item \textbf{Variaciones:} Ninguna.
	\item \textbf{No-funcional:} Ninguna.
	\item \textbf{Cuestiones:} Ninguna.
\end{itemize}

\subsection{Modificar \textit{quiz}}
\begin{itemize}
	\item \textbf{Descripción:} Permite modificar el \textit{quiz}.
	\item \textbf{Actores:} Creador-editor (usuario).
	\item \textbf{Precondiciones:} Debe existir el punto activo correspondiente, se debe estar editando la fantasía y debe existir un \textit{quiz}.
	\item \textbf{Postcondiciones:} Modifica el \textit{quiz} de un punto activo.
	\item \textbf{Escenario principal:}
	\begin{enumerate}
		\item El usuario selecciona el punto activo correspondiente.
		\item El sistema muestra una ventana emergente con la información del punto activo.
		\item El usuario selecciona la opción ``Modificar \textit{quiz}''.
		\item Paso 4 de \hyperlink{crearquiz}{\textcolor{blue}{Crear \textit{Quiz}}}
	\end{enumerate}
	\item \textbf{Extensiones:} \\ *a) En cualquier momento, el usuario puede volver atrás.
	\item \textbf{Variaciones:} Ninguna.
	\item \textbf{No-funcional:} Ninguna.
	\item \textbf{Cuestiones:} Ninguna.
\end{itemize}

\subsection{Borrar \textit{quiz}}
\begin{itemize}
	\item \textbf{Descripción:} Borra el \textit{quiz} del punto activo seleccionado.
	\item \textbf{Actores:} Creador-editor (usuario).
	\item \textbf{Precondiciones:} Debe existir el punto activo correspondiente, se debe estar editando la fantasía y debe existir un \textit{quiz}.
	\item \textbf{Postcondiciones:} Borra el \textit{quiz} del punto activo seleccionado.
	\item \textbf{Escenario principal:}
	\begin{enumerate}
		\item El usuario selecciona el punto activo correspondiente.
		\item El sistema muestra una ventana emergente con la información del punto activo.
		\item El usuario selecciona la opción de ``Borrar \textit{quiz}''.
		\item El sistema muestra un mensaje de confirmación.
		\item El usuario selecciona ``Aceptar''.
		\item El sistema borra el \textit{quiz} del punto activo.
	\end{enumerate}
	\item \textbf{Extensiones:} \\ 5. a) El usuario selecciona ``Cancelar''.
	\begin{enumerate}
		\item El sistema cierra la ventana emergente.
		\item Paso 1.
	\end{enumerate}
	*a) En cualquier momento, el usuario puede volver atrás.
	\item \textbf{Variaciones:} Ninguna.
	\item \textbf{No-funcional:} Ninguna.
	\item \textbf{Cuestiones:} Ninguna.
\end{itemize}

\section{CRUD efecto de audio}
\hypertarget{crearaudio}{}
\subsection{Crear efecto de audio}
\begin{itemize}
	\item \textbf{Descripción:} Establece un efecto de audio de fondo en el punto activo.
	\item \textbf{Actores:} Creador-editor (usuario).
	\item \textbf{Precondiciones:} Debe existir el punto activo correspondiente y se debe estar editando la fantasía.
	\item \textbf{Postcondiciones:} Establece el efecto de audio de fondo.
	\item \textbf{Escenario principal:}
	\begin{enumerate}
		\item El usuario selecciona el punto activo correspondiente.
		\item El sistema muestra una ventana emergente con la información del punto activo.
		\item El usuario selecciona la opción de ``Añadir efecto de audio''.
		\item El sistema muestra una ventana emergente en la que da a elegir al usuario de donde quiere seleccionar el audio (Internet, local, audio ya usado en la fantasía).
		\item EL usuario elige la opción ``Internet'' para incluir un audio de Internet.
		\item El sistema le pide al usuario la url del audio.
		\item El usuario inserta la url del audio.
		\item El sistema guarda el audio en el punto activo.
	\end{enumerate}
	\item \textbf{Extensiones:} \\ 5. a) El usuario elige la opción ``Local'' para incluir un audio desde su ordenador.
	\begin{enumerate}
		\item El sistema abre una ventana del explorador de archivos.
		\item El usuario selecciona el audio deseado y pulsa ``Aceptar''.
		\item El sistema cierra la ventana del explorador de archivos.
		\item Paso 8.
	\end{enumerate}
	5. b) El usuario elige la opción ``Audio usado anteriormente'' para incluir un audio ya usado.
	\begin{enumerate}
		\item El sistema abre una ventana con los audios usados anteriormente.
		\item El usuario selecciona el audio deseado y pulsa aceptar.
		\item El sistema cierra la ventana emergente.
		\item Paso 8.
	\end{enumerate}
	7. a) La url no es correcta.
	\begin{enumerate}
		\item El sistema muestra un mensaje de error.
		\item Paso 6.
	\end{enumerate}
	*a) En cualquier momento, el usuario puede volver atrás.
	\item \textbf{Variaciones:} Ninguna.
	\item \textbf{No-funcional:} Ninguna.
	\item \textbf{Cuestiones:} Ninguna.
\end{itemize}

\subsection{Modificar efecto de audio}
\begin{itemize}
	\item \textbf{Descripción:} Modificar efecto de audio.
	\item \textbf{Actores:} Creador-editor (usuario).
	\item \textbf{Precondiciones:} Debe existir el punto activo correspondiente, se debe estar editando la fantasía y debe existir un audio.
	\item \textbf{Postcondiciones:} Modifica el audio.
	\item \textbf{Escenario principal:}
	\begin{enumerate}
		\item El usuario selecciona el punto activo.
		\item El sistema abre una ventana emergente con la información del punto activo.
		\item Paso 4 de \hyperlink{crearaudio}{\textcolor{blue}{Crear audio}}
	\end{enumerate}
	\item \textbf{Extensiones:} \\ *a) En cualquier momento, el usuario puede volver atrás.
	\item \textbf{Variaciones:} Ninguna.
	\item \textbf{No-funcional:} Ninguna.
	\item \textbf{Cuestiones:} Ninguna.
\end{itemize}

\subsection{Borrar efecto de audio}
\begin{itemize}
	\item \textbf{Descripción:} Borra un efecto de audio de un punto activo.
	\item \textbf{Actores:} Creador-editor (usuario).
	\item \textbf{Precondiciones:} Debe existir el punto activo correspondiente, se debe estar editando la fantasía y debe existir un audio.
	\item \textbf{Postcondiciones:} Borra un efecto de audio de un punto activo.
	\item \textbf{Escenario principal:}
	\begin{enumerate}
		\item El usuario selecciona el punto activo.
		\item El sistema abre una ventana emergente con la información del punto activo.
		\item El usuario selecciona el audio y pulsa el botón ``Suprimir''.
		\item El sistema muestra un mensaje de confirmación.
		\item El usuario selecciona ``Aceptar''.
		\item El sistema borra el audio del punto activo.
	\end{enumerate}
	\item \textbf{Extensiones:} \\ 5. a) El usuario selecciona ``Cancelar''.
	\begin{enumerate}
		\item El sistema cierra la ventana emergente.
		\item Paso 1.
	\end{enumerate}
	*a) En cualquier momento, el usuario puede volver atrás.
	\item \textbf{Variaciones:} Ninguna.
	\item \textbf{No-funcional:} Ninguna.
	\item \textbf{Cuestiones:} Ninguna.
\end{itemize}


\section{CRUD información adicional}
\begin{itemize}
	\item \textbf{Descripción:} Inserta un texto como información adicional de la fantasía.
	\item \textbf{Actores:} Creador-editor (usuario).
	\item \textbf{Precondiciones:} Se debe estar editando la fantasía correspondiente.
	\item \textbf{Postcondiciones:} Inserta un texto como información adicional de la fantasía.
	\item \textbf{Escenario principal:}
	\begin{enumerate}
		\item El usuario selecciona la opción ``Información adicional''.
		\item El sistema muestra una ventana emergente con un cuadro de texto.
		\item El usuario introduce el texto deseado en el cuadro de texto con las opciones de formato que desee. 
		\item El usuario pulsa en el botón ``Aceptar''.
		\item El sistema guarda la información adicional en la fantasía correspondiente.
	\end{enumerate}
	\item \textbf{Extensiones:} \\ *a) En cualquier momento, el usuario puede volver atrás.
	\item \textbf{Variaciones:} Ninguna.
	\item \textbf{No-funcional:} Ninguna.
	\item \textbf{Cuestiones:} Ninguna.
\end{itemize}

\section{Organizar puntos activos}
\begin{itemize}
	\item \textbf{Descripción:} Organiza la aparición de los puntos activos.
	\item \textbf{Actores:} Creador-editor (usuario).
	\item \textbf{Precondiciones:} La fantasía debe estar creada.
	\item \textbf{Postcondiciones:} Establece el orden de aparición de los puntos activos de la fantasía.
	\item \textbf{Escenario principal:}
	\begin{enumerate}
		\item El usuario selecciona la fantasía correspondiente.
		\item El sistema muestra una ventana con la información de la fantasía y las opciones disponibles.
		\item El usuario selecciona la opción ``Organizar puntos activos''.
		\item El sistema muestra una ventana emergente con el nombre de los puntos activos existentes en la fantasía y un recuadro para establecer el orden de aparición.
		\item El usuario establece el orden de aparición en los recuadros junto al nombre de los puntos activos de la fantasía.
		\item El usuario pulsa ``Aceptar'' para guardar los cambios realizados.
		\item El sistema guarda el orden de aparición de los puntos activos.
	\end{enumerate}
	\item \textbf{Extensiones:} \\ *a) En cualquier momento, el usuario puede volver atrás.
	\item \textbf{Variaciones:} Ninguna.
	\item \textbf{No-funcional:} Ninguna.
	\item \textbf{Cuestiones:} Ninguna.
\end{itemize}

\section{CRUD \textit{quiz} final}
\hypertarget{crearquizfinal}{}
\subsection{Crear \textit{quiz} final}
\begin{itemize}
	\item \textbf{Descripción:} Crea un cuestionario sobre el tema del que trata la fantasía.
	\item \textbf{Actores:} Creador-editor (usuario).
	\item \textbf{Precondiciones:} Debe existir y se debe estar editando la fantasía.
	\item \textbf{Postcondiciones:} Crea un cuestionario en relación al tema de la fantasía..
	\item \textbf{Escenario principal:}
	\begin{enumerate}
		\item El usuario pulsa el botón de ``Cuestionario final''.
		\item El sistema muestra las posibles opciones.
		\item El usuario selecciona ``Crear nuevo quiz''.
		\item El sistema muestras las posibles opciones de creación.
		\item El usuario selecciona ``Respuesta simple''.
		\item El sistema muestra una ventana emergente para crear la pregunta con sus posibles respuestas.
		\item El usuario rellena la ventana emergente con la pregunta y las respuestas convenientes y pulsa ``Aceptar'' cuando termina.
		\item El sistema cierra la ventana emergente.
		\item El cuestionario queda registrado en la fantasía.
	\end{enumerate}
	\item \textbf{Extensiones:} \\5. a) El usuario elige la opción ``Palabra''.
	\begin{enumerate}
		\item El sistema abre una ventana emergente para crear la pregunta y su respuesta.
		\item El usuario rellena la ventana emergente con la pregunta y la respuesta conveniente y pulsa ``Aceptar'' cuando termina.
		\item Paso 8.
	\end{enumerate}
	5. b) El usuario elige la opción ``Quiz con imágenes''.
	\begin{enumerate}
		\item El sistema abre una ventana emergente para crear la pregunta con la imagen y su respuesta.
		\item El usuario rellena la ventana emergente con la pregunta, la imagen y la respuesta conveniente, y pulsa ``Aceptar'' cuando termina.
		\item Paso 8.
	\end{enumerate}
	5. c) El usuario elige la opción ``Unir''.
	\begin{enumerate}
		\item El sistema abre una ventana emergente para crear el quiz de unión.
		\item El usuario rellena la ventana emergente con las posibles respuestas y su respuesta correcta y pulsa ``Aceptar'' cuando termina.
		\item Paso 8.
	\end{enumerate}
	*a) En cualquier momento, el usuario puede volver atrás.
	\item \textbf{Variaciones:} Ninguna.
	\item \textbf{No-funcional:} Ninguna.
	\item \textbf{Cuestiones:} Ninguna.
\end{itemize}

\subsection{Visualizar \textit{quiz} final}
\begin{itemize}
	\item \textbf{Descripción:} Muestra el estado del \textit{quiz}.
	\item \textbf{Actores:} Creador-editor (usuario).
	\item \textbf{Precondiciones:} Debe existir y se debe estar editando la fantasía, y debe existir el \textit{quiz} final.
	\item \textbf{Postcondiciones:} Muestra el estado del \textit{quiz} final.
	\item \textbf{Escenario principal:}
	\begin{enumerate}
		\item El usuario selecciona el botón de ``Cuestionario final''.
		\item El sistema muestra las posibles opciones.
		\item El usuario selecciona la opción de ``Leer \textit{quiz} final''.
		\item El sistema muestra una ventana emergente con la visión final del \textit{quiz}.
	\end{enumerate}
	\item \textbf{Extensiones:} \\ *a) En cualquier momento, el usuario puede volver atrás.
	\item \textbf{Variaciones:} Ninguna.
	\item \textbf{No-funcional:} Ninguna.
	\item \textbf{Cuestiones:} Ninguna.
\end{itemize}

\subsection{Modificar \textit{quiz} final}
\begin{itemize}
	\item \textbf{Descripción:} Permite modificar el \textit{quiz} final.
	\item \textbf{Actores:} Creador-editor (usuario).
	\item \textbf{Precondiciones:} Debe existir y se debe estar editando la fantasía, y debe existir el \textit{quiz} final.
	\item \textbf{Postcondiciones:} Modifica el \textit{quiz} final de la fantasía.
	\item \textbf{Escenario principal:}
	\begin{enumerate}
		\item El usuario pulsa el botón de "Cuestionario final".
		\item El sistema muestra las posibles opciones.
		\item El usuario selecciona la opción ``Modificar \textit{quiz} final''.
		\item Caso de uso \hyperlink{crearquizfinal}{\textcolor{blue}{Crear \textit{Quiz} final}}
	\end{enumerate}
	\item \textbf{Extensiones:} \\ *a) En cualquier momento, el usuario puede volver atrás.
	\item \textbf{Variaciones:} Ninguna.
	\item \textbf{No-funcional:} Ninguna.
	\item \textbf{Cuestiones:} Ninguna.
\end{itemize}

\subsection{Borrar \textit{quiz} final}
\begin{itemize}
	\item \textbf{Descripción:} Borra el \textit{quiz} final de la fantasía.
	\item \textbf{Actores:} Creador-editor (usuario).
	\item \textbf{Precondiciones:} Debe existir y se debe estar editando la fantasía, y debe existir el \textit{quiz} final.
	\item \textbf{Postcondiciones:} Borra el \textit{quiz} final de la fantasía.
	\item \textbf{Escenario principal:}
	\begin{enumerate}
		\item El usuario pulsa el botón de ``Cuestionario final''.
		\item El sistema muestra las posibles opciones.
		\item El usuario selecciona la opción de ``Borrar \textit{quiz}'' final.
		\item El sistema muestra un mensaje de confirmación.
		\item El usuario selecciona ``Aceptar''.
		\item El sistema borra el \textit{quiz} final de la fantasía.
	\end{enumerate}
	\item \textbf{Extensiones:} \\ 5. a) El usuario selecciona ``Cancelar''.
	\begin{enumerate}
		\item El sistema cierra la ventana emergente.
		\item Paso 1.
	\end{enumerate}
	*a) En cualquier momento, el usuario puede volver atrás.
	\item \textbf{Variaciones:} Ninguna.
	\item \textbf{No-funcional:} Ninguna.
	\item \textbf{Cuestiones:} Ninguna.
\end{itemize}


%\section{Gestionar porcentaje de un punto activo}
%\begin{itemize}
%	\item \textbf{Descripción:}
%	\item \textbf{Actores:}
%	\item \textbf{Precondiciones:}
%	\item \textbf{Postcondiciones:}
%	\item \textbf{Escenario principal:}
%	\item \textbf{Extensiones:}
%	\item \textbf{Variaciones:}
%	\item \textbf{No-funcional:}
%	\item \textbf{Cuestiones:}
%\end{itemize}
%
%\section{Gestionar ficha alumno}
%\begin{itemize}
%	\item \textbf{Descripción:}
%	\item \textbf{Actores:}
%	\item \textbf{Precondiciones:}
%	\item \textbf{Postcondiciones:}
%	\item \textbf{Escenario principal:}
%	\item \textbf{Extensiones:}
%	\item \textbf{Variaciones:}
%	\item \textbf{No-funcional:}
%	\item \textbf{Cuestiones:}
%\end{itemize}
%
%\section{Asignar nota final}
%\begin{itemize}
%	\item \textbf{Descripción:}
%	\item \textbf{Actores:}
%	\item \textbf{Precondiciones:}
%	\item \textbf{Postcondiciones:}
%	\item \textbf{Escenario principal:}
%	\item \textbf{Extensiones:}
%	\item \textbf{Variaciones:}
%	\item \textbf{No-funcional:}
%	\item \textbf{Cuestiones:}
%\end{itemize}
%
%\section{Asignar fantasía}
%\begin{itemize}
%	\item \textbf{Descripción:}
%	\item \textbf{Actores:}
%	\item \textbf{Precondiciones:}
%	\item \textbf{Postcondiciones:}
%	\item \textbf{Escenario principal:}
%	\item \textbf{Extensiones:}
%	\item \textbf{Variaciones:}
%	\item \textbf{No-funcional:}
%	\item \textbf{Cuestiones:}
%\end{itemize}