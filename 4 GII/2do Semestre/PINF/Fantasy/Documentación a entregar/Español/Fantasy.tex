%%\documentclass[a4paper,12pt,oneside]{llncs}
\documentclass{book}
%\usepackage[right=2cm,left=3cm,top=2cm,bottom=2cm,headsep=0cm]{geometry}

%%%%%%%%%%%%%%%%%%%%%%%%%%%%%%%%%%%%%%%%%%%%%%%%%%%%%%%%%%%
%% Juego de caracteres usado en el archivo fuente: UTF-8
\usepackage{ucs}
\usepackage[utf8x]{inputenc}
\usepackage{eurosym}

%%%%%%%%%%%%%%%%%%%%%%%%%%%%%%%%%%%%%%%%%%%%%%%%%%%%%%%%%%%
%% Juego de caracteres usado en la salida dvi
%% Otra posibilidad: \usepackage{t1enc}
\usepackage[T1]{fontenc}

%%%%%%%%%%%%%%%%%%%%%%%%%%%%%%%%%%%%%%%%%%%%%%%%%%%%%%%%%%%
%% Ajusta maergenes para a4
\usepackage{a4wide}

%%%%%%%%%%%%%%%%%%%%%%%%%%%%%%%%%%%%%%%%%%%%%%%%%%%%%%%%%%%
%% Uso fuente postscript times, para que los ps y pdf queden y pequeños...
\usepackage{times}

%%%%%%%%%%%%%%%%%%%%%%%%%%%%%%%%%%%%%%%%%%%%%%%%%%%%%%%%%%%
%% Posibilidad de hipertexto (especialmente en pdf)
\usepackage{hyperref}

%%%%%%%%%%%%%%%%%%%%%%%%%%%%%%%%%%%%%%%%%%%%%%%%%%%%%%%%%%%
%% Graficos 
\usepackage{graphics,graphicx}

%%%%%%%%%%%%%%%%%%%%%%%%%%%%%%%%%%%%%%%%%%%%%%%%%%%%%%%%%%%
%% Ciertos caracteres "raros"...
\usepackage{latexsym}

%%%%%%%%%%%%%%%%%%%%%%%%%%%%%%%%%%%%%%%%%%%%%%%%%%%%%%%%%%%
%% Matematicas aun más fuertes (american math dociety)
\usepackage{amsmath}

%%%%%%%%%%%%%%%%%%%%%%%%%%%%%%%%%%%%%%%%%%%%%%%%%%%%%%%%%%%
\usepackage{multirow} % para las tablas
\usepackage[spanish,es-tabla]{babel}

%%%%%%%%%%%%%%%%%%%%%%%%%%%%%%%%%%%%%%%%%%%%%%%%%%%%%%%%%%%
%% Fuentes matematicas lo mas compatibles posibles con postscript (times)
%% (Esto no funciona para todos los simbolos pero reduce mucho el tamaño del
%% pdf si hay muchas matamaticas....
\usepackage{mathptm}

%%% VARIOS:
%\usepackage{slashbox}
\usepackage{verbatim}
\usepackage{array}
\usepackage{listings}
\usepackage{multirow}
\usepackage{hhline}
\usepackage{titling}

%% MARCA DE AGUA
%% Este package de "draft copy" NO funciona con pdflatex
%%\usepackage{draftcopy}
%% Este package de "draft copy" SI funciona con pdflatex
%%%\usepackage{pdfdraftcopy}
%%%%%%%%%%%%%%%%%%%%%%%%%%%%%%%%%%%%%%%%%%%%%%%%%%%%%%%%%%%
%% Indenteacion en español...
\usepackage[spanish]{babel}
\usepackage{Estilos/Apuntes}
\usepackage[svgnames,x11names,table]{xcolor}
\usepackage{listings}
% Para escribir código en C
% \begin{verbatim}[language=C]
% #include <stdio.h>
% int main(int argc, char* argv[]) {
% puts("Hola mundo!");
% }
% \end{verbatim}


\title{\huge{Fantasy}}
\author{Luis Gutiérrez Flores\\
	Nicolás Ruiz Requejo\\
	Jesús Rodríguez Heras\\
	Arantzazu Otal Alberro\\
	Alejandro Segovia Gallardo\\
	Alejandro José Caraballo García\\
	Gabriel Fernando Sánchez Reina}


%%Configuracion del paquete listings
\lstset{language=bash, numbers=left, numberstyle=\tiny, numbersep=10pt, firstnumber=1, stepnumber=1}

\begin{document}
%	\maketitle

	\begin{titlepage}
		\centering
		
		{\scshape\LARGE Escuela Superior de Ingeniería \par}
		\vspace{1cm}
		{\scshape\Large Universidad de Cádiz\par}
		\vspace{1.5cm}
		{\huge\bfseries Fantasy\par}
		\vspace{1cm}
		{\Large\itshape Luis Gutiérrez Flores\\
			Nicolás Ruiz Requejo\\
			Jesús Rodríguez Heras\\
			Arantzazu Otal Alberro\\
			Alejandro Segovia Gallardo\\
			Alejandro José Caraballo García\\
			Gabriel Fernando Sánchez Reina\par}
		\vspace{1cm}
		\begin{table}[htb]
			\centering
			\begin{tabular}{ccc}
				\includegraphics[width=0.2\textwidth]{UCA.png}\par\vspace{1.2cm} & \includegraphics[width=0.2\textwidth]{ESI.png}\par\vspace{1.2cm} & \includegraphics[width=0.2\textwidth]{Stimey.png}\par\vspace{1.2cm}
			\end{tabular}
		\end{table}
%		\vfill
		\large{Stimey}\par
		
		
		% Bottom of the page
%		{\large \today\par}
	\end{titlepage}
	
%	\thispagestyle{empty}
	\newpage
	
	\tableofcontents
	\newpage
	
	\include{Requisitos}
	\include{Casos}
%	\chapter{Plano de cableado vertical}
\section{Distribuidores etiquetados. Dentro de cada distribuidor detallar gráficamente los dispositivos instalados}
\begin{center}
	\includegraphics[scale=0.65]{CV.png}
\end{center}
\newpage
Tal como podemos ver en la imagen, el Rack 0 se encontraría en la planta baja y contaría con los siguientes dispositivos:
\begin{itemize}
	\item Una UPS (sistema de alimentación ininterrumpida = SAI).
	\item Cuatro switches con tecnología PoE para suministro eléctrico en caso de ser necesario en teléfonos VoIP o puntos de acceso. Estos switches también cuentan con puertos de fibra óptica.
	\item Cuatro patch panels.
	\item Una centralita telefónica.
	\item Un router.
	\item Un switch de fibra óptica exclusivo para cableado vertical.
\end{itemize}
El Rack 0 (planta baja) está conectado con el Rack 1 (planta alta) mediante fibra óptica multimodo. El Rack 1 cuenta con los siguientes dispositivos:
\begin{itemize}
	\item Una UPS (sistema de alimentación ininterrumpida = SAI).
	\item Tres switches con tecnología PoE para suministro eléctrico en caso de ser necesario en teléfonos VoIP o puntos de acceso. Estos switches también cuentan con puertos de fibra óptica.
	\item Tres patch panels.
	\item Un switch de fibra óptica exclusivo para cableado vertical.
\end{itemize}
%	%Doc4
\chapter{Presupuesto}
\begin{table}[htbp]
	\begin{center}
		\begin{tabular}{|c|c|c|}
			\hline 
			\textbf{Componente} & \textbf{unidades} & \textbf{Precio total (\euro)}  \\
			\hline
			Arduino Leonardo & 1 & 18 \\\hline
			Placa Shield PCB & 1 & 25 \\\hline
			Sensor CNY70 & 3 & 5.10 \\\hline
			Sensor HC-SR04 & 1 & 1.49 \\\hline
			Cables & Varios & 1.20 \\\hline
			Módulo bluetooth GW040 & 1 & 7.73 \\\hline
			Motores Micro Metal & 2 & 27.80 \\\hline
			Sensor SHARP & 2 & 21.58 \\\hline
			Plástico & 20g & 1.46 \\\hline
			Pilas & 6 & 7.33 \\\hline
			Horas & 180 & 20€/hora \\\hline
		\end{tabular}
		\caption{Presupuesto del proyecto.}
		\label{tabla:Presupuesto del proyecto}
	\end{center}
\end{table}

El presupuesto total ascendería a 3716,69\euro.
%	\chapter{Justificaciones}
\section{Cableado horizontal}
Los Rack de ambas plantas se han colocado en las salas informáticas ya habilitadas, las cuales tienen espacio suficiente para ellos y son también una localización lógica, por razones como seguridad tanto contra intrusiones(ya que solo el personal autorizado tendrá acceso a esta salas) como ambientales (se puede mantener la temperatura adecuada para ellos sin afectar al resto del personal). Están además a suficiente distancia de fuentes de interferencias como pueden ser los ascensores. Desde estas salas se cumplen también los criterios de distancia máxima para el entramado de conexión horizontal del estándar IEEE 802.3

Se han elegido cables categoría 7a, permitiendo la implantación de una red 10 Gigabit Ethernet de hasta 10Gb/s. La red de cableado se ha distribuido por el falso techo, de forma que no supongan ningún estorbo para los trabajadores y permitiendo una distribución más flexible, que podría cambiarse fácilmente si se diera la necesidad.

Las rosetas de conexión constan de 4 puertos (3 de datos y uno de voz) de forma que se pueda dedicar uno para dispositivos como teléfonos (pudiéndose usar uno de datos si el de voz fallara) y permitiendo la conexión simultánea de 3 dispositivos que necesiten acceder a la red, o pudiendo mantener conectado el dispositivo más vital en caso de fallos de algunos de los puertos, hasta que estos se arreglaran.

Hay colocados varios puntos de acceso inalámbricos en zonas donde puedan ser necesarios donde por lo general prevalecen dispositivos como móviles o tabletas. Entre ellas, salas amplias donde la cercanía de conexiones físicas puede ser un problema, salas alejadas de puestos de trabajo donde no se necesita gran velocidad de transferencia de datos, o salas de reuniones donde tampoco se necesite gran velocidad de transferencia y se quiera evitar el uso de cables por comodidad.

El etiquetado usado en el cableado horizontal ya ha sido explicado con anterioridad en el capítulo 1 de este documento.
\section{Distribuidores}
Hemos elegido los siguientes elementos para nuestro Racks debido a varias causas. Empezaremos describiendo el Rack de la planta baja, etiquetado como ``Rack 0'' y luego el Rack de la planta alta, etiquetado como ``Rack 1''.
\subsection{RACK 0}
En este rack, hemos incorporado cinco switches CISCO Catalyst 2960S para poder abarcar todos los paneles de datos que se encuentran en dicha planta. El quinto switch queda para la conexión entre el Rack 0 y el Rack 1 mediante fibra óptica. Sus 48 puertos restantes quedan por si queremos realizar una ampliación del cableado en un futuro.

Adicionalmente, dichos switches tienen PoE (Power Over Ethernet); de forma que los mencionados switches pueden proveer corriente eléctrica mediante los cables de la red Ethernet.

El etiquetado para cada switch será de tipo \textbf{SW-X-R0}; donde ‘X’ corresponde al número del switch. A su vez cada swtich irá etiquetado con los paneles de datos correspondientes a los que proporciona red. Los switches irán conectados entre sí mediante fibra óptica multimodo.

El router (CISCO 2811), etiquetado como RE también cuenta con tecnología PoE y cuenta con dos conexiones de fibra óptica para ser conectado al primer switch y al ISP.

Los paneles de parcheo correspondientes a datos son cuatro; de forma que son equivalentes al número de switches usados por los hosts; y por lo tanto también son iguales el número de puertos, cuarenta y ocho. El etiquetado será de tipo \textbf{R0-DXXX – R0-DYYY}; siendo ‘XXX’ el puerto inicial e YYY el puerto final correspondiente a cada panel de parcheo.

En cuanto a la voz, contamos con dos paneles de parcheo de voz para abarcar todos los dispositivos de teléfono VoIP que se encuentran en la planta baja.

Para darle señal telefónica a ambas plantas contamos con una centralita telefónica Nexspan A5000 2XS; que cuenta con 112 puertos, que son pocas más de las que necesitamos para que todos nuestros teléfonos VoIP funcionen correctamente.

Contamos con un SAI Lapara Serie On-Rack (UPS) para prevenir perder datos si la corriente eléctrica falla (y con ello que el Rack se apague). De esta forma, tendremos un plazo de tiempo para guardar todos nuestros datos y/o  configuraciones de la red.

\subsection{RACK 1}
En el Rack 1, debemos tener en cuenta que el etiquetado funciona de la misma forma que en el Rack 0, y que los modelos usados son los mismos.  Con esto:
\begin{itemize}
	\item Contamos con cuatro switches en lugar de cinco. Todos están conectados en serie al router del Rack 0 de planta baja.
	\item El número de paneles de parcheo de datos corresponde al de número de switches que hay en el Rack 1 usados por los hosts, por lo tanto, hay tres.
	\item Contamos con un panel de parcheo de voz para abarcar todos los teléfonos VoIP que hay en planta.
\end{itemize}

\section{Cableado vertical}
Para la conexión entre los distribuidores de ambas plantas utilizamos fibra óptica multimodo, que es ideal para la comunicación en distancias cortas como las de un campus, o el edificio que nos ocupa, con ello disfrutamos de las ventajas que ofrece la fibra sobre el cableado de cobre, como son una mayor velocidad de transmisión, mayor ancho de banda, inmunidad a interferencias, menor peso y tamaño…

Usamos concretamente cableado OM4, que permite el uso del estándar 10 Gigabit Ethernet hasta una distancia máxima aproximada de 500 metros y llega a soportar conexiones de hasta 125 metros con ratios de 40 a 100 Gbps.\\
En cuanto a presupuesto, la fibra es una opción rentable, con un precio no mucho mayor por metro que la mayoría de cables UTP de categoría 5 o 6.

Los switches de cada rack están conectados entre ellos formando una topología en anillo, lo cual proporciona redundancia para recuperación de fallos (failover).

\section{Plano de conexión}
Tal como se muestra en la imagen del plano de conexión tenemos un cable categoría 7a que va desde el patch panel hasta la roseta de pared del puesto de trabajo.

En dicho puesto, contamos con cuatro conexiones rj45, tres de datos y una de voz, donde usamos el etiquetado que comentamos en el apartado del cableado horizontal. El cableado que va desde la roseta hasta el host es también categoría 7a.

La decisión de contar con tres tomas de datos en el puesto de trabajo surge de la posibilidad de la existencia de una avería en cualquiera de las tomas que se estén usando actualmente o para la ampliación de dicho puesto de trabajo.
%	%Doc6
\chapter{Diseño}
\section{Estructura}
\subsection{Programa principal}
Para el desarrollo de los códigos, seguimos una estructura tal que, a partir del archivo \hyperlink{laberinto}{Robot.ino}, se acceda al resto de librerías que controlarán el robot.

En la librería \hyperlink{Robot}{Robot.hpp} se encuentra todo lo necesario para definir los pines y las funciones necesarias para el control de los sensores y motores.

\subsection{Celdas}
Para el mapeado de las celdas utilizamos la librería \hyperlink{Celda}{celda.hpp}.

\subsection{Sensores}
Para el control de los sensores, utilizamos una librería por cada tipo de sensor, de tal forma que tenemos tres librerías: \hyperlink{sUltrasonidos}{sUltrasonidos.hpp}, \hyperlink{sInfrarrojos}{sInfrarrojos.hpp} y \hyperlink{sCNY}{sCNY.hpp}.

\subsection{Actuadores}
Para el control de los motores hemos usado la librería \hyperlink{Motor}{motor.hpp}.

\section{Plan de pruebas}
%Pones el bicho y miras con el serial blueooth lo que va midiendo el nota.
\subsection{Primeras pruebas}
Al principio comenzamos con las \hyperlink{primerasPruebas}{pruebas individuales} de cada sensor, de los motores y de la conexión bluetooth usando los códigos de las prácticas de la asignatura para comprobar su correcto funcionamiento.

\subsection{Pruebas finales}
Una vez que los sensores estuvieron calibrados, pasamos a hacer \hyperlink{pruebasFinales}{pruebas en conjunto} con todos los sensores activos para ir corrigiendo los pequeños fallos que iban surgiendo.
%	%Doc7
\chapter{Implementación}
\section{Librerías}
\subsection{Propias}
Las \hypertarget{bibliotecas}{bibliotecas} usadas para dar soporte al código de Arduino son las siguientes:
\subsubsection{\texttt{Robot.hpp}}
\hypertarget{Robot}{}
\lstinputlisting[language=C]{Robot.hpp}
\subsubsection{\texttt{celda.hpp}}
\hypertarget{Celda}{}
\lstinputlisting[language=C]{celda.hpp}
\subsubsection{\texttt{sUltrasonidos.hpp}}
\hypertarget{sUltrasonidos}{}
\lstinputlisting[language=C]{sUltrasonidos.hpp}
\subsubsection{\texttt{sInfrarrojos.hpp}}
\hypertarget{sInfrarrojos}{}
\lstinputlisting[language=C]{sInfrarrojos.hpp}
\subsubsection{\texttt{sCNY.hpp}}
\hypertarget{sCNY}{}
\lstinputlisting[language=C]{sCNY.hpp}
\subsubsection{\texttt{motor.hpp}}
\hypertarget{Motor}{}
\lstinputlisting[language=C]{motor.hpp}

\subsection{Públicas}
%La que pilló Gabri por internet
\subsubsection{\texttt{StandardCplusplus}}
La librería \hypertarget{StandardCplusplus}{\texttt{StandardCplusplus.hpp}} la hemos descargado de Internet porque nos hacía falta para la implementación del TAD (tipo abstracto de dato) pila que nos ayudará en la vuelta a la casilla inicial.

\section{Apuntes sobre el código}
%Preguntar a Gabri
En la fase de salir del laberinto, entra en un bucle donde escanea la celda en a que se encuentra y luego va aplicando el algoritmo de la mano derecha hasta llegar a la casilla final.

Una vez que alcanza la meta deshace en orden inverso los movimientos que ha hecho, sin tener en cuenta los retrocesos, para volver a la casilla inicial por el camino más corto posible.
%	%Doc8
\chapter{Montaje}
\section{Chasis}
Para el diseño del chasis, hemos usado una impresora 3D. Las piezas fueron diseñadas en Sketchup 2017 y luego fueron impresas mediante una impresora Makerbot Replicator 2X.

\section{Cableado}
Para el cableado de los sensores hemos usado cables hembra-hembra que conectan los sensores con la placa shield, salvo los CNY70, que ya estaban conectados a la shield. Los motores, al igual que los CNY70, ya estaban conectados a la shield.
\begin{table}[htbp]
	\begin{center}
		\begin{tabular}{|c|c|c|}
			\hline 
			\textbf{Sensor/Actuador} & \textbf{Pin en shield} & \textbf{Pin en Arduino}  \\
			\hline
			Ultrasonidos & J23 & A3 \\\hline
			Infrarrojos izquierdo & J24 & A4 \\\hline
			Infrarrojos derecho & J8 & A8 \\\hline
			CNY70 izquierdo & J15 & A1 \\\hline
			CNY70 derecho & J19 & A0 \\\hline
			CNY70 trasero & J11 & A5 \\\hline
			Motor izquierdo & M21 y M22 & 10 y 9 \\\hline
			Motor derecho & M11 y M12 & 6 y 5 \\\hline
		\end{tabular}
		\caption{Conexión de pines.}
		\label{tabla:Conexión de pines}
	\end{center}
\end{table}

\section{Imágenes del proyecto}
Debido a que el diseño no es lo más imprescindible, hemos optado por un estilo minimalista y funcional antes que un diseño demasiado estético que pudiera dificultar incluso el trabajo del robot.
\begin{center}
	\includegraphics[scale=0.124]{F1.jpg}
	\includegraphics[scale=0.124]{F2.jpg}
	\includegraphics[scale=0.124]{F3.jpg}
	\includegraphics[scale=0.124]{F4.jpg}
\end{center}
%	%Doc9
\chapter{Pruebas}
Para las pruebas hemos usado el laberinto del laboratorio haciendo mediciones de los sensores y viendo las velocidades de ambos motores para corregir su trayectoria.

En las \hyperlink{primerasPruebas}{primeras pruebas} se probaron los motores, tanto el avance y retroceso, como los giros a izquierda y derecha. Para la prueba de los sensores usamos la antena bluetooth conectada al móvil mediante la aplicación serial bluetooth terminal por la que recibíamos las mediciones tomadas por los sensores.

Luego, en las \hyperlink{pruebasFinales}{pruebas finales}, pasamos a la interacción del robot con los sensores para probar el algoritmo de salida del laberinto.

Por último, solo faltó algunos arreglos en los parámetros del robot, para que no diera fallos en su trayectoria ni en la medición de los sensores.
%	%Doc10
\chapter{Mejoras}
Mejoras a tener en cuenta:
\begin{itemize}
	\item Implementación de un diferencial para coger las curvas mejor.
	\item Medición a tiempo real de las celdas y del recorrido completo para ver que hay en cada lado del robot por cada instante de tiempo.
\end{itemize}
	
\end{document}
