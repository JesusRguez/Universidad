\chapter{Pruebas del sistema}
\section{Pruebas unitarias}
Para las pruebas unitarias hemos usado la herramienta de pruebas aportada por Laravel, la cual, permite programar pruebas unitarias que se ejecutarán automáticamente y nos darán un resultado directo de si ha pasado dicha prueba o no.

\section{Pruebas de integración}
Las pruebas de integración han sido testeadas mediante toda la duración del proyecto mientras se iban enlazando los distintos módulos de puntos activos y workspace en la plataforma.

\section{Pruebas de sistema}
Las pruebas de sistema se han realizado gracias a la herramienta de pruebas aportada por Laravel, que, al igual que las pruebas unitarias, han sido programadas para luego ser lanzadas automáticamente.

\subsection{Pruebas funcionales}
\textcolor{red}{SUPONGO QUE AQUÍ HABRÁ QUE PONER ALGÚN TIPO DE PRUEBA}

\subsection{Pruebas no funcionales}
\textcolor{red}{SUPONGO QUE AQUÍ HABRÁ QUE PONER ALGÚN TIPO DE PRUEBA}

\section{Pruebas de aceptación}
Las pruebas de aceptación se han ido haciendo con nuestra clienta/directora de este proyecto en cada una de las reuniones, en las que nos iba argumentando los cambios que quería y si aceptaba los cambios que le proponíamos o los rechazaba.