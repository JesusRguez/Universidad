%% LyX 2.0.8.1 created this file.  For more info, see http://www.lyx.org/.
%% Do not edit unless you really know what you are doing.
\documentclass[12pt,spanish,intoc,bibtotoc,idxtotoc,BCOR0mm,tablecaptionabove]{scrbook}
\usepackage[utf8]{luainputenc}
\usepackage[a4paper]{geometry}
\geometry{verbose,lmargin=3.5cm,rmargin=2.5cm}
\usepackage{color}
\definecolor{gray}{rgb}{0.4,0.4,0.4}
\definecolor{darkblue}{rgb}{0.0,0.0,0.6}
\definecolor{cyan}{rgb}{0.0,0.6,0.6}
\usepackage[table,xcdraw]{xcolor}
\usepackage{babel}
\addto\shorthandsspanish{\spanishdeactivate{~<>}}

\usepackage{rotating}
\usepackage{longtable}
\usepackage{booktabs}
\usepackage{graphicx}
\usepackage[unicode=true,pdfusetitle,
 bookmarks=true,bookmarksnumbered=true,bookmarksopen=false,
 breaklinks=true,pdfborder={0 0 1},backref=false,colorlinks=true]
 {hyperref}
\hypersetup{
 linkcolor=black, citecolor=black, urlcolor=blue, filecolor=blue,pdfpagelayout=OneColumn, pdfnewwindow=true,pdfstartview=XYZ, plainpages=false, pdfpagelabels,pdftex}

\makeatletter

\setcounter{tocdepth}{4}
\setcounter{secnumdepth}{4}

%%%%%%%%%%%%%%%%%%%%%%%%%%%%%% LyX specific LaTeX commands.
\providecommand{\LyX}{\texorpdfstring%
  {L\kern-.1667em\lower.25em\hbox{Y}\kern-.125emX\@}
  {LyX}}
%% Because html converters don't know tabularnewline
\providecommand{\tabularnewline}{\\}

%%%%%%%%%%%%%%%%%%%%%%%%%%%%%% User specified LaTeX commands.
% DO NOT ALTER THIS PREAMBLE!!!
%
% This preamble is designed to ensure that the User's Guide prints
% out as advertised. If you mess with this preamble,
% parts of the User's Guide may not print out as expected.  If you
% have problems LaTeXing this file, please contact 
% the documentation team
% email: lyx-docs@lists.lyx.org

%\usepackage[utf8]{inputenc}

\usepackage{ifpdf} % part of the hyperref bundle
\ifpdf % if pdflatex is used

 % set fonts for nicer pdf view
 \IfFileExists{lmodern.sty}{\usepackage{lmodern}}{}

\fi % end if pdflatex is used

% for correct jump positions whe clicking on a link to a float
\usepackage{float}
\usepackage[figure]{hypcap}
%\usepackage[printonlyused]{acronym-custom}
\usepackage{acronym-custom}

% the pages of the TOC is numbered roman
% and a pdf-bookmark for the TOC is added
\let\myTOC\tableofcontents
\renewcommand\tableofcontents{%
  \frontmatter
  \pdfbookmark[1]{\contentsname}{}
  \myTOC
  \mainmatter }

% redefine the \LyX macro for PDF bookmarks
\def\LyX{\texorpdfstring{%
  L\kern-.1667em\lower.25em\hbox{Y}\kern-.125emX\@}
  {LyX}}

% define a short command for \textvisiblespace
\newcommand{\spce}{\textvisiblespace}

% macro for italic page numbers in the index
\newcommand{\IndexDef}[1]{\textit{#1}}

% redefine the greyed out note
%\renewenvironment{lyxgreyedout}
 %{\textcolor{blue}\bgroup}{\egroup}

% workaround for a makeindex bug, 
% see sec. "Index Entry Order"
% only uncomment this when you are using makindex
%\let\OrgIndex\index 
%\renewcommand*{\index}[1]{\OrgIndex{#1}}

\usepackage{graphicx}

\usepackage{listings}
\renewcommand{\lstlistlistingname}{Listados}
\renewcommand{\lstlistingname}{Listado}

\usepackage{hyphenat}
\usepackage{verbatim}

\AtBeginDocument{
   \renewcommand{\tablename}{Tabla}
   \renewcommand{\listtablename}{�ndice de tablas}
}

\newenvironment{changemargin}[2]{%
  \begin{list}{}{%
    \setlength{\topsep}{0pt}%
    \setlength{\leftmargin}{#1}%
    \setlength{\rightmargin}{#2}%
    \setlength{\listparindent}{\parindent}%
    \setlength{\itemindent}{\parindent}%
    \setlength{\parsep}{\parskip}%
  }%
  \item[]}{\end{list}}

\newenvironment{nota}{
  \begin{changemargin}{2em}{2em}
    \textbf{\textsc{Nota: }}
}{
  \end{changemargin}
}

\lstset{
  basicstyle=\ttfamily,
  columns=fullflexible,
  showstringspaces=false,
  commentstyle=\color{gray}\upshape
}

\lstdefinelanguage{XML}
{
  morestring=[b]",
  morestring=[s]{>}{<},
  morecomment=[s]{<?}{?>},
  stringstyle=\color{black},
  identifierstyle=\color{darkblue},
  keywordstyle=\color{cyan},
  morekeywords={xmlns,version,type,name,bpel,partnerLink,portType,operation,variable,createInstance, inputVariable, outputVariable, validate, part}% list your attributes here
}

\makeatother

\begin{document}
\begin{titlepage}
  \centering
  \includegraphics[width=.25\textwidth]{logo_uca}

  \bigskip
  \bigskip
  \bigskip
  
  \begin{changemargin}{3em}{3em}
    \centering

    {\LARGE \textsc{\nohyphens{Escuela Superior de Ingenier�a}}}
    
    \bigskip
    \bigskip
    \bigskip

    {\LARGE \nohyphens{Grado en Ingenier�a Inform�tica}}

    \bigskip
    \bigskip
    \bigskip
    \bigskip
    \bigskip
    \bigskip

    {\LARGE \nohyphens{Reingenier�a de Software para An�lisis Arm�nico de las Mareas}}

    \bigskip
    \bigskip
    \bigskip
    \bigskip
    \bigskip

    {\large Curso 2018-2019}

    \bigskip
    \bigskip
    \bigskip
    \bigskip
    \bigskip
    \bigskip
      
    \end{changemargin}

    {\Large Arantzazu Otal Alberro \\}
    \bigskip
    \bigskip 
    \bigskip 
    {\large Puerto Real, \today}

\end{titlepage}
\newpage{\pagestyle{empty}\cleardoublepage}  
{
  \thispagestyle{empty} 
  \centering
  \includegraphics[width=.2\textwidth]{logo_uca}

  \bigskip
  \bigskip
  \bigskip
  
  \begin{changemargin}{3em}{3em}

    \begin{center}
      {\LARGE \textsc{\nohyphens{Escuela Superior de Ingenier�a}}}
      
      \bigskip
      \bigskip
      
      {\LARGE \nohyphens{Grado en Ingenier�a Inform�tica}}
      
      \bigskip
      \bigskip
      \bigskip
      \bigskip
      
      {\LARGE \nohyphens{Reingenier�a de Software para An�lisis Arm�nico de Mareas}}
      
      \bigskip
      \bigskip
      \bigskip
      \bigskip
      
    \end{center}
  \end{changemargin}

  \begin{flushleft}
    \Large

    \textsc{Departamento}: \nohyphens{Ingenier�a Inform�tica.} \\
    \textsc{Directora del proyecto}: \nohyphens{M� del Carmen de Castro Cabrera} \\
    \textsc{Autor del Proyecto}: \nohyphens{Arantzazu Otal Alberro}. \\
  \end{flushleft}
  
  \bigskip
  \bigskip
  \bigskip
  
  \begin{flushright}
    \large
    Puerto Real, \today

    \bigskip    
    \bigskip
    \bigskip
    \bigskip
    \bigskip
    \bigskip
    \bigskip
    \bigskip
    Fdo.: Arantzazu Otal Alberro
    
  \end{flushright}

}


\input{Conclusiones/Agradecimientos.tex}


\clearpage
\newpage{\pagestyle{empty}\cleardoublepage}  



\tableofcontents{}

\listoffigures


\listoftables


%----------------------------


\chapter{Introducci�n\label{cap:introducci=0000F3n}}
\chapter{Introducción}
\section{Motivación}
Es un trabajo de la asignatura ``Proyectos Informáticos'' que, a nivel profesional, nos sirve para ganar experiencia laboral y enfrentarnos a situaciones reales de cara a una clientela exigente.

\section{Descripción del sistema actual}
Inicialmente, nuestra clienta contaba con una aplicación que mostraba en una página la información a cerca de un tema y los alumnos no se centraban en aprender, sino que iban directamente a hacer el cuestionario final con el objetivo de terminar antes. Esto hace que los alumnos no aprendan como es debido ni fomenten su imaginación ni su creatividad.

\section{Objetivos y alcance del proyecto}
\subsection{Objetivos}
Motivación de la creatividad y fomento de la imaginación en niños.

Para cumplir con el objetivo general, tendremos que cubrir los siguientes puntos:
\begin{itemize}
	\item Recursos de aprendizaje interactivos.
	\item Es evaluable por un profesor.
	\item Se pueden compartir fantasías entre usuarios.
	\item Es simple y manejable por alumnos de primaria.
	\item Fomenta las habilidades y enseñanzas STEM (science, technology, engeneering and maths).
\end{itemize}

\subsection{Alcance}
Los alumnos podrán crear fantasías, compartirlas y podrán ser evaluadas por los profesores, que podrán mandar como tarea el hacer fantasías.

\section{Organización del documento}
Este documento está organizado en función de las especificaciones expuestas para la presentación de un trabajo de fin de grado siguiendo los siguientes apartados:
\begin{enumerate}
	\item Introducción.
	\item Plan de proyecto.
	\item Análisis de requisitos.
	\item Diseño del sistema.
	\item Implementación del sistema.
	\item Pruebas del sistema.
	\item Manual de usuario.
	\item Manual de instalación.
	\item Conclusiones.
\end{enumerate}

Además de este documento, también contamos con un apéndice donde se narra el manual de usuario, paso a paso.


\section{Motivaci�n\label{sec:Motivacion}}
\input{Introduccion/Motivacion.tex}

\section{Objetivos\label{sec:Objetivos}}
\input{Introduccion/Objetivos.tex}

\section{Alcance\label{sec:Alcance}}
\input{Introduccion/Alcance.tex}

\section{Lenguaje WS-BPEL\label{sec:WSBPEL}}
\input{Introduccion/ws-bpel.tex}

\subsection{Actividades del lenguaje WS-BPEL\label{sec:Actividades}}
\input{Introduccion/bpelActivities.tex}

\subsection{Desaf�os en la prueba de composiciones BPEL\label{sec:DesafiosBPEL}}
\input{Introduccion/PruebaBPEL.tex}

\subsection{Casos de prueba en composiciones WS-BPEL\label{sec:CasosPruebaBPEL}}
\input{Introduccion/CasosPrueba.tex}

\section{Composici�n del tri�ngulo\label{sec:ComposicionTriangulo}}
\input{Introduccion/Triangulo.tex}

\section{JSON, JavaScript Object Notation\label{sec:JSON}}
\input{Introduccion/FormatoJSON.tex}



\section{Glosario\label{sec:Glosario}}
\input{Introduccion/Glosario.tex}

\subsection{Acr�nimos\label{sec:Acronimos}}
\input{Introduccion/Acronimos.tex}

\subsection{Definiciones\label{sec:Definiciones}}
\input{Introduccion/Definiciones.tex}

%----------------------------
\chapter{T�cnica de prueba metam�rfica\label{sec:TecnicaMetamorfica}}
\input{Metamorfica/Inicio.tex}

\section{Definici�n\label{sec:DefinicionMetamorfica}}
\input{Metamorfica/Definicion.tex}

\section{Aplicaci�n a una composici�n\label{sec:AplicacionComposicion}}
\input{Metamorfica/AplicacionComposicion.tex}

%----------------------------

%\chapter{Estado del arte\label{sec:EstadoDelArte}}
%\input{EstadoDelArte/EstadoDelArte.tex}

%----------------------------

\chapter{Desarrollo del calendario\label{cap:Calendario}}
\input{Calendario/DesarrolloCalendario.tex}

%\section{Art�culo JCIS 2016\label{sec:JCIS16}}
%\input{Calendario/JCIS16.tex}

%\section{Presentaci�n SGSOACS 2016\label{sec:SGSOACS16}}
%\input{Calendario/SGSOACS16.tex}

\section{Fases\label{sec:Fases}}
\input{Calendario/Fases.tex}

\section{Diagrama de Gantt\label{sec:Gantt}}
\input{Calendario/Gantt.tex}


%----------------------------

\chapter{Descripci�n general del proyecto\label{cap:DescripcionGeneral}}
%\input{Descripcion/DescripcionGeneral.tex}

\section{Perspectiva del producto\label{sec:Perspectiva}}
\input{Descripcion/Perspectiva.tex}

\subsection{Entorno del producto\label{sec:Entorno}}
\input{Descripcion/Entorno.tex}

\subsection{Interfaz de usuario\label{sec:Interfaz}}
\input{Descripcion/Interfaz.tex}

\section{Funciones\label{sec:Funciones}}
\input{Descripcion/Funciones.tex}

\section{Caracter�sticas del usuario\label{sec:CaracteristicasUsuario}}
\input{Descripcion/CaracteristicasUsuario.tex}

\section{Restricciones generales\label{sec:Restricciones}}
\input{Descripcion/Restricciones.tex}

\subsection{Control de versiones\label{sec:ControlVersiones}}
\input{Descripcion/ControlVersiones.tex}

\subsection{Lenguajes de programaci�n y tecnolog�as\label{sec:Tecnologias}}
\input{Descripcion/Tecnologias.tex}

\subsection{Herramientas\label{sec:Herramientas}}
\input{Descripcion/Herramientas.tex}

\subsection{Sistemas operativos y hardware\label{sec:SistemasOperativos}}
\input{Descripcion/SistemasOperativos.tex}



%----------------------------

\chapter{Desarrollo del proyecto\label{cap:DesarrolloProyecto}}
\input{Desarrollo/DesarrolloProyecto.tex}

\section{Modelo de ciclo de vida\label{sec:ModeloCiclo}}
\input{Desarrollo/ModeloCiclo.tex}

\section{Requisitos\label{sec:Requisitos}}
\chapter{Análisis de requisitos}
\section{Workspace}
El diseño gráfico usado deberá ser el mismo que el de la página de Stimey en todos los iconos usados: \url{https://stimey.eu/home}. Si necesitamos algo más, Alicia, nos lo podría facilitar.

El profesor tendrá más permisos y privilegios que el alumno, de forma que pueda crear fantasías para evaluar a sus alumnos y ellos tendrán que completarlas o crear las que el profesor les ponga como trabajo.

En el workspace tendremos una serie de opciones que estarán disponibles tanto para el rol de profesor como de alumno en función de los permisos de cada rol:
\begin{itemize}
	\item \textbf{Background:} Se abre una ventana donde se podrá seleccionar una imagen usada anteriormente, de google o del ordenador. Esta imagen cubrirá todo el workspace. También será posible introducir texto añadiéndolo manualmente o mediante un link.
	\item \textbf{Punto Activo:} Establece un punto activo en el workspace (arrastrando y soltando).
	\begin{itemize}
		\item Será posible moverlo y modificar los contenidos del mismo.
		\item Si se añade una imagen al puto activo, dicho punto, se adapta a la forma de la imagen.
		\item Una vez originado el punto activo se abre un pop\_up con un texto.
		\item También se puede asignar un vídeo, que abrirá una ventana para reproducirlo, o un audio. En caso de que no exista audio o vídeo, no se mostrará el respectivo botón.
		\item Los puntos activos podrán tener música de fondo que será silenciada si se inicia la reproducción de audio o vídeo asignados a dicho punto por el profesorado. La música será restablecida al terminar el audio o vídeo correspondiente.
		\item Los puntos activos pueden ser reorganizados por el profesorado para que emerjan en el orden que ellos quieran.
		\item Un alumno no puede continuar con el siguiente punto activo sin terminar el actual.
		\item El quiz de los puntos activos debe ser divertido.
		\item Las cuestiones planteadas en los quiz de los puntos activos deberán ser 2 y no demasiado difíciles (respuesta múltiple, escribir una palabra, quiz con imágenes y preguntas sobre ésta, unir items, etc).
		\item El quiz saldrá en pantalla cuando se cierra el punto activo actual.
		\item Una vez acabado el quiz, aparece el siguiente punto activo en el orden establecido por el profesorado en el workspace.
		\item Cada punto activo tendrá una puntuación hasta sumar entre todos un máximo de 100 puntos.
		\item Al asignar una puntuación a un punto activo, ésta se restará al total que llevemos (máximo 100 puntos). Si un punto activo es eliminado, el contador general recupera la puntuación que tenía asignada dicho punto activo.
		\item El alumno no sabe el total de puntos activos que hay en total.
		\item Cuando el alumno obtiene una puntuación al completar un punto activo, dicha cantidad se suma al contador global.
		\item Finalmente, podremos tener un resumen estadístico con las preguntas acertadas/falladas de cada punto activo.
		\item Solo se guardará la puntuación obtenida la primera vez que se realice un quiz, luego, se podrán hacer más veces, pero la nota no se registrará.
		\item El alumno tendrá la opción de guardar su progreso con un botón de guardar manualmente o mediante la opción de autoguardado.
	\end{itemize}
\end{itemize}

\section{Características}
\begin{itemize}
	\item Al finalizar todos los puntos activos habrá un botón abajo a la derecha de ``\textbf{más información}'' y en el centro un nuevo quiz que será el examen final. Este examen tendrá una puntuación independiente al de todos los puntos activos y no tendrá el resumen estadístico. Si se repite este quiz, la nota del mismo se actualizaría con un tanto por ciento de la nueva nota, más la nota anterior con el objetivo de que un alumno que repita un quiz no pueda obtener la mejor nota por repetición del mismo.
	\item El profesorado podrá mandar a los alumnos hacer fantasías para aprender como tarea. Estas tareas podrán realizarse en grupos de alumnos en función de dos ideas:
	\begin{enumerate}
		\item \textbf{Obligatoria:} Un alumno realiza la fantasía y el resto busca información adicional.
		\item \textbf{Opcional pero ideal:} Edición concurrente de la fantasía entre todos los integrantes del grupo.
	\end{enumerate}
	\item Cada fantasía tendrá un código para poder ser compartida.
	\item Tendremos dos tipos de permisos en las fantasías: ``\textbf{ver}'' y ``\textbf{ver y editar}''.
	\item La plataforma notificará al profesorado cuando los alumnos hayan terminado sus respectivos trabajos.
	\item Las fantasías podrán ser privadas o públicas. Por defecto, siempre serán públicas y podrán ser accedidas por todo el que utilice la plataforma.
	\item Las fantasías privadas podrán ser accedidas por otras personas mediante una contraseña.
	\item Las fantasías podrán ser clonadas.
\end{itemize}


\subsection{Funcionales\label{sec:Funcionales}}
\input{Desarrollo/Funcionales.tex}

\subsection{De informaci�n\label{sec:DeInformacion}}
\input{Desarrollo/DeInformacion.tex}

\subsection{Reglas de negocio\label{sec:ReglasNegocio}}
\input{Desarrollo/ReglasNegocio.tex}

\subsection{Interfaz\label{sec:Interfaz}}
\input{Desarrollo/Interfaz.tex}

\subsection{No funcionales\label{sec:NoFuncionales}}
\input{Desarrollo/NoFuncionales.tex}

\section{An�lisis del sistema\label{sec:Analisis}}
\input{Desarrollo/Analisis.tex}

\subsection{Casos de uso\label{sec:CasoUso}}
\input{Desarrollo/CasosDeUso.tex}

%\subsection{Modelo conceptual de datos del dominio\label{sec:ModeloConceptual}}
%\input{Desarrollo/ModeloConceptual.tex}

\subsection{Diagramas de secuencia\label{sec:DiagramaSecuencia}}
\input{Desarrollo/DiagramasDeSecuencia.tex}

\section{Dise�o del sistema\label{sec:DisenoSistema}}
\input{Desarrollo/DisenoSistema.tex}

\section{Implementaci�n\label{sec:Implementacion}}
\chapter{Implementación del sistema}
\section{Entorno tecnológico}
El entorno tecnológico de programación usado en el proyecto Fantasy ha sido el framework \href{https://laravel.com/}{Laravel}, el cual tiene integración con PHP y MySQL.

Gracias a PHP hemos conseguido que la página web del proyecto Fantasy sea lo más parecida a la de STIMEY, para su posterior integración en la plataforma.

Y, con MySQL, gestionamos la base de datos de las fantasías que realizan los alumnos y profesores, teniendo un control de la propiedad y acceso a las fantasías.

\section{Código fuente}
El código fuente de la aplicación se encuentra en el archivo comprimido que se adjunta en la entrega.

\section{Calidad de código}
\textcolor{red}{VER COMO SE PUEDE HACER ESTO PORQUE NI IDEA}

\section{Pruebas y validaci�n\label{sec:PruebasValidacion}}
\input{Desarrollo/PruebasValidacion.tex}

%----------------------------

\chapter{An�lisis y Obtenci�n de Informaci�n\label{sec:AnalisisTodo}}
\chapter{Introducción}
\section{Motivación}
Es un trabajo de la asignatura ``Proyectos Informáticos'' que, a nivel profesional, nos sirve para ganar experiencia laboral y enfrentarnos a situaciones reales de cara a una clientela exigente.

\section{Descripción del sistema actual}
Inicialmente, nuestra clienta contaba con una aplicación que mostraba en una página la información a cerca de un tema y los alumnos no se centraban en aprender, sino que iban directamente a hacer el cuestionario final con el objetivo de terminar antes. Esto hace que los alumnos no aprendan como es debido ni fomenten su imaginación ni su creatividad.

\section{Objetivos y alcance del proyecto}
\subsection{Objetivos}
Motivación de la creatividad y fomento de la imaginación en niños.

Para cumplir con el objetivo general, tendremos que cubrir los siguientes puntos:
\begin{itemize}
	\item Recursos de aprendizaje interactivos.
	\item Es evaluable por un profesor.
	\item Se pueden compartir fantasías entre usuarios.
	\item Es simple y manejable por alumnos de primaria.
	\item Fomenta las habilidades y enseñanzas STEM (science, technology, engeneering and maths).
\end{itemize}

\subsection{Alcance}
Los alumnos podrán crear fantasías, compartirlas y podrán ser evaluadas por los profesores, que podrán mandar como tarea el hacer fantasías.

\section{Organización del documento}
Este documento está organizado en función de las especificaciones expuestas para la presentación de un trabajo de fin de grado siguiendo los siguientes apartados:
\begin{enumerate}
	\item Introducción.
	\item Plan de proyecto.
	\item Análisis de requisitos.
	\item Diseño del sistema.
	\item Implementación del sistema.
	\item Pruebas del sistema.
	\item Manual de usuario.
	\item Manual de instalación.
	\item Conclusiones.
\end{enumerate}

Además de este documento, también contamos con un apéndice donde se narra el manual de usuario, paso a paso.


\section{Proceso de An�lisis\label{sec:ProcesoAnalisis}}
\input{Resultados/ProcesoAnalisis.tex}

\section{Resultado del An�lisis\label{sec:ResultadoAnalisis}}
\input{Resultados/ResultadoAnalisis.tex}

\subsection{Ficheros JSON\label{sec:FicherosJSON}}
\input{Resultados/JSON.tex}

\subsection{Lenguaje Natural\label{sec:LenguajeNatural}}
\input{Resultados/LenguajeNatural.tex}

%----------------------------


%\chapter{Resumen\label{cap:Resumen}}
%\newpage{\pagestyle{empty}\cleardoublepage} 
\vspace*{\fill}
\begin{center}
	\textbf{Resumen}
\end{center}
Aplicación web para fomentar el aprendizaje mediante la imaginación y creatividad de niños entre 10 y 13 años en temas científicos-tecnológicos en colaboración con el proyecto europeo STIMEY.

A modo de juego, los niños podrán crear historias interactivas y los profesores podrán evaluarlos.\\

\textbf{Palabras clave:}
Fantasía, aprendizaje, desarrollo, ilusiona, entretenimiento, creatividad, cuestionario, evalua-
ción, enseñanza, ciencia, unión europea.
\vspace*{\fill}

\newpage


%----------------------------

\chapter{Conclusiones y Trabajo Futuro\label{cap:ConclusionesFuturo}}
\input{Conclusiones/ConclusionesFuturo.tex}

\section{Valoraci�n\label{sec:Valoracion}}
\input{Conclusiones/Valoracion.tex}

\section{Objetivos cumplidos\label{sec:ObjetivosCumplidos}}
\input{Conclusiones/ObjetivosCumplidos.tex}

\section{Lecciones aprendidas\label{sec:LeccionesAprendidas}}
\input{Conclusiones/LeccionesAprendidas.tex}

\section{Trabajo futuro\label{sec:Futuro}}
\input{Conclusiones/Futuro.tex}

%----------------------------


\appendix

\chapter{Composiciones Estudiadas\label{cap:ComposicionesEstudiadas}}
\input{Anexos/ComposicionesUtilizadas.tex}

\chapter{Manual de instalaci�n\label{cap:ManualInstalacion}}
\input{Anexos/ManualInstalacion.tex}

\chapter{Manual de usuario\label{cap:ManualUsuario}}
\chapter{Manual de usuario}
Aquí va el manual de usuario.

\section{Introducción}

\section{Características}

\section{Requisitos previos}

\section{Utilización}

\chapter{Manual del desarrollador\label{cap:ManualDesarrollador}}
\input{Anexos/ManualDesarrollador.tex}

\chapter{GNU Free Documentation License\label{cap:LicenciaGNU}}
\input{Anexos/GNULicense.tex}
%----------------------------

\bibliographystyle{splncs03}
\bibliography{referencias}

\end{document}
