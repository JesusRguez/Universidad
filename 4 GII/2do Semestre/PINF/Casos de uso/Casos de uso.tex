%%\documentclass[a4paper,12pt,oneside]{llncs}
\documentclass[12pt,letterpaper]{article}
\usepackage[right=2cm,left=3cm,top=2cm,bottom=2cm,headsep=0cm]{geometry}

%%%%%%%%%%%%%%%%%%%%%%%%%%%%%%%%%%%%%%%%%%%%%%%%%%%%%%%%%%%
%% Juego de caracteres usado en el archivo fuente: UTF-8
\usepackage{ucs}
\usepackage[utf8x]{inputenc}

%%%%%%%%%%%%%%%%%%%%%%%%%%%%%%%%%%%%%%%%%%%%%%%%%%%%%%%%%%%
%% Juego de caracteres usado en la salida dvi
%% Otra posibilidad: \usepackage{t1enc}
\usepackage[T1]{fontenc}

%%%%%%%%%%%%%%%%%%%%%%%%%%%%%%%%%%%%%%%%%%%%%%%%%%%%%%%%%%%
%% Ajusta maergenes para a4
%\usepackage{a4wide}

%%%%%%%%%%%%%%%%%%%%%%%%%%%%%%%%%%%%%%%%%%%%%%%%%%%%%%%%%%%
%% Uso fuente postscript times, para que los ps y pdf queden y pequeños...
\usepackage{times}

%%%%%%%%%%%%%%%%%%%%%%%%%%%%%%%%%%%%%%%%%%%%%%%%%%%%%%%%%%%
%% Posibilidad de hipertexto (especialmente en pdf)
%\usepackage{hyperref}
\usepackage[bookmarks = true, colorlinks=true, linkcolor = black, citecolor = black, menucolor = black, urlcolor = black]{hyperref}

%%%%%%%%%%%%%%%%%%%%%%%%%%%%%%%%%%%%%%%%%%%%%%%%%%%%%%%%%%%
%% Graficos 
\usepackage{graphics,graphicx}

%%%%%%%%%%%%%%%%%%%%%%%%%%%%%%%%%%%%%%%%%%%%%%%%%%%%%%%%%%%
%% Ciertos caracteres "raros"...
\usepackage{latexsym}

%%%%%%%%%%%%%%%%%%%%%%%%%%%%%%%%%%%%%%%%%%%%%%%%%%%%%%%%%%%
%% Matematicas aun más fuertes (american math dociety)
\usepackage{amsmath}

%%%%%%%%%%%%%%%%%%%%%%%%%%%%%%%%%%%%%%%%%%%%%%%%%%%%%%%%%%%
\usepackage{multirow} % para las tablas
\usepackage[spanish,es-tabla]{babel}

%%%%%%%%%%%%%%%%%%%%%%%%%%%%%%%%%%%%%%%%%%%%%%%%%%%%%%%%%%%
%% Fuentes matematicas lo mas compatibles posibles con postscript (times)
%% (Esto no funciona para todos los simbolos pero reduce mucho el tamaño del
%% pdf si hay muchas matamaticas....
\usepackage{mathptm}

%%% VARIOS:
%\usepackage{slashbox}
\usepackage{verbatim}
\usepackage{array}
\usepackage{listings}
\usepackage{multirow}

%% MARCA DE AGUA
%% Este package de "draft copy" NO funciona con pdflatex
%%\usepackage{draftcopy}
%% Este package de "draft copy" SI funciona con pdflatex
%%%\usepackage{pdfdraftcopy}
%%%%%%%%%%%%%%%%%%%%%%%%%%%%%%%%%%%%%%%%%%%%%%%%%%%%%%%%%%%
%% Indenteacion en español...
\usepackage[spanish]{babel}

\usepackage{listings}
% Para escribir código en C
% \begin{lstlisting}[language=C]
% #include <stdio.h>
% int main(int argc, char* argv[]) {
% puts("Hola mundo!");
% }
% \end{lstlisting}


\title{Casos de uso}
\author{Luis Gutiérrez Flores\\
	Nicolás Ruiz Requejo\\
	Jesús Rodríguez Heras\\
	Arantzazu Otal Alberro\\
	Alejandro Segovia Gallardo\\
	Alejandro José Caraballo García\\
	Gabriel Fernando Sánchez Reina}

\begin{document}
	
	\maketitle
%	\begin{abstract} %Poner esto en todas las prácticas de PCTR
%%		\begin{center}
%%			\noindent
%			Descripción de los casos de uso 
%%		\end{center}
%	\end{abstract}
	\thispagestyle{empty}
	\newpage
	
	\tableofcontents
	\newpage
	
	%%\listoftables
	%%\newpage
	
	%%\listoffigures
	%%\newpage
	
	%%%% REAL WORK BEGINS HERE:
	
	%%Configuracion del paquete listings
	\lstset{language=bash, numbers=left, numberstyle=\tiny, numbersep=10pt, firstnumber=1, stepnumber=1, basicstyle=\small\ttfamily, tabsize=1, extendedchars=true, inputencoding=latin1}

\section{Crear fantasía}
\begin{itemize}
	\item \textbf{Descripción:} Crea una nueva fantasía.
	\item \textbf{Actores:} Profesor o alumno (usuario).
	\item \textbf{Precondiciones:} Estar logeado en la plataforma de Stimey.
	\item \textbf{Postcondiciones:} La fantasía queda almacenada y lista para ser modificada, compartida, etc.
	\item \textbf{Escenario principal:}
	\begin{enumerate}
		\item El sistema solicita el identificador de usuario para saber a qué rol pertenece.
		\item El usuario introduce el identificador.
		\item Include (Autorizar conexión).
		\item El usuario selecciona la opción ``Crear nueva fantasía''.
		\begin{enumerate}
			\item 
		\end{enumerate}
		\item El sistema solicita el nombre de la fantasía.
		\item El usuario introduce el nombre de la fantasía.
		\item El sistema da a elegir si la fantasía será pública (por defecto) o privada.
		\item La fantasía queda almacenada en el sistema y se abre el menú de creación.
	\end{enumerate}
	\item \textbf{Extensiones:} \\7. a) El usuario selecciona que la fantasía será privada, en lugar de pública.
	\begin{enumerate}
		\item El sistema muestra permite insertar en una lista los identificadores de otros usuarios con los que quedará compartida la fantasía. Si no introduce ninguno, la fantasía queda almacenada de forma privada, solo para dicho usuario.
	\end{enumerate}
	\item \textbf{Variaciones:} Ninguna.
	\item \textbf{No-funcional:} Ninguna.
	\item \textbf{Cuestiones:} Ninguna.
\end{itemize}

\section{Elegir idioma}
\begin{itemize}
	\item \textbf{Descripción:} Cambia el idioma de la aplicación.
	\item \textbf{Actores:} Profesor o alumno (usuario).
	\item \textbf{Precondiciones:} Ninguna. %el usuario es capaz de encontrar el menú de idioma con la mirada
	\item \textbf{Postcondiciones:} La aplicación cambia al idioma seleccionado por el usuario.
	\item \textbf{Escenario principal:}
	\begin{enumerate}
		\item El usuario pulsa el botón de cambio de idioma.
		\item El sistema despliega una lista de los idiomas disponibles.
		\item El usuario  selecciona un idioma de los que están disponibles en el sistema.
		\item La aplicación cambia el idioma.
	\end{enumerate}
	\item \textbf{Extensiones:} Ninguna.
	\item \textbf{Variaciones:} Ninguna.
	\item \textbf{No-funcional:} Ninguna.
	\item \textbf{Cuestiones:} Ninguna.
\end{itemize}

\section{Insertar imagen} %suponemos que es para un punto activo
\begin{itemize}
	\item \textbf{Descripción:} Inserta una imagen en un punto activo.
	\item \textbf{Actores:} Profesor o alumno (usuario).
	\item \textbf{Precondiciones:} Debe existir el punto activo correspondiente y se debe estar editando la fantasía.
	\item \textbf{Postcondiciones:} Inserta una imagen en el punto activo seleccionado.
	\item \textbf{Escenario principal:}
	\begin{enumerate}
		\item El usuario selecciona la opción de insertar imagen.
		\item El sistema muestra una ventana en la que da a elegir al usuario de donde quiere seleccionar la imagen (Internet, local, imagen ya usada en la fantasía).
		\item El usuario elige la opción ``Internet'' para incluir una imagen de internet.
		\item El sistema le pide al usuario la url de la imagen.
		\item El usuario inserta la url correcta de la imagen.
		\item El punto activo toma la forma de la imagen.
	\end{enumerate}
	\item \textbf{Extensiones:} \\3. a) El usuario elige la opción ``Local'' para incluir una imagen desde su ordenador.
	\begin{enumerate}
		\item El sistema abre una ventana del explorador de archivos.
		\item El usuario selecciona la imagen deseada y pulsa ``Aceptar''.
		\item El sistema cierra la ventana del explorador de archivos.
		\item Paso 6.
	\end{enumerate}
	3. b) El usuario elige la opción ``Imagen usada anteriormente'' para incluir una imagen ya usada.
	\begin{enumerate}
		\item El sistema abre una ventana con las imágenes usadas anteriormente.
		\item El usuario selecciona la imagen deseada y pulsa ``Aceptar''.
		\item El sistema cierra la ventana emergente.
		\item Paso 6.
	\end{enumerate}
	5. a) La url no es correcta.
	\begin{enumerate}
		\item El sistema muestra un mensaje de error.
		\item Paso 4.
	\end{enumerate}
	\item \textbf{Variaciones:} Ninguna.
	\item \textbf{No-funcional:} Ninguna.
	\item \textbf{Cuestiones:} ¿Podrá modificar el tamaño original de la imagen o hacer recortes?
\end{itemize}

\section{Insertar texto} %suponemos que es para un punto activo
\begin{itemize}
	\item \textbf{Descripción:} Inserta un texto en un punto activo.
	\item \textbf{Actores:} Profesor o alumno (usuario).
	\item \textbf{Precondiciones:} Debe existir el punto activo correspondiente y se debe estar editando la fantasía.
	\item \textbf{Postcondiciones:} Inserta un texto en el punto activo seleccionado.
	\item \textbf{Escenario principal:}
	\begin{enumerate}
		\item El usuario selecciona la opción de insertar texto.
		\item El sistema muestra una ventana emergente para que el usuario inserte el texto deseado.
		\item El usuario inserta el texto en la ventana emergente mostrada por el sistema. Cuando termina, pulsa en el botón ``Aceptar''.
		\item El sistema guarda el texto en el punto activo correspondiente.
	\end{enumerate}
	\item \textbf{Extensiones:} Ninguna.
	\item \textbf{Variaciones:} Ninguna.
	\item \textbf{No-funcional:} Ninguna.
	\item \textbf{Cuestiones:} ¿Podrá tener opciones de formato?
\end{itemize}

\section{Crear cuestionario pequeño}
\begin{itemize}
	\item \textbf{Descripción:} Crea un pequeño cuestionario sobre el tema del que trata el punto activo.
	\item \textbf{Actores:} Profesor o alumno (usuario).
	\item \textbf{Precondiciones:} Estar logueado en la plataforma Stimey.
	\item \textbf{Postcondiciones:} Crea un pequeño cuestionario en relación al punto activo correspondiente.
	\item \textbf{Escenario principal:}
	\begin{enumerate}
		\item El usuario selecciona la opción de crear un cuestionario pequeño.
		\item El sistema muestra las posibles opciones.
		\item El usuario selecciona ``Respuesta simple''.
		\item El sistema muestra una ventana emergente para crear la pregunta con sus posibles respuestas.
		\item El usuario rellena la ventana emergente con la pregunta y las respuestas convenientes y pulsa ``Aceptar'' cuando termina.
		\item El sistema cierra la ventana emergente.
		\item El cuestionario queda registrado en el punto activo seleccionado.
	\end{enumerate}
	\item \textbf{Extensiones:} \\3. a) El usuario elige la opción ``Palabra''.
	\begin{enumerate}
		\item El sistema abre una ventana emergente para crear la pregunta y su respuesta.
		\item El usuario rellena la ventana emergente con la pregunta y la respuesta conveniente y pulsa ``Aceptar'' cuando termina.
		\item El sistema cierra la ventana emergente.
		\item Paso 7.
	\end{enumerate}
	3. b) El usuario elige la opción ``Quiz con imágenes''.
	\begin{enumerate}
		\item El sistema abre una ventana emergente para crear la pregunta con la imagen y su respuesta.
		\item El usuario rellena la ventana emergente con la pregunta, la imagen y la respuesta conveniente, y pulsa ``Aceptar'' cuando termina.
		\item El sistema cierra la ventana emergente.
		\item Paso 7.
	\end{enumerate}
	3. c) El usuario elige la opción ``Unir''.
	\begin{enumerate}
		\item El sistema abre una ventana emergente para crear el quiz de unión.
		\item El usuario rellena la ventana emergente con las posibles respuestas y su respuesta correcta y pulsa ``Aceptar'' cuando termina.
		\item El sistema cierra la ventana emergente.
		\item Paso 7.
	\end{enumerate}
	\item \textbf{Variaciones:} Ninguna.
	\item \textbf{No-funcional:} Ninguna.
	\item \textbf{Cuestiones:} Ninguna.
\end{itemize}

\section{Asignar puntos activos}
\begin{itemize}
	\item \textbf{Descripción:} 
	\item \textbf{Actores:}
	\item \textbf{Precondiciones:}
	\item \textbf{Postcondiciones:}
	\item \textbf{Escenario principal:}
	\item \textbf{Extensiones:}
	\item \textbf{Variaciones:}
	\item \textbf{No-funcional:}
	\item \textbf{Cuestiones:}
\end{itemize}

\section{Insertar contenido}
\begin{itemize}
	\item \textbf{Descripción:}
	\item \textbf{Actores:}
	\item \textbf{Precondiciones:}
	\item \textbf{Postcondiciones:}
	\item \textbf{Escenario principal:}
	\item \textbf{Extensiones:}
	\item \textbf{Variaciones:}
	\item \textbf{No-funcional:}
	\item \textbf{Cuestiones:}
\end{itemize}

\section{Crear cuestionario final}
\begin{itemize}
	\item \textbf{Descripción:}
	\item \textbf{Actores:}
	\item \textbf{Precondiciones:}
	\item \textbf{Postcondiciones:}
	\item \textbf{Escenario principal:}
	\item \textbf{Extensiones:}
	\item \textbf{Variaciones:}
	\item \textbf{No-funcional:}
	\item \textbf{Cuestiones:}
\end{itemize}

\section{Autorizar conexión}
\begin{itemize}
	\item \textbf{Descripción:}
	\item \textbf{Actores:}
	\item \textbf{Precondiciones:}
	\item \textbf{Postcondiciones:}
	\item \textbf{Escenario principal:}
	\item \textbf{Extensiones:}
	\item \textbf{Variaciones:}
	\item \textbf{No-funcional:}
	\item \textbf{Cuestiones:}
\end{itemize}

\section{Gestionar porcentaje punto activo}
\begin{itemize}
	\item \textbf{Descripción:}
	\item \textbf{Actores:}
	\item \textbf{Precondiciones:}
	\item \textbf{Postcondiciones:}
	\item \textbf{Escenario principal:}
	\item \textbf{Extensiones:}
	\item \textbf{Variaciones:}
	\item \textbf{No-funcional:}
	\item \textbf{Cuestiones:}
\end{itemize}

\section{Gestionar ficha alumno}
\begin{itemize}
	\item \textbf{Descripción:}
	\item \textbf{Actores:}
	\item \textbf{Precondiciones:}
	\item \textbf{Postcondiciones:}
	\item \textbf{Escenario principal:}
	\item \textbf{Extensiones:}
	\item \textbf{Variaciones:}
	\item \textbf{No-funcional:}
	\item \textbf{Cuestiones:}
\end{itemize}

\section{Asignar nota final}
\begin{itemize}
	\item \textbf{Descripción:}
	\item \textbf{Actores:}
	\item \textbf{Precondiciones:}
	\item \textbf{Postcondiciones:}
	\item \textbf{Escenario principal:}
	\item \textbf{Extensiones:}
	\item \textbf{Variaciones:}
	\item \textbf{No-funcional:}
	\item \textbf{Cuestiones:}
\end{itemize}

\section{Asignar fantasía}
\begin{itemize}
	\item \textbf{Descripción:}
	\item \textbf{Actores:}
	\item \textbf{Precondiciones:}
	\item \textbf{Postcondiciones:}
	\item \textbf{Escenario principal:}
	\item \textbf{Extensiones:}
	\item \textbf{Variaciones:}
	\item \textbf{No-funcional:}
	\item \textbf{Cuestiones:}
\end{itemize}

\end{document}